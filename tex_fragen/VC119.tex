\frage{VC119}
    {Was gilt hinsichtlich der Störfestigkeit der Amateurfunkstelle nach dem Wortlaut des Amateurfunkgesetzes (AFuG)? }
    {Der Funkamateur darf von den grundlegenden Anforderungen nach dem Gesetz über die elektromagnetische Verträglichkeit von Betriebsmitteln (EMVG) abweichen und kann den Grad der Störfestigkeit seiner Amateurfunkstelle selbst bestimmen.}
    {Der Funkamateur muss seine Amateurfunkstelle im Abstand von 2 Jahren einer Störfestigkeitsprüfung durch die BNetzA unterziehen lassen.}
    {Amateurfunkstellen sind hinsichtlich ihrer Störfestigkeit anderen Betriebsmitteln gleichgestellt.}
    {Amateurfunkstellen müssen elektromagnetische Störungen durch andere Betriebsmittel hinnehmen, selbst wenn diese nicht den grundlegenden Anforderungen nach dem Gesetz über die elektromagnetische Verträglichkeit von Betriebsmitteln (EMVG) entsprechen.}
    {false}{false}