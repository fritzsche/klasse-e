\frage{EK207}
    {Wie lassen sich elektrostatische Aufladungen, die insbesondere bei ungeerdeten Drahtantennen auftreten können, wirkungsvoll vermeiden, ohne die Funktion der Funkanlage zu beeinträchtigen?}
    {Durch hochohmige Ableitwiderstände zwischen den Anschlüssen an der Antenne und dem Erdanschluss der Amateurfunkstelle.}
    {Durch niederohmige Ableitwiderstände zwischen den Anschlüssen an der Antenne und dem Erdanschluss der Amateurfunkstelle.}
    {Das Einschleifen eines Anpassgerätes zwischen Transceiver und Antenne neutralisiert die Aufladungen.}
    {Mit Hilfe der Abblockkondensatoren in einem zwischengeschalteten Stehwellenmessgerät.}
    {false}{false}