\frage{AG429}
    {Wodurch können Mantelwellen im Falle einer koax-gespeisten symmetrischen Antenne auftreten, obwohl ein Spannungs-Balun verwendet wird?}
    {Ungleichmäßige Belastung der Antenne durch Störeinflüsse der Umgebung (z. B. Bäume oder Gebäude) sowie Einkopplung in den Koax-Schirm}
    {Fehlanpassung durch Impedanztransformation des Baluns (z. B. 4:1-Spartransformator) sowie Stehwellen in der Zuleitung}
    {Dämpfung der Abstrahlung durch als Oberwellenfilter wirkenden Balun (z. B. 1:1-Transformator) sowie Einkopplung in den Koax-Schirm}
    {Erhitzung des Ringkerns durch unzureichende Abschirmung (z. B. Kunststoffgehäuse) des Baluns sowie Stehwellen in der Zuleitung}
    {false}{false}