\frage{AH216}
    {Wie erkennt ein Funkamateur in der Regel, dass er mit "PY" auf dem indirekten und somit längeren Weg gearbeitet hat?}
    {Aus der Stellung seiner Richtantenne erkennt er, dass diese der Richtung des kürzesten Weges nach Brasilien um 180 ° entgegengesetzt ist. Das heißt, er hat "PY" auf dem "langen Weg" gearbeitet.}
    {Durch die verhallte Tonlage der Verbindung erkennt er, dass diese in zwei Richtungen nach Brasilien stattgefunden hat. Das heißt, er hat "PY" nicht nur direkt, sondern auf einem längeren Weg gearbeitet.}
    {Aus der Stellung seiner Richtantenne erkennt er, dass diese in Richtung des längeren Weges nach Brasilien eingesetzt ist. Das heißt, er hat "PY" auf dem direkten Weg gearbeitet.}
    {Durch die verhallte Tonlage der Verbindung nach Brasilien, Ausbreitung der Funkwellen über zwei entgegengesetzte Wege.}
    {false}{false}