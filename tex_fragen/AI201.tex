\frage{AI201}
    {Wie funktioniert ein vektorieller Netzwerkanalysator (VNA)? Ein HF-Generator erzeugt ein ...}
    {frequenzveränderliches HF-Signal, mit dem z. B. ein Filter oder eine Antenne beaufschlagt wird. Die durch das angeschlossene Messobjekt veränderten Amplituden und Phasen des HF-Signals werden als Verläufe von z. B. Impedanz und Phasenwinkel, Wirk- und Blindanteil oder dem Stehwellenverhältnis grafisch dargestellt.}
    {frequenzstabiles HF-Signal, mit dem z. B. ein Filter oder eine Antenne beaufschlagt wird. Die durch das angeschlossene Messobjekt erzeugten Strom- und Spannungsbäuche werden als Verläufe von z. B. Impedanz und Phasenwinkel, Wirk- und Blindanteil oder dem Stehwellenverhältnis grafisch dargestellt.}
    {frequenzveränderliches HF-Signal, mit dem z. B. ein Filter oder eine Antenne beaufschlagt wird. Aus den durch das Messobjekt entstehenden Spannungseinbrüchen wird der Scheinwiderstand des Messobjektes ermittelt.}
    {frequenzstabiles HF-Signal, mit dem  z. B. ein Filter oder eine Antenne beaufschlagt wird. Aus der durch das Messobjekt entstehenden Fehlanpassung werden Dämpfungsverlauf oder Antennengewinn ermittelt.}
    {false}{false}