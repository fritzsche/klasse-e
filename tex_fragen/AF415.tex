\frage{AF415}
    {Weshalb wurden jeweils \$C\_1\$ und \$C\_2\$, \$C\_3\$ und \$C\_4\$ sowie \$C\_5\$ und \$C\_6\$ parallel geschaltet?}
    {Der Kondensator geringer Kapazität dient jeweils zum Abblocken hoher Frequenzen, der Kondensator hoher Kapazität zum Abblocken niedriger Frequenzen.}
    {Die Kapazität nur eines Kondensators reicht bei hohen Frequenzen nicht aus.}
    {Der Kondensator mit der geringen Kapazität dient zur Siebung der niedrigen und der Kondensator mit der hohen Kapazität zur Siebung der hohen Frequenzen.}
    {Zu einem Elektrolytkondensator muss immer ein keramischer Kondensator parallel geschaltet werden, weil er sonst bei hohen Frequenzen zerstört werden würde.}
    {true}{false}