\frage{AK103}
    {In welchem Fall hat die folgende Formel zur Berechnung des Sicherheitsabstandes Gültigkeit und was sollten Sie tun, wenn die Gültigkeit nicht mehr sichergestellt ist? \textbackslash\{\}[d = \textbackslash\{\}frac\{\textbackslash\{\}sqrt\{30 Ohm\textbackslash\{\}cdot P\_\{\textbackslash\{\}textrm\{EIRP\}\}\}\}\{E\}\textbackslash\{\}]}
    {Die Formel gilt nur für Abstände \$d > \textbackslash\{\}frac\{\textbackslash\{\}lambda\}\{2\textbackslash\{\}cdot\textbackslash\{\}pi\}\$ bei den meisten Antennenformen (z. B. Dipol-Antennen). Für Antennen, die z. B. geometrisch klein im Verhältnis zur Wellenlänge sind und/oder in kürzerem Abstand zur Antenne muss der Sicherheitsabstand zum Beispiel durch Feldstärkemessungen oder Nahfeldberechnungen (Simulationen) ermittelt werden.}
    {Im Bereich von Amateurfunkstellen ist der Unterschied zwischen Nah- und Fernfeld so gering, dass obige Formel, die eigentlich nur im Fernfeld gilt, trotzdem für alle Raumbereiche verwendet werden kann.}
    {Die Formel gilt nur für Abstände \$d > \textbackslash\{\}frac\{\textbackslash\{\}lambda\}\{2\textbackslash\{\}cdot\textbackslash\{\}pi\}\$ bei horizontal polarisierten Antennen.
Bei kleineren Abständen und immer bei vertikal polarisierten Antennen muss der Sicherheitsabstand durch zum Beispiel Feldstärkemessungen oder Nahfeldberechnungen (Simulationen) ermittelt werden.}
    {Die Formel gilt nur für Abstände \$d > \textbackslash\{\}frac\{\textbackslash\{\}lambda\}\{2\textbackslash\{\}cdot\textbackslash\{\}pi\}\$ bei vertikal polarisierten Antennen.
Bei kleineren Abständen und immer bei horizontal polarisierten Antennen muss der Sicherheitsabstand durch zum Beispiel Feldstärkemessungen oder Nahfeldberechnungen (Simulationen) ermittelt werden.}
    {false}{false}