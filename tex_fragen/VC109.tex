\frage{VC109}
    {Welches Recht haben Funkamateure in Bezug auf den Betrieb von Sendeanlagen? Ein Funkamateur ...}
    {ist berechtigt, im Handel erhältliche, selbst gefertigte oder auf Amateurfunkfrequenzen umgebaute Sendeanlagen zu betreiben.}
    {benötigt in keinem Fall eine Standortbescheinigung der BNetzA für seine Amateurfunkstelle.}
    {muss die einschlägigen Bestimmungen der BNetzA zur elektrischen Sicherheit nicht beachten.}
    {darf mit seiner Amateurfunkstelle jederzeit Nachrichten für und an Dritte übermitteln, die nicht den Amateurfunkdienst betreffen.}
    {false}{false}