\frage{NB304}
    {Welche Polarisationen unterscheidet man üblicherweise bei der Funkwellenausbreitung im Amateurfunk und wieso sollte man diese beachten?}
    {Man unterscheidet horizontale, vertikale sowie links- und rechtszirkulare Polarisation. Die Polarisation von Sende- und Empfangsantenne sollten angeglichen sein, um eine verlustarme Übertragung zu gewährleisten.}
    {Man unterscheidet transversale, longitudinale und orthogonale Polarisation. Die Polarisation des Funkgeräts muss an das Stromnetz angepasst sein, um Kurzschlüsse zu vermeiden.}
    {Man unterscheidet kohärente, inkohärente und korrelierte Polarisation. Die Polarisation der Funkwellen sollte regelmäßig geändert werden, um die Störfestigkeit zu erhöhen.}
    {Man unterscheidet parallele, koaxiale und drahtlose Polarisation. Die Polarisation der Antennenkabel muss auf die Antennen abgestimmt sein, um Verluste zu minimieren.}
    {false}{false}