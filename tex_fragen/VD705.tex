\frage{VD705}
    {Wie ist ein sekundärer Funkdienst laut Amateurfunkverordnung (AFuV) definiert?}
    {Ein sekundärer Funkdienst ist ein Funkdienst, dessen Funkstellen weder Störungen bei den Funkstellen eines primären Funkdienstes verursachen dürfen noch Schutz vor Störungen durch solche Funkstellen verlangen können.}
    {Ein sekundärer Funkdienst ist ein Funkdienst, dessen Frequenzzuteilung zeitlich später erfolgte. Die Einteilung bedeutet nicht, dass der sekundäre Funkdienst dem primären Funkdienst nachgeordnet ist.}
    {Ein sekundärer Funkdienst muss Störungen durch andere hinnehmen und kann die Störungen nicht an die Funkstörungsannahme der Bundesnetzagentur melden.}
    {Die Unterteilung in primäre und sekundäre Funkdienste gilt nur für kommerzielle Funkstellen oder Funkstellen von Behörden und Organisationen mit Sicherheitsaufgaben.}
    {false}{false}