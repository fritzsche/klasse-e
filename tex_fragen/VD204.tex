\frage{VD204}
    {Warum ist ""DL250BTHVN"" ein zulässiges deutsches Amateurfunkrufzeichen?}
    {Weil der Rufzeichenplan zu besonderen allgemeinen Anlässen auch Rufzeichen mit bis zu 7 Zeichen langem Suffix vorsieht, der Ziffern enthalten kann und mit einem Buchstaben endet.}
    {Weil für besonders verdiente Funkamateure auch personengebundene Rufzeichen ausgegeben werden, für die der Rufzeichenplan keine Anwendung findet.}
    {Weil an bestimmte öffentliche Stellen, wie z. B. Kunst- und Kultureinrichtungen, besondere Rufzeichen mit mindestens 3 Ziffern ausgegeben werden.}
    {Weil dies in einer Sonderverfügung der Bundesnetzagentur aufgrund besonderen historischen Anlass mit internationaler Wirkung festgelegt wurde.}
    {false}{false}