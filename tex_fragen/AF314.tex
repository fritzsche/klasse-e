\frage{AF314}
    {Ein quarzgesteuertes Funkgerät mit einer Ausgangsfrequenz von 432 MHz verursacht Störungen bei 144 MHz. Der Quarzoszillator des Funkgeräts schwingt auf einer Grundfrequenz bei 12 MHz.  Bei welcher Vervielfachungskombination kann die Störfrequenz von 144 MHz auftreten?  }
    {Grundfrequenz \$\textbackslash\{\}cdot 2 \textbackslash\{\}cdot 2 \textbackslash\{\}cdot 3 \textbackslash\{\}cdot 3\$}
    {Grundfrequenz \$\textbackslash\{\}cdot 2 \textbackslash\{\}cdot 3 \textbackslash\{\}cdot 3 \textbackslash\{\}cdot 2\$}
    {Grundfrequenz \$\textbackslash\{\}cdot 3 \textbackslash\{\}cdot 3 \textbackslash\{\}cdot 2\textbackslash\{\}cdot 2\$}
    {Grundfrequenz \$\textbackslash\{\}cdot 3 \textbackslash\{\}cdot 2 \textbackslash\{\}cdot 3 \textbackslash\{\}cdot 2\$}
    {false}{false}