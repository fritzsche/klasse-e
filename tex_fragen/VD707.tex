\frage{VD707}
    {Das 80 m-Amateurfunkband ist unter anderem dem Amateurfunkdienst und dem Seefunkdienst auf primärer Basis zugewiesen. Unter welchen Umständen dürfen Sie in einer Amateurfunkverbindung fortfahren, wenn Sie erst nach Betriebsaufnahme bemerken, dass Ihre benutzte Frequenz auch von einer Küstenfunkstelle benutzt wird?}
    {Sie dürfen die Frequenz unter keinen Umständen weiterbenutzen (außer im echten Notfall), da der Küstenfunkstelle eine feste Frequenz zugeteilt ist, die sie nicht verändern kann.}
    {Sie dürfen die Frequenz weiter benutzen, wenn aus der dauernd wiederholten, automatisch ablaufenden Morseaussendung klar hervorgeht, dass die Küstenfunkstelle keinen zweiseitigen Funkverkehr abwickelt, sondern offenbar nur die Frequenz belegt.}
    {Sie dürfen die Frequenz weiter benutzen, wenn der Standort Ihrer Amateurfunkstelle mehr als 200 km von einer Meeresküste entfernt ist und Sie weniger als 100 W Sendeleistung anwenden.}
    {Sie dürfen die begonnene Funkverbindung mit Ihrer Gegenfunkstelle solange fortführen, bis Sie von der Küstenfunkstelle zum Frequenzwechsel aufgefordert werden.}
    {false}{false}