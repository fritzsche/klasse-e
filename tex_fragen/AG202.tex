\frage{AG202}
    {Warum muss eine Antenne mechanisch etwas kürzer als der theoretisch errechnete Wert sein?}
    {Weil sich diese Antenne nicht im idealen freien Raum befindet und weil sie nicht unendlich dünn ist. Kapazitive Einflüsse der Umgebung und der Durchmesser des Strahlers verlängern die Antenne elektrisch.}
    {Weil sich diese Antenne nicht im idealen freien Raum befindet und weil die Antennenelemente nicht die Idealform des Kugelstrahlers besitzen. Kapazitive Einflüsse der Umgebung und die Abweichung von der idealen Kugelform verlängern die Antenne elektrisch.}
    {Weil sich durch die mechanische Verkürzung die elektromagnetischen Wellen leichter von der Antenne ablösen. Dadurch steigt der Wirkungsgrad.}
    {Weil sich durch die mechanische Verkürzung der Verlustwiderstand eines Antennenstabes verringert. Dadurch steigt der Wirkungsgrad.}
    {false}{false}