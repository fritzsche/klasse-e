\frage{AB108}
    {Das folgende Bild zeigt den prinzipiellen Aufbau einer Halbleiterdiode. Wie entsteht die Sperrschicht?}
    {An der Grenzschicht wandern Elektronen aus dem N-Teil in den P-Teil. Dadurch wird auf der N-Seite der Elektronenüberschuss teilweise abgebaut, auf der P-Seite der Elektronenmangel teilweise neutralisiert. Es bildet sich auf beiden Seiten der Grenzfläche eine isolierende Schicht.}
    {An der Grenzschicht wandern Elektronen aus dem P-Teil in den N-Teil. Dadurch wird auf der P-Seite der Elektronenüberschuss teilweise abgebaut, auf der N-Seite der Elektronenmangel teilweise neutralisiert. Es bildet sich auf beiden Seiten der Grenzfläche eine isolierende Schicht.}
    {An der Grenzschicht wandern Atome aus dem P-Teil in den N-Teil. Dadurch wird auf der P-Seite der Atommangel abgebaut, auf der N-Seite der Atommangel vergrößert. Es bildet sich auf beiden Seiten der Grenzfläche eine leitende Schicht.}
    {An der Grenzschicht wandern Atome aus dem N-Teil in den P-Teil. Dadurch wird auf der N-Seite der Atommangel abgebaut, auf der P-Seite der Atommangel vergrößert. Es bildet sich auf beiden Seiten der Grenzfläche eine leitende Schicht.}
    {true}{false}