\frage{AG224}
    {Welche Eigenschaften besitzt eine in geringer Höhe aufgebaute, auf Kurzwelle betriebene NVIS-Antenne (Near Vertical Incident Skywave)?}
    {Sie ermöglicht durch annähernd senkrechte Abstrahlung eine Raumwellenausbreitung ohne tote Zone um den Sendeort herum.}
    {Sie vergrößert durch ihre flache Abstrahlung den Bereich der Bodenwelle.}
    {Ihre senkrechte Abstrahlung bringt die D-Region zum Verschwinden, so dass die Tagesdämpfung über dem Sendeort lokal aufgehoben wird.}
    {Sie erzeugt mit ihrer Reflexion am nahen Erdboden eine zirkular polarisierte Abstrahlung, die Fading reduziert.}
    {false}{false}