\frage{VE306}
    {Durch den Betrieb einer Amateurfunkstelle auf 144,250 MHz wird der Kabelfernsehempfang eines Nachbarn beeinträchtigt. Eine Überprüfung ergibt, dass der Funkamateur am Ort der beeinträchtigten Empfangsanlage eine Feldstärke erzeugt, die den in der Norm empfohlenen Grenzwert für die Störfestigkeit von Kabelverteilanlagen nicht erreicht. Was folgt daraus für den Funkamateur?}
    {Er kann seinen Funkbetrieb fortsetzen.}
    {Er hat den Betrieb seiner Amateurfunkstelle einzustellen.}
    {Er hat seine Sendeleistung so einzurichten, dass der Empfang nicht mehr beeinträchtigt wird.}
    {Er kann seine Sendeleistung uneingeschränkt erhöhen.}
    {false}{false}