\frage{AD113}
    {Die Spannung an der Brückenschaltung beträgt 11 V. Die Widerstände haben folgende Werte: $R\_1$ = 1 kOhm; $R\_2$ = 10 kOhm; $R\_3$ = 10 kOhm; $R\_4$ = 1 kOhm. Wie groß ist die Spannung zwischen A und B im Brückenzweig (gemessen von A nach B)?}
    {$U\_\{AB\} = 9 V$ }
    {$U\_\{AB\} = -9 V$}
    {$U\_\{AB\} = 10 V$}
    {$U\_\{AB\} = -10 V$}
    {true}{false}