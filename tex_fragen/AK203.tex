\frage{AK203}
    {Ihr 400 W-Kurzwellensender ist über eine separate Erdungsleitung mit dem Potentialausgleich Ihres Hauses verbunden. Im Sendebetrieb stellen Sie fest, dass auf bestimmten Bändern das Gehäuse des Senders ""heiß"" ist, d. h. Hochfrequenzspannung merklicher Amplitude auf dem Gerätegehäuse liegt. Was kann die Ursache hierfür sein?}
    {Die Länge der Erdleitung entspricht annähernd einem Viertel der Wellenlänge der
Sendefrequenz oder einem ungeraden Vielfachen davon.}
    {Die verwendete Kupfer-Erdleitung ist nicht versilbert und somit zur guten Ableitung von Hochfrequenz nicht geeignet.}
    {Die Länge der Erdleitung entspricht annähernd einer halben Wellenlänge der Sendefrequenz oder Vielfachen davon.}
    {Für die verwendete Erdleitung wurde ein massiver Leiter anstatt einer für Hochfrequenz besser geeigneten mehradrigen Litze verwendet.}
    {false}{false}