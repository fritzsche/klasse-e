\frage{EB504}
    {An einem Widerstand $R$ wird die elektrische Leistung $P$ in Wärme umgesetzt. Sie kennen die Größen $P$ und $R$. Nach welcher der Formeln können Sie die Spannung ermitteln, die an dem Widerstand $R$ anliegt?}
    {$U = \sqrt{P\cdot R}$}
    {$U = R\cdot P$}
    {$U = \sqrt{\dfrac{P}{R}}$}
    {$U = \dfrac{P}{R}$}
    {false}{false}