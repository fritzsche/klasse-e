\frage{EB504}
    {An einem Widerstand \$R\$ wird die elektrische Leistung \$P\$ in Wärme umgesetzt. Sie kennen die Größen \$P\$ und \$R\$. Nach welcher der Formeln können Sie die Spannung ermitteln, die an dem Widerstand \$R\$ anliegt?}
    {\$U = \textbackslash\{\}sqrt\{P\textbackslash\{\}cdot R\}\$}
    {\$U = R\textbackslash\{\}cdot P\$}
    {\$U = \textbackslash\{\}sqrt\{\textbackslash\{\}dfrac\{P\}\{R\}\}\$}
    {\$U = \textbackslash\{\}dfrac\{P\}\{R\}\$}
    {false}{false}