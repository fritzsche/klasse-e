\frage{EI206}
    {Sie ermitteln die Resonanzfrequenz und die Impedanz ihrer selbstgebauten Antennen mit Hilfe eines vektoriellen Netzwerkanalysators (VNA). Wie könnten Sie die Funktion des Gerätes vorher prüfen?}
    {Durch Prüfen der Anzeigewerte in den Betriebszuständen Kurzschluss, Leerlauf und Anpassung. Das SWR sollte bei Anpassung nahe bei 1, bei Kurzschluss und Leerlauf unendlich sein.}
    {Durch Prüfen der Anzeigewerte in den Betriebszuständen Leerlauf und Anpassung. Der Messanschluss des Gerätes darf keinesfalls kurzgeschlossen werden.}
    {Durch Beschalten des Messeingangs am VNA mit einem Abschlusswiderstand. Das angezeigte SWR sollte im gesamten Frequenzbereich größer als 2 sein.}
    {Durch Beschalten des Messeingangs am VNA mit einem Blindwiderstand. Der Anzeigewert des SWR muss bei allen Frequenzen nahe bei 1 sein.}
    {false}{false}