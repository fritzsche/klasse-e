\frage{NK304}
    {Welche Maßnahme ist nach einem Elektrounfall mit Körperdurchströmung (Stromschlag) zu ergreifen?}
    {Es ist ein Arzt aufzusuchen, da Herzrhythmusstörungen und Herzkammerflimmern auch noch viele Stunden nach einem Stromschlag auftreten können.}
    {Personen, die einen Stromschlag erlitten haben, sind unverzüglich in eine stabile Seitenlage zu bringen.}
    {Sofern sich die verunfallte Person gut fühlt, sind keine Maßnahmen erforderlich.}
    {Bei Stromschlag mit Wechselstrom (AC) ist ein Arzt aufzusuchen, bei Stromschlag mit Gleichstrom (DC) ist kein Arzt erforderlich.}
    {false}{false}