\frage{AG403}
    {In den Eingang einer Antennenleitung mit einer Dämpfung von 3 dB werden 10 W HF-Leistung eingespeist. Mit der am Leitungsende angeschlossenen Antenne misst man am Leitungseingang ein SWR von 3. Mit einer künstlichen 50 Ohm-Antenne am Leitungsende beträgt das SWR am Leitungseingang etwa 1. Was lässt sich aus diesen Messergebnissen schließen?}
    {Die Antenne ist fehlerhaft. Sie strahlt so gut wie keine HF-Leistung ab.}
    {Die Antennenleitung ist fehlerhaft, an der Antenne kommt so gut wie keine HF-Leistung an.}
    {Die Antennenanlage ist in Ordnung. Es werden etwa 5 W HF-Leistung abgestrahlt.}
    {Die Antennenanlage ist in Ordnung. Es werden etwa 3,75 W HF-Leistung abgestrahlt.}
    {false}{false}