\frage{AB104}
    {Was versteht man unter Halbleitermaterialien?}
    {Einige Stoffe (z. B. Silizium) sind in reinem Zustand bei Raumtemperatur gute Isolatoren. Durch geringfügige Zusätze von geeigneten anderen Stoffen (z. B. Bor, Phosphor) oder bei hohen Temperaturen werden sie jedoch zu Leitern.}
    {Einige Stoffe (z. B. Silizium) sind in reinem Zustand bei Raumtemperatur gute Leiter. Durch geringfügige Zusätze von geeigneten anderen Stoffen (z. B. Bor, Phosphor) oder bei hohen Temperaturen nimmt jedoch ihre Leitfähigkeit ab.}
    {Einige Stoffe (z. B. Silizium) sind in reinem Zustand bei Raumtemperatur gute Leiter. Durch geringfügige Zusätze von geeigneten anderen Stoffen (z. B. Bismut, Tellur) fällt ihr Widerstand auf den halben Wert.}
    {Einige Stoffe (z. B. Silizium) sind in reinem Zustand bei Raumtemperatur gute Elektrolyten. Durch geringfügige Zusätze von geeigneten anderen Stoffen (z. B. Bismut, Tellur) kann man daraus entweder N-leitendes- oder P-leitendes Material für Anoden bzw. Kathoden von Batterien herstellen.}
    {false}{false}