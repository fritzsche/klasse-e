\frage{EH214}
    {Ein plötzlicher Anstieg der Intensitäten von UV- und Röntgenstrahlung nach einem Flare (Energieausbruch auf der Sonne) führt zu erhöhter Ionisierung der D-Region und damit zu zeitweiligem Ausfall der Raumwellenausbreitung auf der Kurzwelle. Diese Erscheinung bezeichnet man als ...}
    {Mögel-Dellinger-Effekt.}
    {sporadische E-Ausbreitung.}
    {kritischer Schwund.}
    {Aurora-Effekt.}
    {false}{false}