\frage{AD613}
    {Welche Bedingungen müssen zur Erzeugung ungedämpfter Schwingungen in Oszillatoren erfüllt sein?}
    {Das an einem Schaltungspunkt betrachtete Oszillatorsignal muss auf dem Signalweg im Oszillator so verstärkt und phasengedreht werden, dass es wieder gleichphasig und mit mindestens der gleichen Amplitude zum selben Punkt zurückgekoppelt wird.}
    {Die Grenzfrequenz des verwendeten Verstärkerelements muss mindestens der Schwingfrequenz des Oszillators entsprechen, und das entstehende Eingangssignal muss über den Rückkopplungsweg wieder gegenphasig zum Eingang zurückgeführt werden.}
    {Die Schleifenverstärkung des Signalwegs im Oszillator muss kleiner als 1 sein, und das entstehende Oszillatorsignal darf auf dem Rückkopplungsweg nicht in der Phase gedreht werden.}
    {Die Schleifenverstärkung des Signalwegs im Oszillator muss größer als 1 sein, und das Ausgangssignal muss über den Rückkopplungsweg in der Phase so gedreht werden, dass es gegenphasig zum Ausgangspunkt zurückgeführt wird.}
    {false}{false}