\frage{AG126}
    {Für die Erzeugung von zirkularer Polarisation mit Yagi-Uda-Antennen wird eine horizontale und eine dazu um 90 ° um die Strahlungsachse gedrehte Yagi-Uda-Antenne zusammengeschaltet. Was ist dabei zu beachten, damit tatsächlich zirkulare Polarisation entsteht?}
    {Bei einer der Antennen muss die Welle um \$\textbackslash\{\}lambda\$/4 verzögert werden. Dies kann entweder durch eine zusätzlich eingefügte Viertelwellen-Verzögerungsleitung oder durch mechanische ""Verschiebung"" beider Yagi-Uda-Antennen um \$\textbackslash\{\}lambda\$/4 gegeneinander hergestellt werden.}
    {Bei einer der Antennen muss die Welle um \$\textbackslash\{\}lambda\$/2 verzögert werden. Dies kann entweder durch eine zusätzlich eingefügte \$\textbackslash\{\}lambda\$/2-Verzögerungsleitung oder durch mechanische ""Verschiebung"" beider Yagi-Uda-Antennen um \$\textbackslash\{\}lambda\$/2 gegeneinander hergestellt werden.}
    {Die Zusammenschaltung der Antennen muss über eine Halbwellen-Lecherleitung erfolgen. Zur Anpassung an den Wellenwiderstand muss zwischen der Speiseleitung und den Antennen noch ein \$\textbackslash\{\}lambda\$/4-Transformationsstück eingefügt werden.}
    {Die kreuzförmig angeordneten Elemente der beiden Antennen sind um 45 ° zu verdrehen, so dass in der Draufsicht ein liegendes Kreuz gebildet wird. Die Antennen werden über Leitungsstücke gleicher Länge parallel geschaltet. Die Anpassung erfolgt mit einem Symmetrierglied.}
    {false}{false}