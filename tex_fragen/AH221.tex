\frage{AH221}
    {Massiv erhöhte UV- und Röntgenstrahlung, wie sie vor allem durch starke Sonneneruptionen hervorgerufen wird, beeinflusst in der Ionosphäre vor allem ...}
    {die D-Region, die die Kurzwellen-Signale dann so massiv dämpft, dass keine Ausbreitung über die Raumwelle mehr möglich ist.}
    {die F2-Region, die dann so stark ionisiert wird, dass fast die gesamte KW-Ausstrahlung reflektiert wird.}
    {die E-Region, die dann für die höheren Frequenzen durchlässiger wird und durch Refraktion (Brechung) in der F2-Region für gute Ausbreitungsbedingungen sorgt.}
    {die F1-Region, die durch Absorption der höheren Frequenzen die Refraktion (Brechung) an der F2-Region behindert.}
    {false}{false}