\frage{VD118}
    {Wie ist der Begriff ""Relaisfunkstelle"" nach dem Wortlaut der Amateurfunkverordnung (AFuV) definiert?}
    {Eine ""Relaisfunkstelle"" ist eine fernbediente oder automatisch arbeitende Amateurfunkstelle oder fernbedient und automatisch arbeitende Amateurfunkstelle in Satelliten, die empfangene Amateurfunk-Aussendungen, Teile davon oder sonstige eingespeiste oder eingespeicherte Inhalte fernausgelöst wieder aussendet oder weiterleitet.}
    {Eine ""Relaisfunkstelle"" ist eine von einem Funkamateur ständig besetzte Amateurfunkstelle, die empfangene Amateurfunkaussendungen, Teile davon oder sonstige eingespeiste oder eingespeicherte Signale fern ausgelöst aussendet und dabei zur Erhöhung der Erreichbarkeit von Amateurfunkstellen dient.}
    {Eine ""Relaisfunkstelle"" ist eine fernbediente Amateurfunkstelle, die mit dem persönlichen Rufzeichen des Inhabers betrieben wird.}
    {Eine ""Relaisfunkstelle"" ist eine fernbediente Amateurfunkstelle, die nur an einem geografisch exponierten Standort betrieben werden darf.}
    {false}{false}