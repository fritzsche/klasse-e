\frage{AF633}
    {Was bildet der I- bzw. der Q-Anteil eines I/Q-Signals ab?}
    {Die phasengleichen (I) bzw. die um 90 ° phasenverschobenen (Q) Anteile eines Signals in Bezug auf eine Referenzschwingung}
    {Den Wechselstrom (I) in Abhängigkeit der Güte (Q) eines Schwingkreises bei seiner Resonanzfrequenz}
    {Den Stromanteil (I) und den Blindleistungsanteil (Q) eines Signals}
    {Die erste (I) bzw. die vierte (Q) Harmonische in Bezug auf ein normiertes Rechtecksignal}
    {false}{false}