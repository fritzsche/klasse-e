\frage{AG428}
    {Die Darstellung zeigt die bei Ankopplung eines Koaxialkabels an eine Antenne auftretenden Ströme. Wie kann man den als $I\_3$ bezeichneten, unerwünschten Mantelstrom reduzieren?}
    {Einfügen einer Gleichtaktdrossel oder bei symmetrischen Antennen auch eines Spannungs-Baluns}
    {Einfügen eines Oberwellenfilters oder bei unsymmetrischen Störeinflüssen auch eines Spannungs-Baluns}
    {Auftrennen des Koax-Schirms vom Arm 2 der dargestellten Antenne (direkt an oder kurz vor der Antenne)}
    {Herstellung einer direkten Verbindung zwischen dem Arm 1 der Antenne mit einer guten HF-Erde}
    {true}{false}