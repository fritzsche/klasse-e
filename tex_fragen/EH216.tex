\frage{EH216}
    {Was ist mit der Aussage ""Funkverkehr über den langen Weg (long path)"" gemeint?}
    {Die Funkverbindung läuft nicht über den direkten Weg zur Gegenstation, sondern über die dem kürzesten Weg entgegengesetzte Richtung.}
    {Bei guten Ausbreitungsbedingungen treten mehrfache Refraktionen (Brechungen) mit vielen Sprüngen (hops) auf. Dann ist es möglich, sehr weite Entfernungen - ""lange Wege"" - zu überbrücken.}
    {Bei guten Ausbreitungsbedingungen treten mehrfache Refraktionen (Brechungen) mit vielen Sprüngen (hops) auf. Sie hören dann Ihre eigenen Zeichen zeitverzögert als ""Echo"" im Empfänger wieder. Sie laufen also den ""langen Weg einmal um die Erde"".}
    {Bei sehr guten Ausbreitungsbedingungen liegen die reflektierenden Regionen in großer Höhe. Die Sprungdistanzen werden dann sehr groß, so dass sie die Reichweite der Bodenwelle um ein Vielfaches übertreffen. Dann kann man mit einem Sprung einen ""sehr langen Weg"" zurücklegen.}
    {false}{false}