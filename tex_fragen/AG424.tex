\frage{AG424}
    {Zur Anpassung von Antennen werden häufig Umwegleitungen verwendet. Wie arbeitet die folgende Schaltung?}
    {Der \$\textbackslash\{\}lambda\$/2-Faltdipol hat an jedem seiner Anschlüsse eine Impedanz von 120 Ohm gegen Erde. Durch die \$\textbackslash\{\}lambda\$/2-Umwegleitung erfolgt eine 1:1-Widerstandstransformation mit Phasendrehung um 180 °. An der Seite der Antennenleitung erfolgt eine phasenrichtige Parallelschaltung von 2 mal 120 Ohm gegen Erde, womit eine Ausgangsimpedanz von 60 Ohm erreicht wird.
}
    {Der \$\textbackslash\{\}lambda\$/2-Faltdipol hat eine Impedanz von 240 Ohm. Durch die \$\textbackslash\{\}lambda\$/2-Umwegleitung erfolgt eine Widerstandstransformation von 4:1 mit Phasendrehung um 360 °, womit an der Seite der Antennenleitung eine Ausgangsimpedanz von 60 Ohm erreicht wird.}
    {Der \$\textbackslash\{\}lambda\$/2-Dipol hat eine Impedanz von 60 Ohm. Durch die \$\textbackslash\{\}lambda\$/2-Umwegleitung erfolgt eine Widerstandstransformation von 1:2 mit Phasendrehung um 180 °. An der Seite der Antennenleitung erfolgt eine phasenrichtige Parallelschaltung von 2 mal 120 Ohm gegen Erde, womit eine Ausgangsimpedanz von 60 Ohm erreicht wird.}
    {Der \$\textbackslash\{\}lambda\$/2-Dipol hat eine Impedanz von 240 Ohm. Durch die \$\textbackslash\{\}lambda\$/2-Umwegleitung erfolgt eine Widerstandstransformation von 4:1 mit Phasendrehung um 360 °, womit an der Seite der Antennenleitung eine Ausgangsimpedanz von 60 Ohm erreicht wird.}
    {true}{false}