\frage{AF230}
    {Sie empfangen das Signal eines Satelliten auf 10 GHz. Die Kabellänge zwischen LNB und Empfänger beträgt 20 m. Warum ist die Kabeldämpfung trotz der hohen Empfangsfrequenz eher vernachlässigbar? }
    {Der LNB verstärkt das Empfangssignal und mischt dieses auf eine niedrigere Frequenz, auf der die Kabeldämpfung geringer ist. }
    {Durch die Fernspeisespannung, die den LNB versorgt, sinkt die Kabeldämpfung.}
    {Durch die Mischung des Empfangssignals mit der TCXO-Frequenz wird nur noch das Basisband übertragen. }
    {Der LNB demoduliert das Signal. Die entstehende NF ist unempfindlich gegen Kabeldämpfung.}
    {true}{false}