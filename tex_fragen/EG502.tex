\frage{EG502}
    {Nach welcher der Antworten kann die EIRP berechnet werden?}
    {\$P\_\{\textbackslash\{\}textrm\{EIRP\}\} = (P\_\{\textbackslash\{\}textrm\{Sender\}\} - P\_\{\textbackslash\{\}textrm\{Verluste\}\}) \textbackslash\{\}cdot G\_\{\textbackslash\{\}textrm\{Antenne\}\}\$, bezogen auf einen isotropen Strahler}
    {\$P\_\{\textbackslash\{\}textrm\{EIRP\}\} = (P\_\{\textbackslash\{\}textrm\{Sender\}\} \textbackslash\{\}cdot P\_\{\textbackslash\{\}textrm\{Verluste\}\}) \textbackslash\{\}cdot G\_\{\textbackslash\{\}textrm\{Antenne\}\}\$, bezogen auf einen Halbwellendipol}
    {\$P\_\{\textbackslash\{\}textrm\{EIRP\}\} = (P\_\{\textbackslash\{\}textrm\{Sender\}\} - P\_\{\textbackslash\{\}textrm\{Verluste\}\}) + G\_\{\textbackslash\{\}textrm\{Antenne\}\}\$, bezogen auf einen isotropen Strahler}
    {\$P\_\{\textbackslash\{\}textrm\{EIRP\}\} = (P\_\{\textbackslash\{\}textrm\{Sender\}\} - P\_\{\textbackslash\{\}textrm\{Verluste\}\}) + G\_\{\textbackslash\{\}textrm\{Antenne\}\}\$, bezogen auf einen Halbwellendipol}
    {false}{false}