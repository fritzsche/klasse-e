\frage{VE514}
    {Was hat ein Funkamateur zu beachten, nachdem er seine ortsfeste Amateurfunkstelle bei der Bundesnetzagentur gemäß BEMFV angezeigt hat?}
    {Er hat eine Dokumentation über die Einhaltung der Anforderungen mit allen erforderlichen Unterlagen bereitzuhalten und fortlaufend zu prüfen, ob die Bedingungen, unter denen die Anzeige durchgeführt wurde, noch zutreffend sind. Bei wesentlichen Änderungen ist die Amateurfunkstelle erneut anzuzeigen.}
    {Mit der Anzeige seiner ortsfesten Amateurfunkstelle ist ein Funkamateur seinen Verpflichtungen zum Schutz von Personen in elektromagnetischen Feldern nachgekommen und muss diesbezüglich nichts weiter beachten.}
    {Das Anzeigeverfahren ist jedes Jahr erneut durchzuführen, um die Aktualität zu gewährleisten.}
    {Nachdem die ortsfeste Amateurfunkstelle in Betrieb genommen wurde, ist die Dokumentation über die Einhaltung der Anforderungen mit allen erforderlichen Unterlagen der zuständigen Außenstelle der Bundesnetzagentur vorzulegen.}
    {false}{false}