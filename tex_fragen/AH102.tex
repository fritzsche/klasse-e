\frage{AH102}
    {Der solare Flux F ...}
    {ist die gemessene Energieausstrahlung der Sonne im GHz-Bereich. Fluxwerte über 100 führen zu einem stark erhöhten Ionisationsgrad in der Ionosphäre und zu einer erheblich verbesserten Fernausbreitung auf den höheren Kurzwellenbändern.}
    {ist die gemessene Energieausstrahlung der Sonne im Kurzwellenbereich. Fluxwerte über 60 führen zu einem stark erhöhten Ionisationsgrad in der Ionosphäre und zu einer erheblich verbesserten Fernausbreitung auf den höheren Kurzwellenbändern.}
    {wird aus der Sonnenfleckenrelativzahl R abgeleitet und ist ein Indikator für die Aktivität der Sonne. Fluxwerte über 100 führen zu einem stark erhöhten Ionisationsgrad der D-Region und damit zu einer erheblichen Verschlechterung der Fernausbreitung auf den Kurzwellenbändern.}
    {wird aus der Sonnenfleckenrelativzahl R abgeleitet und ist ein Indikator für die Aktivität der Sonne. Fluxwerte über 60 führen zu einem stark erhöhten Ionisationsgrad in der Ionosphäre und zu einer erheblich verbesserten Fernausbreitung auf den höheren Kurzwellenbändern.}
    {false}{false}