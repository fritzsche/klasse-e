\frage{VC118}
    {Was muss ein Funkamateur beim Betrieb seiner Amateurfunkstelle in Bezug auf die elektromagnetische Verträglichkeit beachten?}
    {Der Funkamateur muss die Schutzanforderungen zur Gewährleistung der elektromagnetischen Verträglichkeit im Sinne des Gesetzes über die elektromagnetische Verträglichkeit von Betriebsmitteln (EMVG) einhalten.}
    {Der Funkamateur benötigt für seine Amateurfunkstelle eine aktuelle  Verträglichkeitsbescheinigung der BNetzA.
}
    {Die Amateurfunkstelle darf nur aus baumustergeprüften Funkgeräten bestehen, die den Anforderungen des Gesetzes über Funkanlagen (FuAG) entsprechen.}
    {Die Amateurfunkstelle muss von einem zertifizierten Elektromeister auf die Einhaltung der elektromagnetischen Verträglichkeit geprüft werden. Das Abnahmeprotokoll ist für die BNetzA bereitzuhalten.
}
    {false}{false}