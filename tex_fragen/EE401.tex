\frage{EE401}
    {Welcher Unterschied besteht zwischen der Bandbreite und der Datenübertragungsrate?}
    {Als Bandbreite wird der genutzte Frequenzbereich (in Hz) und als Datenübertragungsrate die je Zeiteinheit übertragene Datenmenge (in Bit/s) bezeichnet.}
    {Als Bandbreite wird die übertragene Datenmenge (in Hz) und als Datenübertragungsrate die je Zeiteinheit übertragenen Symbole (in Baud) bezeichnet. }
    {Die Datenübertragungsrate (in Bit/s) entspricht der Symbolrate (in Baud). Die Bandbreite (in Hz) entspricht der maximal möglichen Datenübertragungsrate (in Bit/s).}
    {Die Datenübertragungsrate (in Baud) entspricht der Symbolrate (in Bit/s). Die Bandbreite (in Hz) entspricht der minimal möglichen Datenübertragungsrate (in Baud).}
    {false}{false}