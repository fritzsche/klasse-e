\frage{VE201}
    {Darf ein Funkamateur seine Amateurfunkstelle zum Abhören des nichtöffentlich gesprochenen Wortes verwenden?}
    {Das Abhören des nichtöffentlich gesprochenen Wortes ist ein Straftatbestand.}
    {Der Funkamateur gilt als sachkundige Person und darf daher selbst entscheiden, auf welchen Frequenzen er hören darf.}
    {Das Abhören des nichtöffentlich gesprochenen Wortes durch den Funkamateur bedarf einer besonderen Zulassung der BNetzA. }
    {Der Funkamateur ist dazu berechtigt, wenn er dazu technisch zugelassene Empfänger benutzt.}
    {false}{false}