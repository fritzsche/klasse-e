\frage{AD220}
    {Wie ergibt sich die Bandbreite $B$ eines Parallelschwingkreises aus der Resonanzkurve?}
    {Die Bandbreite ergibt sich aus der Differenz der beiden Frequenzen, bei denen die Spannung auf den 0,7-fachen Wert gegenüber der maximalen Spannung bei der Resonanzfrequenz abgesunken ist.}
    {Die Bandbreite ergibt sich aus der Differenz der beiden Frequenzen, bei denen die Spannung auf den 0,5-fachen Wert gegenüber der maximalen Spannung bei der Resonanzfrequenz abgesunken ist.}
    {Die Bandbreite ergibt sich aus der Multiplikation der Resonanzfrequenz mit dem Faktor 0,5.}
    {Die Bandbreite ergibt sich aus der Multiplikation der Resonanzfrequenz mit dem Faktor 0,7.}
    {false}{false}