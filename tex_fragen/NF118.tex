\frage{NF118}
    {Was wird unter einem Digipeater verstanden?}
    {Eine Funkstation, die empfangene Datenpakete oder Teile davon automatisch erneut aussendet, ggf. auch zeitversetzt oder wiederholt. Hierbei können einzelne Datenfelder geändert werden.}
    {Ein Lineartransponder, der empfangene Datenpakete auf ein anderes Frequenzband umsetzt. Hierbei bleiben die verwendete Modulationsart sowie der Inhalt des Pakets erhalten.}
    {Ein integrierter Schaltkreis, der digitale Signale für die Modulation im Funkgerät vorbereitet. Hierbei wird das Rufzeichen der Station regelmäßig in den Datenstrom eingefügt.}
    {Eine Relaisstation, die Sprachübertragungen auf einer anderen Frequenz erneut aussendet. Hierbei wird die Lautstärke adaptiv mittels digitaler Signalverarbeitung angepasst.}
    {false}{false}