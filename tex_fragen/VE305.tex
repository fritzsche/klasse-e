\frage{VE305}
    {Durch den Betrieb einer Amateurfunkstelle auf 145,550 MHz wird der UKW-Rundfunkempfänger eines Nachbarn durch Direkteinstrahlung beeinträchtigt. Eine Überprüfung ergibt, dass der Funkamateur am Ort des beeinträchtigten Empfängers eine Feldstärke erzeugt, die den in der Norm empfohlenen Grenzwert für die Störfestigkeit von Geräten nicht erreicht. Was folgt daraus für den Funkamateur?}
    {Er kann seinen Funkbetrieb fortsetzen.}
    {Er hat seine Sendeleistung so einzurichten, dass der Empfang nicht mehr beeinträchtigt wird.}
    {Er kann seine Sendeleistung uneingeschränkt erhöhen.}
    {Er hat den Betrieb seiner Amateurfunkstelle einzustellen.}
    {false}{false}