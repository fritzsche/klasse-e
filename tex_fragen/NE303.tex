\frage{NE303}
    {Welche Auswirkung hat Frequenzmodulation (FM) auf die Amplitude des Sendesignals?}
    {Idealerweise hat das Modulationssignal keine Auswirkung auf die Amplitude des Sendesignals.}
    {Idealerweise entspricht die Amplitude des Sendesignals der Amplitude des Modulationssignals.}
    {Je schneller die Schwingung des Modulationssignals ist, umso größer wird die Amplitude des Sendesignals.}
    {Je größer die Amplitude des Modulationssignals ist, umso größer wird die Amplitude des Sendesignals.}
    {false}{false}