\frage{VD206}
    {Welches Buchstabieralphabet ist nach der Verfügung 13/2005 bei der Nennung des Rufzeichens zur Identifikation einer Amateurfunkstation zu verwenden?}
    {Das internationale Buchstabieralphabet nach den Radio Regulations (Anhang 14)}
    {Das europäische Buchstabieralphabet von 1992}
    {Das englische Buchstabieralphabet der ITU-Konferenz in Madrid von 1932}
    {Das deutsche Buchstabieralphabet nach DIN 5009}
    {false}{false}