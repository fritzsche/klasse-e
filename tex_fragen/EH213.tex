\frage{EH213}
    {Bei der Ausbreitung auf Kurzwelle spielt die so genannte ""Greyline"" eine besondere Rolle. Was ist die ""Greyline""?}
    {Die Zone der Dämmerung um Sonnenauf- und -untergang herum.}
    {Die instabilen Ausbreitungsbedingungen in der Äquatorialzone.}
    {Die Zeit mit den besten Möglichkeiten für ""Short-Skip""-Ausbreitung.}
    {Die Übergangszeit vor und nach dem Winter, in der sich die D-Region ab- und wieder aufbaut.}
    {false}{false}