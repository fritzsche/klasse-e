\frage{AH215}
    {Eine Aussendung auf dem 20 m-Band kann von der Funkstelle A in einer Entfernung von 1500 km, nicht jedoch von der Funkstelle B in 60 km Entfernung empfangen werden. Der Grund hierfür ist, dass ...}
    {die Funkstelle B die Bodenwelle nicht mehr und die Raumwelle noch nicht empfangen kann.}
    {die Boden- und Raumwellen sich bei Funkstelle B gegenseitig aufheben.}
    {zwei in verschiedenen ionosphärischen Regionen reflektierte Wellen mit auslöschender Phase bei Funkstelle B eintreffen.}
    {bei Funkstelle B der Mögel-Dellinger-Effekt aufgetreten ist.}
    {false}{false}