\frage{EB502}
    {Die mittlere Leistung eines Senders ist ...}
    {die durchschnittliche Leistung, die ein Sender unter normalen Betriebsbedingungen an die Antennenspeiseleitung während eines Zeitintervalls abgibt, das im Verhältnis zur Periode der tiefsten Modulationsfrequenz ausreichend lang ist.}
    {die unmittelbar nach dem Senderausgang messbare Leistung über die Spitzen der Periode einer durchschnittlichen Hochfrequenzschwingung, bevor Zusatzgeräte (z. B. Anpassgeräte) durchlaufen werden.}
    {die durchschnittliche Leistung, die ein Sender unter normalen Betriebsbedingungen während einer Periode der Hochfrequenzschwingung bei der höchsten Spitze der Modulationshüllkurve der Antennenspeiseleitung zuführt.}
    {das Produkt aus der Leistung, die unmittelbar der Antenne zugeführt wird, und ihrem Gewinnfaktor in einer Richtung, bezogen auf den Halbwellendipol.}
    {false}{false}