\frage{EK105}
    {Sie möchten den Personenschutz-Sicherheitsabstand für ihren neuen, fest aufgebauten Halbwellendipol für das 80 m-Band (3,5 - 3,8 MHz) bestimmen. Bei 100 W Sendeleistung errechnen Sie mit Hilfe der Näherungsformel für die Fernfeldberechnung einen erforderlichen Abstand von 3,65 m. Ist dieser Sicherheitsabstand gültig?}
    {Der errechnete Abstand ist ungültig, da er im reaktiven Nahfeld der Antenne liegt, und muss deshalb durch andere Methoden wie z. B. Messungen der E- und H-Feldanteile, Simulations- oder Nahfeldberechnungen bestimmt werden.}
    {Der errechnete Personenschutz-Sicherheitsabstand ist gültig, da Berechnungen mit der Näherungsformel für die Fernfeldberechnung im Amateurfunk hinreichend genau sind.}
    {Der errechnete Personenschutz-Sicherheitsabstand muss erst noch mit einem Sicherheitszuschlag (\$\textbackslash\{\}sqrt\{2\}\$) multipliziert werden.}
    {Der errechnete Personenschutz-Sicherheitsabstand ist akzeptiert, sofern die vor Inbetriebnahme einzureichende ""Anzeige ortsfester Amateurfunkanlagen"" gemäß § 9 BEMFV von der Bundesnetzagentur nicht beanstandet wird.}
    {false}{false}