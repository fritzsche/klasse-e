\frage{NF113}
    {Warum befinden sich bei Satellitenbetrieb Up- und Downlink in der Regel nicht im gleichen Frequenzband? Man benutzt unterschiedliche Frequenzbänder, weil ...}
    {dies eine einfachere Trennung von Sende- und Empfangssignal ermöglicht und die Baugröße von Filtern auf dem Satelliten reduziert wird.}
    {der Uplink durch die Ionosphäre stärker bedämpft wird als der Downlink.}
    {die Bandbreite auf beiden Frequenzbändern aufgeteilt wird und Bandbereiche besser ausgenutzt werden können. }
    {man damit den Dopplereffekt vermindert.}
    {false}{false}