\frage{BE401}
    {Was ist damit gemeint, wenn man sagt, die Relaisfunkstelle hat eine Eingabe- und eine Ausgabefrequenz?}
    {Die Relaisfunkstelle empfängt auf der Eingabefrequenz und sendet auf der Ausgabefrequenz.}
    {Die Relaisfunkstelle stellt bei starker Belegung der Eingabefrequenz eine zusätzliche Ausgabefrequenz zur Verfügung.}
    {Die Relaisfunkstelle benutzt eine Eingabefrequenz zur Umsetzung des empfangenen Signals und die Ausgabefrequenz zur Fernsteuerung.}
    {Die Relaisfunkstelle muss auf der Ausgabefrequenz mit einem Tonruf geöffnet werden, bevor sie auf der Eingabefrequenz in Betrieb gehen kann.}
    {false}{false}