\frage{EJ218}
    {Wie sollte bei digitalen Übertragungsverfahren (z. B. FT8, JS8, PSK31) der NF-Pegel am Eingang eines Funkgerätes mit automatischer Pegelregelung (ALC) im SSB-Betrieb eingestellt sein, um Störungen zu vermeiden?}
    {So niedrig, dass die automatische Pegelregelung (ALC) nicht eingreift.}
    {18 dB höher als die Lautstärke, bei der die automatische Pegelregelung (ALC) eingreift.}
    {Alle Bedienelemente sind auf das Maximum einzustellen.}
    {Die NF-Lautstärke muss $-\infty$ dB (also Null) betragen.}
    {false}{false}