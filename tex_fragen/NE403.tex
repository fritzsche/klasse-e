\frage{NE403}
    {Ist es bei bestimmten digitalen Verfahren zur Sprachübertragung (z. B. DMR oder TETRA) möglich, mehrere Sprechverbindungen gleichzeitig auf derselben Frequenz innerhalb eines Empfangsgebiets abzuwickeln?}
    {Ja. Die Sprachdaten werden abwechselnd in periodischen, kurzen Zeitschlitzen übertragen.}
    {Ja. Die Sendeleistung wird zur Verbesserung der digitalen Fehlerkorrektur erhöht.}
    {Nein. Zeitgleich stattfindende digitale Übertragungen stören sich prinzipbedingt gegenseitig.}
    {Nein. Sprachübertragungen können nicht in Datenpakete aufgeteilt werden.}
    {false}{false}