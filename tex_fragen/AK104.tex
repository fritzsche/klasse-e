\frage{AK104}
    {Wie errechnen Sie die Leistung am Einspeisepunkt der Antenne (Antenneneingangsleistung) bei bekannter Senderausgangsleistung?}
    {Sie ermitteln die Verluste zwischen Senderausgang und Antenneneingang und berechnen aus dieser Dämpfung einen Dämpfungsfaktor\$ D\$; die Antenneneingangsleistung ist dann: \$P\_\{\textbackslash\{\}textrm\{Ant\}\} = D\textbackslash\{\}cdot P\_\{\textbackslash\{\}textrm\{Sender\}\}\$}
    {Antenneneingangsleistung und Senderausgangsleistung sind gleich, da die Kabelverluste bei Amateurfunkstationen vernachlässigbar klein sind, d. h. es gilt: \$P\_\{\textbackslash\{\}textrm\{Ant\}\} = P\_\{\textbackslash\{\}textrm\{Sender\}\}\$}
    {Die Antenneneingangsleistung ist der Spitzenwert der Senderausgangsleistung, also: \$P\_\{\textbackslash\{\}textrm\{Ant\}\} = \textbackslash\{\}sqrt\{2\textbackslash\{\}cdot P\_\{\textbackslash\{\}textrm\{Sender\}\}\}\$}
    {Die Antenneneingangsleistung ist der Spitzen-Spitzen-Wert der Senderausgangsleistung, also: \$P\_\{\textbackslash\{\}textrm\{Ant\}\} = 2\textbackslash\{\}cdot\textbackslash\{\}sqrt\{2\textbackslash\{\}cdot P\_\{\textbackslash\{\}textrm\{Sender\}\}\}\$}
    {false}{false}