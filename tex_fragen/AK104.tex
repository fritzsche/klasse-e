\frage{AK104}
    {Wie errechnen Sie die Leistung am Einspeisepunkt der Antenne (Antenneneingangsleistung) bei bekannter Senderausgangsleistung?}
    {Sie ermitteln die Verluste zwischen Senderausgang und Antenneneingang und berechnen aus dieser Dämpfung einen Dämpfungsfaktor$ D$; die Antenneneingangsleistung ist dann: $P_{\textrm{Ant}} = D\cdot P_{\textrm{Sender}}$}
    {Antenneneingangsleistung und Senderausgangsleistung sind gleich, da die Kabelverluste bei Amateurfunkstationen vernachlässigbar klein sind, d. h. es gilt: $P_{\textrm{Ant}} = P_{\textrm{Sender}}$}
    {Die Antenneneingangsleistung ist der Spitzenwert der Senderausgangsleistung, also: $P_{\textrm{Ant}} = \sqrt{2\cdot P_{\textrm{Sender}}}$}
    {Die Antenneneingangsleistung ist der Spitzen-Spitzen-Wert der Senderausgangsleistung, also: $P_{\textrm{Ant}} = 2\cdot\sqrt{2\cdot P_{\textrm{Sender}}}$}
    {false}{false}