\frage{AC203}
    {Beim Anlegen einer Gleichspannung $U$ = 1 V an eine Spule messen Sie einen Strom. Wird der Strom beim Anlegen von einer Wechselspannung mit $U_{\textrm{eff}}$ = 1 V größer oder kleiner?}
    {Beim Betrieb mit Gleichspannung wirkt nur der Gleichstromwiderstand der Spule. Beim Betrieb mit Wechselspannung wird der induktive Widerstand $X_{\textrm{L}}$ wirksam und erhöht den Gesamtwiderstand. Der Strom wird kleiner.}
    {Beim Betrieb mit Gleichspannung wirkt nur der Gleichstromwiderstand der Spule. Beim Betrieb mit Wechselspannung wirkt nur der kleinere induktive Widerstand $X_{\textrm{L}}$. Der Strom wird größer.}
    {Beim Betrieb mit Gleich- oder Wechselspannung wirkt nur der ohmsche Widerstand $X_{\textrm{L}}$ der Spule. Der Strom bleibt gleich.}
    {Beim Betrieb mit Wechselspannung wirkt nur der Wechselstromwiderstand der Spule. Beim Betrieb mit Gleichspannung wird nur der ohmsche Widerstand $X_{\textrm{L}}$ wirksam. Der Strom wird größer.}
    {false}{false}