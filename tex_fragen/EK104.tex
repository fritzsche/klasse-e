\frage{EK104}
    {Muss ein Funkamateur als Betreiber einer ortsfesten Amateurfunkstelle bei FM-Telefonie und einer Sendeleistung von 6 W an einer 15-Element-Yagi-Uda-Antenne mit 13 dBd Gewinn im 2 m-Band die Einhaltung der Personenschutzgrenzwerte nachweisen?}
    {Ja, er ist in diesem Fall verpflichtet die Einhaltung der Personenschutzgrenzwerte nachzuweisen.}
    {Nein, der Schutz von Personen in elektromagnetischen Feldern ist durch den Funkamateur erst bei einer Strahlungsleistung von mehr als 10 W EIRP sicherzustellen.}
    {Ja, für ortsfeste Amateurfunkstellen ist die Einhaltung der Personenschutzgrenzwerte in jedem Fall nachzuweisen.}
    {Nein, bei FM-Telefonie und Sendezeiten unter 6 Minuten in der Stunde kann der Schutz von Personen in elektromagnetischen Feldern durch den Funkamateur vernachlässigt werden.}
    {false}{false}