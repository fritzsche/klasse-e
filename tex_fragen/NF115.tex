\frage{NF115}
    {Manche FM-Transceiver verfügen über einen analogen Datenanschluss (z. B. mit DATA beschriftet oder als 9600-Port bezeichnet). Dieser dient im Wesentlichen dazu, ...}
    {durch Umgehung von Verstärker- und Filterstufen ein NF-Signal (z. B. für DV oder POCSAG) möglichst verzerrungsfrei abzugreifen oder einzuspeisen.}
    {mittels eines seriellen Kommunikationsprotokolls den Transceiver z. B. mit einem Computer zu steuern und Werte abzufragen, z. B. Frequenz, Sendeleistung oder PTT.}
    {das empfangene HF-Signal möglichst ungefiltert an einen Computer auszuleiten und mittels digitaler Signalverarbeitung weiterzuverarbeiten.}
    {ohne weitere Beschaltung einen Drehwinkelgeber (Encoder) oder ein Potentiometer zur präzisen Frequenzeinstellung anzuschließen.}
    {false}{false}