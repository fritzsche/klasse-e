\frage{NF116}
    {Manche Transceiver verfügen über eine sogenannte CAT-Schnittstelle. Dieser Anschluss dient dazu, ...}
    {mittels eines seriellen Kommunikationsprotokolls den Transceiver z. B. mit einem Computer zu steuern oder Werte abzufragen, z. B. Frequenz, Sendeleistung oder PTT.}
    {durch Umgehung von Verstärker- und Filterstufen ein NF-Signal (z. B. für DV oder POCSAG) möglichst verzerrungsfrei abzugreifen oder einzuspeisen.}
    {das empfangene HF-Signal möglichst ungefiltert an einen Computer zur Weiterverarbeitung mittels digitaler Signalverarbeitung auszuleiten.}
    {ohne weitere Beschaltung einen Drehwinkelgeber (Encoder) oder ein Potentiometer zur präzisen Frequenzeinstellung anzuschließen.}
    {false}{false}