\setcounter{section}{13}
\section{Antennen und Übertragungsleitungen}
\subsection{Polarisation II}


\begin{sol}{EB306}
Wir müssen und daran orientieren in welche Richtung das elektrische Feld in Bezug auf die Erde orientiert ist.
Die elektrische Feldkomponente (mit E bezeichnet) ist horizontal.
\end{sol}

\begin{sol}{EB307}
Analog zu Frage \qref{EB305}, nur ist das elektrische Feld hier vertikal ausgerichtet.
\end{sol}

\begin{sol}{EB308}
Die elektrische Feldkomponente dreht sich quasi im Kreis in Bezug auf Erde und Hauptstahlrichtung. Wir nennen dies als zirkulare Polarisation.
\end{sol}

\begin{sol}{EB309}
Bei allen dipolartigen Antennen können wir die Orientierung des Strahlers orientieren.
Dies hilt auch für diesen Beam bei dem der Strahler offenbar horizontal orientiert ist.
\end{sol}


\begin{sol}{EB310}
Analog zu Frage \qref{EB309}, die Ausrichtung des Strahlers ist vertical orientiert.
\end{sol}


\begin{ohmchapter}
    Die Polarization einer Antenne wird nach der Richtung der Hauptstahlrichtung in Bezug zur Erdoberfläche angegeben.
\end{ohmchapter}



\subsection{Antennenformen II}

\begin{sol}{EG101}
  Es gibt verschiedene Schleifenantennen, die wie der Name schon verrät eine Schleife bilden. Sie können z.B. als Quadrat oder Dreieck aufgespannt werden. In der Konfiguration als Dreieck sprechen wir von einer \textbf{Delta-Loop-Antenne}.
\end{sol}


\begin{sol}{EG103}
 Wir sehen die Antenne besteht aus einen Draht der offenbar am Ende gespeist wird, also eine sogenannte \textbf{Endgespeiste Antenne}.
 Die hohe Impedanz (einige tausend Ohm) einer Endgespeisten Antenne müssen wir auf die 50 Ohm anpassen die das Funkgerät erwartet. Wie haben nur noch die Option A und C über.
 In diesem Fall können wir vielleicht den Parallelschwingkreis erkennen. Dies ist eine sogenannte Fuchs Antenne (benannt nach Josef Fuchs) mit einfachem Anpassglied. 
\end{sol}

\begin{sol}{EG104}
    Siehe Frage \qref{EG103}.
\end{sol}


\begin{sol}{EG105}
    Die meisten Antennen strahlen über die elektrische Komponente des elektromagnetischen Felds.
    Das Schlüsselwort \qq{magnetisches Feld} finden wir bereits in der Frage. Wie wählen also die Antwort \textbf{magnetische Ringantenne}. Die Antenne ist beliebt, da sie eine relativ kleine Bauform von nur etwas $\frac{\lambda}{10}$ hat.
\end{sol}

\begin{sol}{EG106}
 Die einzige Antenne die wir hier noch nicht explizit besprochen haben ist die \textbf{Windom-Antenne}. Dies ist eine Mehrband Drahtantenne für die Kurzwelle. 
Die anderen Antenne aus Antwort (A) sollten Dir bekannt vorkommen.
Gegenprobe: (B) eine Parabolantenne kennst Du vermutlich als Satellitenschüssel. Für Kurzwelle mit Sicherheit zu groß.
(C) eine Patchantenne kann direkt auf einer Leiterbahn verwendet werden. GPS Empfänger verwenden manchmal solche Antenne. Nichts für die Kurzwelle.
(D) Ein Hornstrahler ist eine Mikrowellen-Anntenne (z.B. für Radio Astronomie), also keine Kurzwelle.
\end{sol}

\begin{sol}{EG107}
    Eine W3DZZ Antenne ist eine für 80m geeignete Antenne die mit Sperrkreisen arbeitet. Oft wird die W3DZZ auf 40m und 80m betrieben.
Wir können wieder einige Antennen aus den Antworten für das 80m Band ausschließen.
(B) Sowohl die Kreuz-Yagi-Uda Antenne wie auch die Groundplane währen für 80 m einfach zu groß.
(C) Eine Sperrkopfantenne ist eher für 70cm und (D) der Parabolspiegel müsste für 80 m gigantisch sein.
\end{sol}


\begin{sol}{EG108}
Eine $5/8-\lambda$ Antenne ist gegenüber einer $\lambda / 4$ Antenne zunächst länger. Im 70 cm Band ist dies aber oft noch unkritisch.
Bauartbedingt hat sie mehr Gewinn. Für die Prüfung kannst Du es Dir einfach so merken Längere Antenne ergibt mehr Gewinn. (gilt natürlich nicht im Allgemeinen, aber hoffentlich für diese Frage.)  
\end{sol}


\begin{sol}{EG213}
Eine symmetrische Antenne ist, wie der Name vermuten lässt symmetrische aufgebaut. Was wichtigste Beispiel ist der Dipol, der wegen der gleich aufgebauter Dipolhälften symmetrische ist.
Dies ist für (B),(C) und (D) der Fall. Hier wird aber gefragt, welche Antenne \underline{nicht} symmetrisch ist. Die Groundplane hat keine zweite Dipolhälfte und dafür Radials. Sie ist \underline{nicht} symmetrisch.
\end{sol}


\begin{sol}{EG214} 
    Wir erkennen Strahlungsdiagramm (B) als Beam (z.B. Yagi-Uda) (ähnlich Dipol aber mit Gewinn in eine Richtung) und (C) als Groundplane (wir sehen 3 Radials). 
    Für einen Dipol gilt, dass die Hauptstahlrichtung wie in (A) senkrecht zur Aufspannrichtung der Dipolhälften ist.    
\end{sol}


\begin{sol}{EG215} 
    Analog zur Frage \qref{EG214}.
\end{sol}

\begin{sol}{EG216} 
    Analog zur Frage \qref{EG214}.
\end{sol}


\begin{sol}{EG217} 
    Analog zur Frage \qref{EG214}.
\end{sol}

\begin{sol}{EG219} 
    Bereits eine $\lambda /4$ Vertikalantenne hat eine flache Abstrahlung. 
\end{sol}


\begin{ohmchapter}
Antennen sind für den Funkamateur eines der wichtigsten Themen. Die perfekte Antenne für alles gibt es nicht, jede Antenne bringt unterschiedliche Vor- und Nachteile mit sich.
Wie fangen direkt mit den Fragen an umd klären Vor- und Nachteile.
\end{ohmchapter}


\subsection{Antennenlänge und -resonanz}

\begin{sol}{EG102} 
    Diese Frage ist etwas verwirrend für jeden der zunächst an $\lambda / 2$, $\lambda / 4$ die in den falschen Antworten genannt werden.
    Tatsächlich kann man sehr viele verschiedene Längen von Drahtantennen anpassen damit sie auf einem Amateurfunk Kurzwellenband resonant ist.
    Dies ist die \qq{richtige} Antwort.

    Realitätscheck: Das ist dies natürlich nicht general korrekt: die Länge ist \underline{nicht} beliebig. Ist die Antenne viel zu kurz, so wird selbst im perfekt angepassten Aufbau diese Antenne einen so schlechten Wirkungsgrad haben, dass sie quasi unbrauchbar ist.
\end{sol}


\begin{sol}{EG109}
    Rechnung: 
    $$\lambda = \frac{300}{\SI{28,5}{\mega\hertz}} \approx 10,53$$
    $$\frac{5}{8}\cdot \lambda = \frac{5}{8} \cdot 10,53 \approx  6,58$$
\end{sol}

\begin{sol}{EG110}
  Ein \textbf{Faltdipol} ist quasi eine plattgedrückte Ganzwellenschleifen. Die Drahtlänge ist eine Wellenlänge.
\end{sol}


\begin{ohmchapter}
In diesem Kapitel geht es um die Resonanz von Antenne. Hier geht es aber nur um 3 relativ einfache Fragen.
\end{ohmchapter}


\subsection{Verkürzungsfaktor I}

Die Lichtgeschwindigkeit beträgt im Vakuum $c= \SI{299792458} {\meter/\second}$. Wir haben die bereits für die Klasse N verwendet um die Wellenlänge zu berechnen:
$$ \lambda = \frac{c}{f}$$.

Die Lichtgeschwindigkeit ist allerdings in Leitungen (z.B. Antennendrähten) etwas langsamer. Nach einer Faustregel ist die Geschwindigkeit etwa 95\%. Der Verkürzungsfaktor $k_v$ gibt dies an. Also nach Faustregel: $k_v  \approx 0.95$.

Für die Wellenlänge gilt:
$$ \lambda_{\text{Leitung}} = k_v \cdot \frac{c}{f}$$.

In der Realität gibt es unterschiedliche Verkürzungsfaktoren abhängig von der Art der Leitung. Oft gibt es ein Datenblatt in dem man genauere Angaben finden kann.

Wenn Du die Länge eines Antennendrahtes berechnest solltest Du trotz Berücksichtigung des Verkürzungsfaktor in der Regel immer ein 10-15\% längeres Stück abschneiden, dass kannst Du dann immer noch trimmen:

\qq{Abschneiden ist einfacher als dranschneiden.}

\begin{sol}{EG201}
Die falschen Antworten enthalten immer Begriffe / Dinge die mit der Verkürzungsfaktor zu tun haben. Selbst wenn Du die Formel nicht im Kopf hast. Du solltest Dir merken, dass der \textbf{Verkürzungsfaktor} etwas mit der Ausbreitungsgeschwindigkeit im Vakuum zu tun hat und landest sofort bei (A).
\end{sol}

\begin{sol}{EG202}
 Die Faustregel nach der der Verkürzungsfaktor etwa 95\% ist musst Du dir merken.
\end{sol}

\subsection{Fußpunktimpedanz I}

\begin{sol}{EG207}
Der Halbwellendipol ist die bekannteste Antenne. Du musst Dir merken, dass die Impedanz (wenn noch montiert) nicht \SI{50}{\ohm} beträgt sondern etwas höher ist.
Die Beträgt etwa \SI{75}{\ohm}. Die falschen Antworten kannst Du ggf. auch ausschließen.
\end{sol}

\begin{sol}{EG208}
    In Frage \qref{EG207} haben wir den Dipol mit \SI{75}{\ohm} angegeben. In der Praxis lieft es oft niedriger und stellt für unseren Transceiver kein Problem dar.    
\end{sol}

\begin{sol}{EG209}
    Analog zu Frage \qref{EG208}.
\end{sol}

\begin{sol}{EG210}
    Das kannst Du Dir die richtigen Werte merken: nimm den größten Wert. Generell gilt für den Faltdipol, dass die Spannung verdoppelt wird und der benötigte Strom sich halbiert. Dies entspricht einer Vervierfachung. 
\end{sol}

\begin{sol}{EG211}
    Die Groundplane Antenne ist ja eine Art von Dipol (also nur eine Dipolhälfte + Radials). Da oft auch sehr bodennah, kannst Du Dir merken, dass wir einen niedrigen Fußpunktwiederstand haben. Die wählen also Antwort (A) die auch unsere markanten \SI{50}{\ohm} enthält.
\end{sol}


\begin{ohmchapter}
Wir haben den Begriff \textbf{Impedanz} bereits als Wechselstromwiderstand kennengelernt.
Bei der  \textbf{Fußpunktimpedanz} geht es um die Impedanz am Einspeisepunkt der Antenne.

Unsere Transceiver erwarten in der Regel eine Impedanz von \SI{50}{\ohm}.
Unterschiedliche Antennen und Aufbauvarianten (z.B. Höhe) haben unterschiedliche Fußpunktimpedanz. Diese musst Du einfach lernen.

\end{ohmchapter}


\subsection{Yagi-Uda Antenne II}

\begin{sol}{EG211}
    Wie im Eingang des Kapitels beschrieben.
\end{sol}


\begin{sol}{EG212}
    Wie im Eingang des Kapitels beschrieben.
\end{sol}


\begin{sol}{EG218}
    Das Strahlungsdiagramm zeigt eine klare Richtcharakteristik. D.h. viel mehr Leistung geht nach rechts als nach links. Dies ist typisch für einen Beam.
    Nur Antwort (A) enthält mit der Yagi-Uda Antenne einen Beam der in Frage kommt.
\end{sol}

\begin{ohmchapter}
Die Yagi-Uda Antenne wurde ab 1924 von den Japanern Hidetsugu Yagi und Shintaro Uda entwickelt. 

Der generelle Aufbau ist vielen zugmindestems durch TV- und Rundfunk Antennen bekannt, grob gesprochen besteht sie aus unterschiedlichen Elementen die in Hauptstahlrichtung immer kleiner werden. 
Eines der Elemente ist der \textbf{Strahler}. Oft ist er als Dipol oder als Faltdipol ausgeführt. Er hat den Einspeisepunkt der ganzen Antenne.

Die Elemente länger als der Strahler werden \textbf{Reflektor} genannt. Die Elemente kürzer als der Strahler werden \textbf{Direktor} genannt. 



\end{ohmchapter}
