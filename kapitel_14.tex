\setcounter{section}{13}
\section{Antennen und Übertragungsleitungen}
\subsection{Polarisation II}


\begin{sol}{EB306}
Wir müssen und daran orientieren in welche Richtung das elektrische Feld in Bezug auf die Erde orientiert ist.
Die elektrische Feldkomponente (mit E bezeichnet) ist horizontal.
\end{sol}

\begin{sol}{EB307}
Analog zu Frage \qref{EB305}, nur ist das elektrische Feld hier vertikal ausgerichtet.
\end{sol}

\begin{sol}{EB308}
Die elektrische Feldkomponente dreht sich quasi im Kreis in Bezug auf Erde und Hauptstahlrichtung. Wir nennen dies als zirkulare Polarisation.
\end{sol}

\begin{sol}{EB309}
Bei allen dipolartigen Antennen können wir die Orientierung des Strahlers orientieren.
Dies hilt auch für diesen Beam bei dem der Strahler offenbar horizontal orientiert ist.
\end{sol}


\begin{sol}{EB310}
Analog zu Frage \qref{EB309}, die Ausrichtung des Strahlers ist vertical orientiert.
\end{sol}


\begin{ohmchapter}
    Die Polarization einer Antenne wird nach der Richtung der Hauptstahlrichtung in Bezug zur Erdoberfläche angegeben.
\end{ohmchapter}



\subsection{Antennenformen II}

\begin{sol}{EG101}
  Es gibt verschiedene Schleifenantennen, die wie der Name schon verrät eine Schleife bilden. Sie können z.B. als Quadrat oder Dreieck aufgespannt werden. In der Konfiguration als Dreieck sprechen wir von einer \textbf{Delta-Loop-Antenne}.
\end{sol}


\begin{sol}{EG103}
 Wir sehen die Antenne besteht aus einen Draht der offenbar am Ende gespeist wird, also eine sogenannte \textbf{Endgespeiste Antenne}.
 Die hohe Impedanz (einige tausend Ohm) einer Endgespeisten Antenne müssen wir auf die 50 Ohm anpassen die das Funkgerät erwartet. Wie haben nur noch die Option A und C über.
 In diesem Fall können wir vielleicht den Parallelschwingkreis erkennen. Dies ist eine sogenannte Fuchs Antenne (benannt nach Josef Fuchs) mit einfachem Anpassglied. 
\end{sol}

\begin{sol}{EG104}
    Siehe Frage \qref{EG103}.
\end{sol}


\begin{sol}{EG105}
    Die meisten Antennen strahlen über die elektrische Komponente des elektromagnetischen Felds.
    Das Schlüsselwort \qq{magnetisches Feld} finden wir bereits in der Frage. Wie wählen also die Antwort \textbf{magnetische Ringantenne}. Die Antenne ist beliebt, da sie eine relativ kleine Bauform von nur etwas $\frac{\lambda}{10}$ hat.
\end{sol}


\begin{ohmchapter}
Antennen sind für den Funkamateur eines der wichtigsten Themen. Die perfekte Antenne für alles gibt es nicht, jede Antenne bringt unterschiedliche Vor- und Nachteile mit sich.
Wie fangen direkt mit den Fragen an umd klären Vor- und Nachteile.
\end{ohmchapter}
