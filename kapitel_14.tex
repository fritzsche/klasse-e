\setcounter{section}{13}
\section{Antennen und Übertragungsleitungen}
\subsection{Polarisation II}


\begin{sol}{EB306}
Wir müssen und daran orientieren in welche Richtung das elektrische Feld in Bezug auf die Erde orientiert ist.
Die elektrische Feldkomponente (mit E bezeichnet) ist horizontal.
\end{sol}

\begin{sol}{EB307}
Analog zu Frage \qref{EB305}, nur ist das elektrische Feld hier vertikal ausgerichtet.
\end{sol}

\begin{sol}{EB308}
Die elektrische Feldkomponente dreht sich quasi im Kreis in Bezug auf Erde und Hauptstahlrichtung. Wir nennen dies als zirkulare Polarisation.
\end{sol}

\begin{sol}{EB309}
Bei allen dipolartigen Antennen können wir die Orientierung des Strahlers orientieren.
Dies hilt auch für diesen Beam bei dem der Strahler offenbar horizontal orientiert ist.
\end{sol}


\begin{sol}{EB310}
Analog zu Frage \qref{EB309}, die Ausrichtung des Strahlers ist vertical orientiert.
\end{sol}


\begin{ohmchapter}
    Die Polarization einer Antenne wird nach der Richtung der Hauptstahlrichtung in Bezug zur Erdoberfläche angegeben.
\end{ohmchapter}



\subsection{Antennenformen II}

\begin{sol}{EG101}
  Es gibt verschiedene Schleifenantennen, die wie der Name schon verrät eine Schleife bilden. Sie können z.B. als Quadrat oder Dreieck aufgespannt werden. In der Konfiguration als Dreieck sprechen wir von einer \textbf{Delta-Loop-Antenne}.
\end{sol}


\begin{sol}{EG103}
 Wir sehen die Antenne besteht aus einen Draht der offenbar am Ende gespeist wird, also eine sogenannte \textbf{Endgespeiste Antenne}.
 Die hohe Impedanz (einige tausend Ohm) einer Endgespeisten Antenne müssen wir auf die 50 Ohm anpassen die das Funkgerät erwartet. Wie haben nur noch die Option A und C über.
 In diesem Fall können wir vielleicht den Parallelschwingkreis erkennen. Dies ist eine sogenannte Fuchs Antenne (benannt nach Josef Fuchs) mit einfachem Anpassglied. 
\end{sol}

\begin{sol}{EG104}
    Siehe Frage \qref{EG103}.
\end{sol}


\begin{sol}{EG105}
    Die meisten Antennen strahlen über die elektrische Komponente des elektromagnetischen Felds.
    Das Schlüsselwort \qq{magnetisches Feld} finden wir bereits in der Frage. Wie wählen also die Antwort \textbf{magnetische Ringantenne}. Die Antenne ist beliebt, da sie eine relativ kleine Bauform von nur etwas $\frac{\lambda}{10}$ hat.
\end{sol}

\begin{sol}{EG106}
 Die einzige Antenne die wir hier noch nicht explizit besprochen haben ist die \textbf{Windom-Antenne}. Dies ist eine Mehrband Drahtantenne für die Kurzwelle. 
Die anderen Antenne aus Antwort (A) sollten Dir bekannt vorkommen.
Gegenprobe: (B) eine Parabolantenne kennst Du vermutlich als Satellitenschüssel. Für Kurzwelle mit Sicherheit zu groß.
(C) eine Patchantenne kann direkt auf einer Leiterbahn verwendet werden. GPS Empfänger verwenden manchmal solche Antenne. Nichts für die Kurzwelle.
(D) Ein Hornstrahler ist eine Mikrowellen-Anntenne (z.B. für Radio Astronomie), also keine Kurzwelle.
\end{sol}

\begin{sol}{EG107}
    Eine W3DZZ Antenne ist eine für 80m geeignete Antenne die mit Sperrkreisen arbeitet. Oft wird die W3DZZ auf 40m und 80m betrieben.
Wir können wieder einige Antennen aus den Antworten für das 80m Band ausschließen.
(B) Sowohl die Kreuz-Yagi-Uda Antenne wie auch die Groundplane währen für 80 m einfach zu groß.
(C) Eine Sperrkopfantenne ist eher für 70cm und (D) der Parabolspiegel müsste für 80 m gigantisch sein.
\end{sol}


\begin{sol}{EG108}
Eine $5/8-\lambda$ Antenne ist gegenüber einer $\LAMBDA / 4$ Antenne zunächst länger. Im 70 cm Band ist dies aber oft noch unkritisch.
Bauartbedingt hat sie mehr Gewinn. Für die Prüfung kannst Du es Dir einfach so merken Längere Antenne ergibt mehr Gewinn. (gilt natürlich nicht im Allgemeinen, aber hoffentlich für diese Frage.)  
\end{sol}


\begin{sol}{EG213}
Eine symmetrische Antenne ist, wie der Name vermuten lässt symmetrische aufgebaut. Was wichtigste Beispiel ist der Dipol, der wegen der gleich aufgebauter Dipolhälften symmetrische ist.
Dies ist für (B),(C) und (D) der Fall. Hier wird aber gefragt, welche Antenne \underline{nicht} symmetrisch ist. Die Groundplane hat keine zweite Dipolhälfte und dafür Radials. Sie ist \underline{nicht} symmetrisch.
\end{sol}


\begin{sol}{EG214} 
    Wir erkennen Strahlungsdiagramm (B) als Beam (z.B. Yagi-Uda) (ähnlich Dipol aber mit Gewinn in eine Richtung) und (C) als Groundplane (wir sehen 3 Radials). 
    Für einen Dipol gilt, dass die Hauptstahlrichtung wie in (A) senkrecht zur Aufspannrichtung der Dipolhälften ist.    
\end{sol}


\begin{sol}{EG215} 
    Analog zur Frage \qref{EG214}.
\end{sol}

\begin{sol}{EG216} 
    Analog zur Frage \qref{EG214}.
\end{sol}


\begin{sol}{EG217} 
    Analog zur Frage \qref{EG214}.
\end{sol}

\begin{sol}{EG219} 
    Bereits eine $\lambda /4$ Vertikalantenne hat eine flache Abstrahlung. 
\end{sol}


\begin{ohmchapter}
Antennen sind für den Funkamateur eines der wichtigsten Themen. Die perfekte Antenne für alles gibt es nicht, jede Antenne bringt unterschiedliche Vor- und Nachteile mit sich.
Wie fangen direkt mit den Fragen an umd klären Vor- und Nachteile.
\end{ohmchapter}
