\section*{Einleitung}
In diesem Dokument stellen wir einige Informationen für den Klasse E Aufbaukurs des Ortsverbands A02 zusammen.
Da sich Funker immer per \qq{Du} ansprechen, will ich in diesem Dokument auch so machen.


Hauptfokus dieses Dokuments ist die Prüfungsvorbereitung und Lernhilfen zu geben. Die Inhalte können deshalb an einigen Stellen verkürzt oder gar Fehlerhaft sein. Damit möchte ich an die von Gunther Lindemann veröffentlichen Lernhilfen für den alten Fragenkatalog anknüpfen die mir beim Erwerb meiner eigenen Amateurfunklizenz viel geholfen hat. (Homepage: \href{https://dl9hcg.a36.de}{https://dl9hcg.a36.de}).
Dieses Dokument verwendet die Kapitalstruktur der DARC Lernplattform \url{http://50ohm.de}. Du kannst also alle Inhalte dort nachlesen und vertiefen. In diesem Dokument fassen wir die Inhalte absichtlich nur sehr knapp zusammen. In diesem Dokument beschränke ich mich auf die Inhalte die im Fragenkatalog vorkommen.
Die Fragen und Musterantworten in diesem Dokument stammen aus der maschinenlesbaren Version des Fragenkatalog wie er am 16.6.2024 von der Bundesnetzagentur veröffentlicht wurde. Fragen und Musterantworten sind nur technisch konvertiert worden um mit dem Satzsystem Latex verarbeitet werden zu können.

Die Inhalte des Fragenkatalog unterliegt dabei den Bestimmungen: \href{https://www.govdata.de/dl-de/by-2-0}{https://www.govdata.de/dl-de/by-2-0}.
Wenn es sich nicht um ein triviale Definition handelt wird die Lösung jeder Frage im Detail im Block \qq{Lösungsansatz} erklärt.
Für die Musterantworten gilt, dass immer Antwort A die korrekte Antwort ist. Die falschen Antworten B/C/D sind auch angegeben, da es an einigen Stellen für Dich hilfreich sein kann mit dem Ausschlussprinzip zu arbeiten.


Viel Spaß und Erfolg beim gemeinsamen Hobby Amateurfunk!

73 DE DJ1TF - Thomas

\section*{Haftung}
Es sei darauf hingewiesen, dass der Author ein Funkamateur im wahrsten Sinne des Wortes ist. Als Amateur hat er keine berufliche Ausbildung im Bereich der hier dargestellten Amateurfunkthemen hat. 

Deshalb kann dieses Dokument inhaltliche Fehler, sachlich falsche Aussagen enthalten. Der Author ist dafür nicht haftbar.
Das Ziel des Dokuments ist auch nicht ein möglichst genaue Fachliche Darstellung der Themen, sondern vielmehr Lernhilfen zu geben, damit die Fragen in der Amateurfunkprüfung der Klasse E richtig beantwortet werden können.
Jegliche Haftung ist ausgeschlossen.

