\section*{Einleitung}
In diesem Dokument stellen wir einige Informationen für den Klasse E Aufbaukurs des Ortsverbands A02 zusammen.
Es sei darauf hingewiesen, dass der Author ein Funkamateur im wahrsten Sinne des Wortes ist, und als Amateur keine berufliche Ausbildung im Bereich der hier dargestellten Amateurfunkthemen hat. 

Deshalb kann dieses Dokument inhaltliche Fehler, sachlich falsche Aussagen enthalten. Der Author ist dafür nicht haftbar.
Das Ziel des Dokuments ist auch nicht ein möglichst genaue Fachliche Darstellung der Themen, sondern vielmehr Lernhilfen zu geben, damit die Fragen in der Amateurfunkprüfung der Klasse E richtig beantwortet werden können.

Da sich Funker immer per \qq{Du} ansprechen, will ich in diesem Dokument auch nicht anders machen.

Dieses Dokument verwendet die Kapitalstruktur der DARC Lernplattform \url{http://50ohm.de}. Du kannst also alle Inhalte dort nachlesen. In diesem Dokument fassen wir die Inhalte absichtlich nur sehr knapp zusammen. Wir beschränken uns auf die nach Inhalte die im Fragenkatalog vorkommen. 

Wenn es sich nicht im ein triviale definition handelt wird die Lösung jeder Frage im Detail im Block \qq{Lösungsansatz} erklärt.