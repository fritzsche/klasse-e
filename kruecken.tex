\documentclass[10pt,a4paper,ngerman]{article}
\usepackage[utf8]{inputenc}
\usepackage{tkz-euclide}
\usepackage[T1]{fontenc}

%\usepackage[]{ntheorem}
%\usepackage[amsthm,thmmarks]{ntheorem}
\usepackage{amsmath}
%\usepackage{amsthm}
\usepackage{hyperref}
\usepackage{cleveref}

\usepackage{amsthm}
\usepackage{thmtools}
\usepackage{amsfonts}
\usepackage{amssymb}
\usepackage{babel}
\usepackage{textcomp}

\usepackage{adjustbox}


\newtheoremstyle{mytheorem}
  {} % Space above (default: 3pt)
  {} % Space below (default: 3pt)
  {\itshape} % Body font: set to italic (standard for plain theorem style)
  {} % Indent amount
  {\bfseries} % Head font: set to bold
  {.} % Punctuation after theorem head (e.g., Theorem 1.1.)
  {.5em} % Space after theorem head (e.g., .5em or " " for normal interword space)
  {\thmname{#1}\thmnumber{ #2}\thmnote{ (\textbf{#3})}}


\theoremstyle{definition}
\newtheorem{defn}{Definition}
\newtheorem*{example}{Beispiel}

\theoremstyle{plain}
\theoremstyle{mytheorem}
\newtheorem{theorem}{Satz}
\newtheorem{lemma}{Lemma}
\newtheorem*{axiom}{Axiom}

\newcommand{\R}{\mathbb{R}}
\newcommand{\N}{\mathbb{N}}
\newcommand{\Q}{\mathbb{Q}}
\newcommand{\Z}{\mathbb{Z}}
\newcommand{\Prim}{\mathbb{P}}
\newcommand{\Nnull}{\mathbb{N}_0}

%*****************************************
\usepackage{adjustbox}
%\usepackage{stackengine}
%\usepackage{paralist}
\usepackage{xfp}
\usepackage{siunitx} % Wichtig für Zahlenformatierung
\usepackage{xparse}
\usepackage{xstring}
\usepackage{enumitem}
\usepackage{environ}
\usepackage{textcmds}
\usepackage[left=25mm,right=25mm,top=25mm,bottom=25mm,paper=a4paper]{geometry}

\usepackage[HTML]{xcolor}


\newcommand{\qref}[1]{\hyperref[#1]{#1}}

\definecolor{c00a0dc}{HTML}{00A0DC}
\definecolor{cbdbdbd}{HTML}{0BBDBD}
\definecolor{ce05022}{HTML}{E05022}
\definecolor{ce5e5e5}{HTML}{E5E5E5}
\definecolor{c7fcfed}{HTML}{7FCFED}

\definecolor{cbfbfbf}{HTML}{BFBFBF}
\definecolor{c7f7f7f}{HTML}{7F7F7F}

\usepackage{svg}

\usepackage{siunitx}
\sisetup{locale = DE}
%\usepackage{array}

\theoremstyle{definition}
\newtheorem*{auf}{Aufgabe}

\usepackage{circuitikz}[european]


%\newcommand{\frage}[6]{%
%  \item[#1] #2
%  \begin{enumerate}
%    \item #3
%    \item #4
%    \item #5
%    \item #6
%  \end{enumerate}
%}


%\makeatletter
\NewEnviron{sol}[1]{%
  \edef\macroname{savedata@#1}%
  % Verwende \let anstelle von \edef, um eine Expansion des Inhalts zu verhindern
  \expandafter\global\expandafter\let\csname \macroname\endcsname\BODY
}
%\makeatother

\newcommand{\getsavedata}[1]{%
  \ifcsname savedata@#1\endcsname
    \par\noindent\hrulefill\par
    \textbf{Lösungsansatz:}\par
    \csname savedata@#1\endcsname
    \par\noindent
    \hrulefill%\par
  \else
    % Nichts tun oder eine Warnung ausgeben
  \fi
}

\newcommand{\pic}[2][1]{%
\begin{center}
    \begin{tikzpicture}[scale=#1]
        \input{bilder/#2.tex}        
    \end{tikzpicture}
\end{center}    
}

\newcommand{\qgrafic}[2][1.3]{%
\begin{center}  
    \begin{tikzpicture}[scale=#1]
        \input{pic/#2_q.tex}        
    \end{tikzpicture}
\end{center}    
}


\newcommand{\qgraficmeasure}[2][1.3]{%
  \begin{tikzpicture}[scale=#1]
    \input{pic/#2_q.tex}
  \end{tikzpicture}%
}

% Automatic layout switcher
\newsavebox{\tempbox}
\newlength{\imgwidth}
\newlength{\imgheight}

\newcommand{\agrafic}[2]{%
\resizebox{!}{3cm}{
  \begin{tikzpicture}[scale=1.5]
      \input{pic/#1_#2.tex}        
  \end{tikzpicture}
}
}


\ExplSyntaxOn
\NewDocumentCommand{\frage}{ m m m m m m m m }{
  \item[#1] \label{#1} #2
    \tl_if_eq:nnTF {#7} {true} {
      \sbox{\tempbox}{\qgraficmeasure{#1}}%
      \settoheight{\imgheight}{\usebox{\tempbox}}%
        \ifdim \imgheight>10cm          
          \qgrafic[0.7]{#1}
        \else          
          \qgrafic{#1}
        \fi  
    }{}
\getsavedata{#1}

\tl_if_eq:nnTF {#8} {true} {
\begin{center}

  \sbox{\tempbox}{\adjustbox{valign=c}{\agrafic{#1}{a}}}%
  \settowidth{\imgwidth}{\usebox{\tempbox}}%

  % If two side-by-side images exceed 0.9 of text width, switch to 1x4
  \ifdim 2\imgwidth>0.9\linewidth
    % ---------- Wide layout: 1×4 ----------
    \begin{enumerate}[label=(\Alph*)]
      \item \adjustbox{valign=c}{\agrafic{#1}{a}}
      \item \adjustbox{valign=c}{\agrafic{#1}{b}}
      \item \adjustbox{valign=c}{\agrafic{#1}{c}}
      \item \adjustbox{valign=c}{\agrafic{#1}{d}}
    \end{enumerate}
  \else
    % ---------- Normal layout: 2×2 ----------
    \begin{tabular}{cc}
      (A) \adjustbox{valign=c}{\agrafic{#1}{a}} &
      (B) \adjustbox{valign=c}{\agrafic{#1}{b}} \\
      \multicolumn{2}{c}{\vspace{10pt}} \\
      (C) \adjustbox{valign=c}{\agrafic{#1}{c}} &
      (D) \adjustbox{valign=c}{\agrafic{#1}{d}} \\
    \end{tabular}
  \fi

\end{center}


%\begin{enumerate}[label=(\Alph*)]
%    \item \adjustbox{valign=c}{\agrafic{#1}{a}} 
%    \item \adjustbox{valign=c}{\agrafic{#1}{b}} 
%    \item \adjustbox{valign=c}{\agrafic{#1}{c}} 
%    \item \adjustbox{valign=c}{\agrafic{#1}{d}}   
%  \end{enumerate}        
} {
  \begin{enumerate}[label=(\Alph*)]
    \item #3
    \item #4
    \item #5
    \item #6
  \end{enumerate}      
}

}
\ExplSyntaxOff

\newenvironment{ohmchapter}{}
{
  \subsubsection*{Lösungen}
  \input{tex_sections/\arabic{section}S\arabic{subsection}.tex}
}



\newcommand{\DEnumber}[1]{%
    \num[
        parse-numbers=true,          % Wichtig: Erzwingt die Analyse der Zahl
        output-decimal-marker={,},   % Für das Komma als Dezimaltrennzeichen
        group-separator={\,},        % Optional: Dünner Zwischenraum als Tausender-Trennzeichen
        scientific-notation=false    % Sicherstellen, dass keine wissenschaftliche Notation verwendet wird
    ]{#1}%
}

\newcommand{\calc}[1]{%
  \DEnumber{\fpeval{#1}}
}

\newcommand{\mischer}[2]{%    
    Wir rechnen: 
    \begin{itemize}
      \item \DEnumber{#1} MHz - \DEnumber{#2} MHz = \calc{#1-#2} MHz
      \item \DEnumber{#1} MHz + \DEnumber{#2} MHz = \calc{#1+#2} MHz        
    \end{itemize}  
}

\author{Thomas Fritzsche}
\title{Lernkrücken für den Amateurfunkkurs der Klasse E von A02}
\begin{document}
\maketitle
\tableofcontents





%%%%%%%%%%%%%%%%%
%%%  Kapitel  %%%
%%%%%%%%%%%%%%%%%


\section*{Einleitung}
In diesem Dokument stellen wir einige Informationen für den Klasse E Aufbaukurs des Ortsverbands A02 zusammen.
Es sei darauf hingewiesen, dass der Author ein Funkamateur im wahrsten Sinne des Wortes ist, und als Amateur keine berufliche Ausbildung im Bereich der hier dargestellten Amateurfunkthemen hat. 

Deshalb kann dieses Dokument inhaltliche Fehler, sachlich falsche Aussagen enthalten. Der Author ist dafür nicht haftbar.
Das Ziel des Dokuments ist auch nicht ein möglichst genaue Fachliche Darstellung der Themen, sondern vielmehr Lernhilfen zu geben, damit die Fragen in der Amateurfunkprüfung der Klasse E richtig beantwortet werden können.

Da sich Funker immer per \qq{Du} ansprechen, will ich in diesem Dokument auch nicht anders machen.

Dieses Dokument verwendet die Kapitalstruktur der DARC Lernplattform \url{http://50ohm.de}. Du kannst also alle Inhalte dort nachlesen. In diesem Dokument fassen wir die Inhalte absichtlich nur sehr knapp zusammen. Wir beschränken uns auf die nach Inhalte die im Fragenkatalog vorkommen. 

Wenn es sich nicht im ein triviale definition handelt wird die Lösung jeder Frage im Detail im Block \qq{Lösungsansatz} erklärt.

\setcounter{section}{7}
\section{Grundlegende Schaltungen}

%\subsection{Schwingkreis I}


\begin{sol}{ED201}
In diesem Diagram sehen wir wie viel Leistung bei welcher Frequenz $f$ durchgelassen wird.
Wie sehen, dass unterhalb der mit gestrichelten Linien markiertem Grenzfrequenz haben wir eine konstant hohe Leistung.
Über der \qq{Grenzfrequenz} fällt die Leistung schnell ab.
Da niedrige Frequenzen durchgelassen werden aber hohe Frequenzen sind, handelt es sich um einen Tiefpassfilter.
\end{sol}

\begin{sol}{ED201}
    Hier ist alles nur umgekehrt wie in Frage \qref{ED201}. Es handelt sich also um einen Hochpassfilter.
\end{sol}

\begin{sol}{ED202}
    Hier ist alles nur umgekehrt wie in Frage \qref{ED201}. Es handelt sich also um einen Hochpassfilter.
\end{sol}


\begin{sol}{ED203}
 Es wird nur ein kleines Frequenzspektrum durchgelassen. Es ist ein Bandpass.
\end{sol}


\begin{sol}{ED204}
 Umgekehrt wie in Frage \qref{ED203} wird hier jede Frequenz außerhalb eines Bereichs durchgelassen. 
 Es ist eine Bandsperre.
\end{sol}


\begin{sol}{ED205}
 Niedriger Widerstand bei einer Frequenz. Wie in diesem Kapitel gelernt haben ist die ein Serienschwingkreises.
\end{sol}

\begin{sol}{ED206}
Gegenteil von Frage \qref{ED205}. Ein Parallelschwingkreis.
\end{sol}


\begin{sol}{ED207}
 Wie in Frage \qref{ED206}: ein hochohmiger Widerstand.
\end{sol}

\begin{sol}{ED208}
 Der Kondensator ist \qq{tief}: Tiefpass.
\end{sol}

\begin{sol}{ED209}
 Der Kondensator ist \qq{tief}: Tiefpass.
\end{sol}

\begin{sol}{ED209}
 Nur in Bild A ist der Kondensator \qq{tief}.
\end{sol}


\begin{sol}{ED211}
 Der Kondensator ist \qq{hoch}: Hochpass.
\end{sol}

\begin{sol}{ED212}
 Der Kondensator ist \qq{hoch}: Hochpass.
\end{sol}

\begin{sol}{ED213}
 Nur in Bild A ist der Kondensator \qq{hoch}.
\end{sol}

\begin{sol}{ED214}
 Wie haben einen Parallelschwingkreis im Signalweg. Der wird bei Resonanzfrequenz sperren: Sperrkreis.
\end{sol}


\begin{sol}{ED215}
 Wie haben einen Serienschwingkreises parallel zum Signalweg: Saugkreis. Diesen ulkigen Namen einfach merken!
\end{sol}

\begin{ohmchapter}
    In diesem Kapitel geht es um verschiedene Schwingkreise und Filter.
    Zunächst erinnern wir uns an dem Spannungsteiler.
    Zudem erinnern wir uns an den frequenzabhängigen Wechselstrohmwiderstand eines Kondensator $X_C$.
    $X_C$ ist bei niedrigen Frequenzen sehr hoch und niedrig für hohe Frequenzen. Merkhilfe: (Bei Frequent \SI{0}{\hertz} leitet ein Kondensator nicht).

    Schauen wir uns einen sehr einfachen Filter an:
    \qgrafic[0.9]{ED208}

    Hätten wir in der Schaltung statt des Kondensator einen Widerstand so hätten wir einen einfachen Spannungsteiler. Da wir hier aber einen Kondensator verwenden der einen von der Frequenz abhängigen Widerstand hat.
    D.h. der Spannungsteiler ist frequenzabhängig.
    Bei niedrigen Frequenzen wird der Kondensator einen sehr hohen Widerstand haben und die Leistung wird nicht reduziert. Bei hohen Frequenzen wird der Kondensator zunehmend leitend und damit ist die Leistung die den Filter passiert reduziert.
    Dies ist ein \textbf{Tiefpassfilter}.
    In einem Diagramm können wir dies wie folgt darstellen:
        \qgrafic[0.9]{ED201}


    Was passiert, wenn wir in einem Spannungsteiler den anderen Widerstand durch einen Kondensator ersetzen?
    Also etwa so:
    \qgrafic[0.9]{ED211}

   Bei niedrigen Frequenzen wird hier fast keine Leistung durchgelassen und bei hohen Frequenzen sehr viel. Es ist ein \textbf{Hochpassfilter}.
   Die Filtercharakteristik eines solchen Hochpassfilter sieht also folgendermaßen aus:
  \qgrafic[0.9]{ED202}   


   Jetzt, da wir RC-Filter kennengelernt haben, erinnern wir uns an den Wechselstrohmwiderstand $X_L$ einer Spule. Der war genau umgekehrt zum Kondensator. Bei niedrigen Frequenzen leitet die Spule sehr gut und bei hohen Frequenzen haben wir mehr und mehr Widerstand.
   D.h. in unserem Spannungsteiler hätten wir auch einen Widerstand durch eine Spule ersetzten können und Hochpass und Tiefpass bauen können. 
   Ersetzten wir einen der Widerstände durch einen Kondensator und den anderen durch eine Spule so sind beide Widerstände unseres Spannungsteilers frequenzabhängig. Der Effekt wird verstärkt. Wie sprechen von einem LC-Filter.

   Eine wichtige \textbf{Merkregel für die Prüfung}:
   Um zu beantworten ob ein Filter ein Hochpass oder Tiefpass ist schauen wir uns die Position des Kondensators im Schaltbild an.
   Ist der \qq{tief} eingezeichnet, so ist es ein Tiefpass. Ist er \qq{hoch} eingezeichnet haben wir einen Hochpassfilter.

Nachdem wir uns mit Hochpass und Tiefpass beschäftigt haben, die im Grunde frequenzabhängige Spannungsteiler sind, wollen wir uns mit einer neuen Anordnung von Kondensator und Spule beschäftigen.
Wenn wir sie parallel anordnen wie in diesem Schaltkreis:
\begin{center}
\ctikzset{bipoles/length=1cm}   
\begin{circuitikz}[scale=0.8,european,american inductors]
%\draw[style=help lines] (-3,-3) grid (10,5);    
\draw (0,0) to [R, o-*] ++(4,0) 
  to [short,*-*] ++(0,-0.5)
  to [short] ++(-1,0)
  to [L,mirror ] ++(0,-2)
  to [short,-*] ++(+1,0)
  to [short,*-*] ++(0,-0.5)
  to [short,-o] ++(-4,0); 

\draw (4,0) to [short, -o] ++(2,0);   
\draw (4,-3) to [short, -o] ++(2,0);

\draw (4,-0.5) to [short, *-] ++(1,0)
 to [C] ++(0,-2)
 to [short] ++(-1,0);
\end{circuitikz}
\end{center}

entsteht ein \textbf{Parallelschwingkreis}.
Das bedeutet, dass wenn der Widerstand von Kondensator und Induktivität gleich sind, wird die Elektrische Energie ständig zwischen diesen beiden Komponenten ausgetauscht, also zwischen dem elektrischen Feld des Kondensators und dem magnetischen Feld der Spule. Man sagt auch der Schwingkreis ist in \textbf{Resonanz}. In einen idealen Parallelschwingkreis kann bei Resonant kein Strom mehr hineinfließen, er hat einen hohen Widerstand. Es entsteht ein \textbf{Bandpass}.

Wir können den Parallelschwingkreis aber auch in dieser Konfiguration verwenden:
 \qgrafic[0.9]{ED214} 
 Dies ist eine Sperrkreis, da der Parallelschwingkreis bei Resonanz einen hohen Widerstand hat.

Neben dem Parallelschwingkreis gibt es natürlich auch den \textbf{Serienschwingkreises}. 
 \qgrafic[0.9]{ED215} 
 In dieser Konfiguration hört er auf den ulkigen Namen \textbf{Saugkreis}. Auf der Resonanzfrequenz sind Kondensator und Spule beide leitend und diese Frequenzen werden quasi aus dem Signalweg gesaugt.
\end{ohmchapter}    


\subsection{Oszillatoren}

\begin{sol}{ED501}
L steht für die Spule und C für den Kondensator und ein Schwingkreis ist eine Art Oszillator.
\end{sol}

\begin{sol}{ED502}
Nach Tipp: \qq{zunehmend}, also Antwort mit \qq{niedriger}.
\end{sol}


\begin{sol}{ED503}
Nach Tipp: \qq{kleiner}, also Antwort mit \qq{höher}.
\end{sol}

\begin{sol}{ED504}
Nach Tipp: \qq{zunehmend}, also Antwort mit \qq{niedriger}.
\end{sol}


\begin{sol}{ED505}
Nach Tipp: \qq{kleiner}, also Antwort mit \qq{höher}.
\end{sol}


\begin{sol}{ED506}
    Sollte klar sein.
\end{sol}


\begin{sol}{ED507}
    Anfang des Kapitel erklärt: ein Quarz hat eine stabilere Frequenz. (z.B. bei ändernden Temperaturen)
\end{sol}

\begin{sol}{EF207}
    Hier wird Hochfrequenz erzeugt. Eine Abschirmung macht Sinn.
\end{sol}

\begin{sol}{EF305}
    Wir hatten bereits sehr viele Fragen zu dem Thema.
\end{sol}


\begin{ohmchapter}
Im letzten Kapitel haben wir bereits den Parallelschwingkreis kennen gelernt, in dem ein Kondensator und ein Spule in Resonanz gebracht werden. Solch ein Parallelschwingkreis ist das einfachste Beispiel von Schaltungen die Schwingungen erzeugen. Wie sagen zu so einem Schwingkreis auch \textbf{Oszillator}. In Fall des Parallelschwingkreis handelt es sich um einen \textbf{LC-Oszillator}.

LC-Oszillatoren sind leider Temperaturabhängig. Deshalb wird in modernen Schaltungen auf einen Schwingquarz zurückgegriffen. Wir haben dann einen \textbf{Quarz-Oszillator}.

Tipps für die Prüfung: bei allen Fragen bei denen es um die Temperatur eines LC-Oszillator und dessen Frequenz geht, kannst Du immer das Gegenteil der Frage nehmen. Soll heißen wenn die Temperatur steigt (Frage) ist die Antwort mit Frequenz fällt richtig.

\end{ohmchapter} 

\subsection{Frequenzvervielfacher I}


\begin{sol}{EF302}
    Wir haben die Frequenzvervielfacher mit Faktoren $2\cdot 3 \cdot 2 = 12$. Wir rechen rückwärts und teilen.
    Rechnung: $\frac{\SI{145.200}{\mega\hertz}}{2\cdot 3\cdot 2} = \SI{12.1}{\mega\hertz}$
\end{sol}

\begin{sol}{EF302}
    Auf dem Bild erkennen wir, dass wir auf dem Weg vom VFO zu Punkt $a$ an zwei Frequenzvervielfachern vorbeikommen, die den Faktor 2 und 3 haben. Wir gehen aber rückwärts und deshalb teilen wir durch 6.
    
    Rechnung: $\frac{\SI{21.360}{\mega\hertz}}{2\cdot 3} = \SI{3.56}{\mega\hertz}$
\end{sol}
\begin{sol}{EF303}
    Auf dem Bild erkennen wir, dass wir auf dem Weg vom VFO zu Punkt $a$ an zwei Frequenzvervielfachern vorbeikommen, die jeweils einen Faktor 2 haben.
    
    Rechnung: $\SI{3.51}{\mega\hertz} \cdot 2 \cdot 2 = \SI{14.04}{\mega\hertz}$
\end{sol}
\begin{ohmchapter}
    Um nicht für jede Frequenz die ein Funkgerät bracht einen anderen Quarz zu brachen, kommen Frequenzvervielfacher zum Einsatz. Wie der Name schon angibt vervielfacht sich die Frequenz entsprechend eines Faktors. Im Blockschaltdiagramm ist dieser Faktor angegeben.
\end{ohmchapter}

\setcounter{subsection}{3}
\subsection{Mischer}

\begin{sol}{EF201}
    \mischer{31.7}{21}  
\end{sol}
\begin{sol}{EF202}
    \mischer{38.7}{28}
\end{sol}
\begin{sol}{EF203}
    \mischer{39}{30} 
\end{sol}
\begin{sol}{EF204}
    \mischer{145}{136} 
\end{sol}
\begin{sol}{EF205}
    \mischer{145}{136} 
\end{sol}
\begin{sol}{EF206}
    In der Frage geht es um \glqq unerwünschte Abstrahlungen\grqq{}, wir müssen also abschirmen.    
\end{sol}


\begin{ohmchapter}
In einem Mischer werden zwei Eingangssignale zu einem Ausgangssignal gemischt. Das Blockschaltdiagramm eines Mischers sieht aus wie eine Waschmaschine. Tatsächlich soll uns das Kreuz in der Mitte des Symbols an ein Multiplikationszeichen erinnern, da es sich um meine multiplikative Mischung von Frequenzen handelt.
Beim Mischen entsteht aus den beiden Eingangsfrequenzen die Summe und Differenz Frequenz:


\begin{center}
\begin{adjustbox}{margin=10pt}
\begin{circuitikz}
\draw (0,0) node[mixer,boxed] (M) {}; 
\draw[->] (-2, 0)node[left] {$f_{\text{in 1}}$} -- (M.w); 
\draw[->] (0, -2)node[below] {$f_{\text{in 2}}$} -- (M.s); 
\draw[->] (M.out) -- (2, 0) node[right,text width=5cm] {
$\begin{aligned}
      &f_{\text{out 1}} = f_{\text{in 1}} + f_{\text{in 2}}\\
      &f_{\text{out 2}} = | f_{\text{in 1}} - f_{\text{in 2}} | 
    \end{aligned}$
    }; 
\end{circuitikz}
\end{adjustbox}
\end{center}



Wie solch ein Mischer funktioniert kannst Du mit dieser \href{https://fritzsche.github.io/klasse-e/mischer.html}{App} interaktiv ausprobieren.


%$$f_{\text{out 1}} = f_{\text{in 1}} + f_{\text{in 2}} \textrm{\ und\ }f_{\text{out 2}} = | f_{\text{in 1}} - f_{\text{in 2}} |$$
%Als Blockschaltdiagramm sieht es folgendermaßen aus:


Hier nur eine kurze Erklärung wie sich dies Mathematisch herleiten lässt. Für die Prüfung brauchst Du diese Details nicht wissen!
In unserem Beispiel haben wir ein empfangenes Signal $S_{\text{empf}}$ 
und das Signal eines lokalen Oszillator  $S_{\text{LO}}$.

% --- Definition der Eingangssignale ---
\begin{align*}
    S_{\text{empf}}(t) &= A_{\text{empf}} \cdot \sin(\omega_{\text{empf}} t) \\
    S_{\text{LO}}(t) &= A_{\text{LO}} \cdot \sin(\omega_{\text{LO}} t)
\end{align*}

Ein Multiplikativer Mixer wird unser Signal zu einem  wie eine einfache Ausgangssignal $S_{\text{out}}$ multiplizieren:

% --- Mischer-Multiplikation ---
\begin{equation*}
    S_{\text{out}}(t) = S_{\text{empf}}(t) \cdot S_{\text{LO}}(t)
\end{equation*}

% --- Produkt-zu-Summe-Formel (Trigonometrie) ---
Wir verwenden die folgende Trigonometrie Formel:
\begin{equation*}
    \label{eq:sin_prod}
    \sin(A) \cdot \sin(B) = \frac{1}{2} \left[ \cos(A-B) - \cos(A+B) \right]
\end{equation*}

% --- Das resultierende Ausgangssignal ---

Also gilt:
\begin{align*}
    S_{\text{out}}(t) &= A_{\text{empf}} A_{\text{LO}} \cdot \sin(\omega_{\text{empf}} t) \cdot \sin(\omega_{\text{LO}} t) \\
    S_{\text{out}}(t) &= \frac{1}{2} A_{\text{empf}} A_{\text{LO}} \left[ \cos((\omega_{\text{empf}} - \omega_{\text{LO}}) t) - \cos((\omega_{\text{empf}} + \omega_{\text{LO}}) t) \right]
\end{align*}

Summe und Differenz entstehen also einfach aus unserer Formel.

% --- Komponenten-Bezeichnung (Optional, als Text) ---
% Die Differenzfrequenz ($\omega_{\text{empf}} - \omega_{\text{LO}}$) ist die Zwischenfrequenz (ZF).


\end{ohmchapter}

\begin{sol}{EF501}
    Der Transverter setzt natürlich vom 70cm Signal ins 10m Band um und umgekehrt.
    Aufpassen bei Antwort (B): Hier wird beim Senden und Empfangen jeweils von 70cm in's 10m Band umgesetzt. Das macht keinen Sinn. 

\end{sol}
\begin{sol}{EF502}
    Im letzten Kapitel haben wir über den Mixer gesprochen. Hier wird Summe und Differenz Frequenz gebildet.
\end{sol}
\begin{sol}{EF503}
    Im Blockschaltbild können wir die Sende-/Empfangsumschaltung erkennen wie zwischen RX und TX umschaltet. Es ist also der \textbf{Transverter}.
\end{sol}

\begin{sol}{EF504}
    Es gibt keine Sende-/Empfangsumschaltung und überhaupt nur den Empfang. Es ist also ein \textbf{Konverter}.
\end{sol}

\begin{sol}{EF505}
    Diese Fragen hat viele ähnlich Antworten. Liess dies alle genau durch!
    Es geht um den Satellitenbetrieb über die hohe Frequenz von 2,4 GHz. Wir müssen also die Sendefrequenz vervielfachen und damit vervielfachen wir auch Frequenzabweichungen.
\end{sol}

\begin{sol}{ED403}
    Die Antwort sollte klar sein, die alternativen Antworten (B),(C),(D) machen überhaupt keinen Sinn.
\end{sol}

\begin{sol}{EF307}
    Hier geht es um Audio Signale vom Mikrofon. Das Menschliche Ohr kann bis maximal ca. 20k Hz hören, allerdings verwenden wird im Amateurfunk nur die untersten 2700Hz davon um nicht unnötig Bandbreite zu verschwenden. Die untersten 300Hz können wir nicht hören, deshalb kann ein Mikrofonverstärker mit der Kennlinie (A) auch als extra Filter dienen. 
\end{sol}

\subsection{Konverter und Transverter}
\begin{ohmchapter}
  Wir müssen Konverter und Transverter unterscheiden können.
  \begin{description}
  \item[Konverter] setzen das Signal nur in eine Richtung um (entweder im Sendepfad oder im Empfangspfad).
  \item[Transverter] verfügen über eine interne Sende-/Empfangsumschaltung und setzen das Signal in Sende- und Empfangsrichtung um (ähnlich wie ein Transceiver).
  \end{description} 
  Wenn also eine \qq{Sende-/Empfangsumschaltung} vorhanden ist, dann ist es ein Transverter.
\end{ohmchapter}

\begin{sol}{ED401}
 Die Frage ist einfach zu beantworten, hat aber mal wider viele ähnlich Antworten.
 Zunächst schließen wir Antwort (C) und (D) aus, da wir ja mit dem Verstärker die Ausgangsleistung erhöhen wollen. Der unterschied von (A) und (B) ist nur ab eine Spannungsquelle notwendig ist und auch dies ist einleuchtend, dass für ein Verstärkung Energie zugeführt werden muss. Deshalb brauchen wir ein Spannungsquelle.  
\end{sol}

\begin{sol}{ED401}
 Die Frage ist einfach zu beantworten, hat aber mal wider viele ähnlich Antworten.
 Zunächst schließen wir Antwort (C) und (D) aus, da wir ja mit dem Verstärker die Ausgangsleistung erhöhen wollen. Der unterschied von (A) und (B) ist nur ab eine Spannungsquelle notwendig ist und auch dies ist einleuchtend, dass für ein Verstärkung Energie zugeführt werden muss. Deshalb brauchen wir ein Spannungsquelle.  
\end{sol}

\begin{sol}{ED402}
 In der Schaltung finden wir ganz Zentral den Transistor, der ja typisch ist für den Verstärker, also schließen wir schon mal (D) aus. Weiterhin finden wir das Schaltzeichen eines Lautsprechers im Schema, es geht also um Audio (NF).   
\end{sol}

\begin{sol}{EF308}
  Bereits aus der Frage erfahren wir, dass es um einen NF-Verstärker geht, auch wenn zur Verwirrung noch Mixer und Bandpass eingezeichnet sind. Die Bezeichnungen SSB und LSB/USB lässt uns erkennen, dass es um das gewünschte Audiospektrum von ca. 2,5 kHz geht.
\end{sol}

\begin{sol}{EF403}
 Wichtig ist, dass wir uns merken, dass ein SSB Verstärker die Signale \textbf{linear} verstärken soll. Er muss dabei z.B. die gesamte Bandbreite des Signals gleichmäßig abdecken und sollte nicht bei gewünschten Frequenzen (SSB) oder Amplituden einbrechen (die Amplitude eines SSB Signals hängt von der Lautstärke des NF Signals ab).
\end{sol}

\begin{sol}{EF405}
  Die Stromversorgung in einem Sender, sollte niederohmig sein, um eine stabile und effiziente Energieversorgung der Senderendstufe zu gewährleisten. Die Antwort (C) und (D) macht ebenso keinen Sinn. Also merken wir uns, dass wir keine HF in der Stromzufuhr haben wollen. Bei Netzversorgung würden wir ja sonst auch die HF über das Stromnetz in der ganzen Nachbarschaft verteilen.
\end{sol}

\subsection{Verstärker}
\begin{ohmchapter}
Der Transistor ist für moderne Verstärker das Entscheidende Bauelement, dass uns hilft die Schaltungen aufeinander halten zu können. Für viele Jahre wurden auch Röhren verwendet, die auch heute noch viele Amateurfunker verwenden. Allerdings kommen sie nicht mehr im Fragenkatalog vor.
\end{ohmchapter}  



\setcounter{section}{8}
\section{Modulation} \label{sec:modulation}
\subsection{Unmodulierter Träger}


\begin{sol}{EE101}
  In (A) haben wir einen unmodulierten Sinus. (B) ist Frequenzmoduliert (C) ist Phasenmoduliert und (D) ist Amplitudenmoduliert. Schau Dir einfach an was sich abweichend von einem Sinus Signal in den Diagrammen ändert.
\end{sol}

\begin{ohmchapter}
  Der unmodulierter Träger entspricht im zeitlichen Verlauf eine Sinus Funktion.

\end{ohmchapter}  

\subsection{Einseitenbandmodulation (SSB)}

\begin{sol}{EE201}
  SSB unterscheidet sich von AM dadurch, dass nur eins von den beiden Seitenbändern hat und keinen Träger. In Bezug auf die Bandbreite ist es deshalb nur etwa halb so breit. Ansonsten unterscheidet sich SSB von AM nicht, Du kannst mit einem SSB Empfänger AM Empfangen, in dem Du deinen Empfänger auf jeweils eines der Seitenbänder einstellst.
\end{sol}

\begin{sol}{EE202}
  Die Bandbreite des NF Signals überträgt sich auf das HF Signal. Praktisch für Dich in der Prüfung, es gibt einige Fragen zur Bandbreite von NF und/oder SSB die nur minimal abweichen. Bei 2,4 kHz - 2,7 kHz liegst Du also fast immer richtig.

\end{sol}

\begin{sol}{EE203}
  Wir addieren, da das Signal im oberen Seitenband liegt (USB). Pass mit MHz bzw. KHz auf!
  
  Rechnung: 21,250Mhz + 1 kHz = 21,251 MHz 
\end{sol}

\begin{sol}{EE204}
  Wir subtrahieren, da das Signal im unteren Seitenband liegt (LSB). Pass mit MHz bzw. kHz auf!
  
  Rechnung: 3,65 Mhz + 2 kHz = 3,648 MHz 
\end{sol}


\begin{sol}{EE205}
  Die Amplitude des NF Signal regelt bei SSB die Ausgangsleistung. Wenn wir die Ausgangsleistung reduzieren wollen sollten wir die Amplitude des NF Signals reduzieren.
\end{sol}
\begin{sol}{EE206}
  Die Amplitude des NF Signal regelt bei SSB die Ausgangsleistung. Wenn unsere Mikrofonverstärkung nicht ausreicht haben wir auch nur eine geringe Ausgangsleistung.
\end{sol}
\begin{sol}{EE207}
  CW hat eine deutliche geringere Bandbreite als Sprachsignale via SSB oder AM. Deshalb ist es deutlich effektiver und erfreut sich großer Beliebtheit der der Welt des Amateurfunk.
\end{sol}

\begin{sol}{EF310}
Wie bei vielen anderen SSB Fragen ist die Antwort um 2,5 kHz richtig, also (A)!
\end{sol}


\begin{sol}{EJ210}
Wie bei vielen anderen SSB Fragen ist die Antwort um 2,5 kHz richtig, also (A). In dieser Frage liegt der Wert bei 2,7 kHz noch ca. 300 Hz zum (gefilterten) Träger Abstand sind. Dies entspricht den tiefen NF Frequenzen die wir Menschen nicht hören können.
\end{sol}

\begin{sol}{EJ215}
 Eine zu hohe Mikrofonverstärkung führt zu einer Übersteuerung der Verstärkerendstufe und zu Splatter auf die Nachbarfrequenzen.  Zudem machen wir es unserem Filter schwerer die Frequenzen außerhalb des Bandpass-Filter zu unterdrücken.
\end{sol}

\begin{ohmchapter}
  Wir haben die SSB Modulation bereits im Klasse N Kurs kennengelernt. Ein SSB Signal entspricht im Grunde der Amplitudenmodulation AM, bei der der Träger und ein Seitenband unterdrückt werden.
  
  Im Amateurfunk verwenden wir in der Regel die Audio Frequenzen von 300 Hz bis 3000 Hz, dies entspricht also in etwa 2.7 kHz. Auch die Bandbreite des ausgesendeten HF Seitenbandes ist in etwas so groß. Es gibt im Katalog viele Fragen zur Bandbreite von SSB oder des NF Signals die Du alle mit der Antwort um die 2.5-3 kHz richtig beantwortest.
  \pic[1.3]{ssb}
\end{ohmchapter}  

\subsection{Frequenzmodulation (FM)}
\begin{sol}{EE301}
  In diesem Bild ändert sich die Frequenz des Signals, wie sehen also FM.
\end{sol}

\begin{sol}{EE302}
  Schon beim Empfang von FM Rundfunk hast Du bestimmt bemerkt, dass FM klarer klingt. Das liegt u.A. daran dass FM nicht von der Amplitude abhängt, die von vielen Einflüssen z.B. in der Atmosphäre (QRN / QRM) beeinflusst wird.
  Früher haben auch die Zündung in Automotoren für Störungen in AM gesorgt, die mit FM nicht auftreten.
\end{sol}

\begin{sol}{EE303}
  FM wie in Frage EE302.
\end{sol}

\begin{sol}{EE304}
  Der Frequenzhub gibt an wie weit (Frequenz) der der Träger moduliert wird. Deshalb führt ein großer Frequenzhub zu einer großen HF Bandbreite.
\end{sol}

\begin{sol}{EE305}
 Wir müssen den Frequenzhub reduzieren.
\end{sol}

\begin{sol}{EE306}
  Wie der Name Frequenzmodulation (FM) bereits impliziert wird die Lautstärke (NF Amplitude) über die Trägerfrequenzauslenkung moduliert.  
\end{sol}

\begin{ohmchapter}
  Wie der Name Frequenzmodulation (FM) bereits verrät wird beim FM die Frequenz des HF Trägers moduliert (verändert). Der Hub gibt an wie weit die Frequenz von der Grundfrequenz abgelenkt wird. Hier wird das NF Signal und die entsprechende Auslenkung des HF Trägers gezeigt:
  \pic[0.8]{fm}
  Da FM über die Frequenz moduliert wird ist FM \textbf{unempfindlicher gegenüber Amplitudenstörungen}. 
\end{ohmchapter}  

\subsection{Bandbreite}
\begin{sol}{EA105}
  In Hertz (Hz).
\end{sol}

\begin{ohmchapter}
\end{ohmchapter}  

\subsection{Dynamikkompressor}

\begin{sol}{EF306}
  Da SSB von der Amplitude des NF (Audio) Signals abhängt, gebt der Dynamikkompressor schwache Audio Anteile an um ein stärkeres und klarer verständlicheres Signal zu erzeugen. Ist der Dynamikkompressor zu hoch eingestellt klingt das Signal aber unnatürlich und übermoduliert.
\end{sol} 
\begin{ohmchapter}
\end{ohmchapter}  


%\setcounter{section}{9}
\section{Empfänger}
\subsection{Detektorempfänger}
\begin{ohmchapter}
  Das in Frage EF101 gezeigte Schaltbild zeigt bereits alles was einen Detektorempfänger ausmacht. Wir haben keine externe Spannungsversorgung. Die Antenne fängt das HF Signal ein. Variabler Kondensator und eine Induktivität (Spule) bilden ein \textbf{Parallelschwingkreis} und selektieren die gewünschte Frequenz.
  Das Signal wird über eine Diode \textbf{gleichgerichtet}. Durch die Trägheit eines (hochohmigen) Kopfhörers wird ein hörbares NF Signal erzeugt.
  Die Nachteile sind klar: ohne Verstärker können nur sehr starke (AM) Stationen empfangen werden. Der Parallelschwingkreis ist sehr ungenau es wird ein großer Teil des Frequenzspektrums empfangen.
   Dennoch auch heute noch ein faszinierendes Bastelprojekt.
  
\end{ohmchapter}  


\begin{sol}{EF102}
  Die Zwischenfrequenz eines Überlagerungsempfänger hat hat den Vorzeit, dass mit speziellen Filtern eine höhere \textbf{Trennschärfe} erreicht werden kann.s
\end{sol} 

\begin{sol}{EF208}
 Der Direktempfänger mischt das HF Signal direkt auf Audiofrequenz NF. Im Mischer wird die Tatsache ausgenutzt, dass die Differenz der Frequenzen im gemischten Ausgang erzeugt wird. 
 Wenn jetzt Empfangsfrequenz und HF annähernd die selbe Frequenz haben kommt man also in der Differenz in den NF Bereich.
\end{sol} 


\subsection{Überlagerungsempfänger (Einfachsuper)}
\begin{ohmchapter}
  Wir haben im letzten Kapitel mit dem Detektorempfänger ein Beispieleines sogenannten \textbf{Geradeausempfänger} kennengelernt.
  Hier entsteht die Audio Frequenz direkt aus der HF. 
  Üblicher weise wird die HF direkt auf Audio Frequenz gemischt. Deshalb spricht man auch von einem \textbf{Direktüberlagerungsempfänger}.
  Es ist jedoch üblich zunächst auf eine feste \textbf{Zwischenfrequenz} zu mischen. Diese Art von Empfänger nennt man Überlagerungsempfänger.
  Der Vorteil besteht einer festen Zwischenfrequenz besteht darin, dass speziell für diese Zwischenfrequenz optimierte Filter verwendet werden können, z.B. für CW mit nur 300 Hz oder SSB mit 2400 Hz. Dadurch ergibt sich eine bessere \textbf{Trennschärfe}.
  \par
  \pic[0.8]{receiver} 
\end{ohmchapter}  
\subsection{Trennschärfe I }
\begin{ohmchapter}
Je kleiner die Empfangsbandbreite ist, desto enger ist auch mein Filter und das Signal wird deutlich besser. D.h. eine schmale Empfängerbandbreite führt zu einer hohen \textbf{Trennschärfe}. Für guten Empfang ist also eine schmale Bandbreite von Vorteil. Deshalb sind schmalbandige Übertragungsverfahren effektiver. Vergleiche z.B. CW mit SSB. 
\end{ohmchapter}  

\subsection{BFO I}

\begin{sol}{EF217}
  Seht starke Signale können einen Empfänger überlasten und müssen gedämpft werden. Dazu verwenden wir ein \textbf{Dämpfungsglied}.
\end{sol} 

\begin{ohmchapter}
Mit dem \qq{Beat Frequenz Oscillator} (BFO) wird in Überlagerungsempfänger die ZF auf Audio gemischt und damit hörbar gemacht.
\end{ohmchapter}  
\subsection{Vorverstärker und Dämpfungsglied}



\begin{sol}{EF218}
Im UHF (Ultra Hoch Frequenz) sind die Verluste auf den Zuleitungen besonders hoch. Im schlimmsten Fall ist das Nutzsignal durch diese Dämpfung bereits komplett im Rauschen verschwunden. Deshalb werden HF (Vor-)Verstärker im UHF Bereit möglichst direkt an der Antenne montiert.
\end{sol} 

\begin{ohmchapter}
\end{ohmchapter}  

\subsection{Automatische Verstärkungsregelung (AGC) I} \label{agc}
\begin{ohmchapter}
  \textbf{AGC} steht für Automatic Gain Control oder auf Deutsch auch Automatische Verstärkerregelung. 

  Sie steuert der HF Verstärker automatisch nach. Wenn sehr starte Signale empfangen werden reduziert die AGC die Verstärker Leistung, wenn die Signale schwach sind regelt die AGC die Verstärkung nach oben.
  Dadurch wird das NF signal stabiler.

  Achtung: Die AGC regelt den Empfänger. Es gibt eine Verstärkerregelung für den Sender (ALC), diese solltest Du in der Prüfung nicht verwechseln.
\end{ohmchapter}  

\subsection{Notch-Filter}
\begin{sol}{EF216}
Der Notchfilter ist ein Kerb-filter, d.h. er filtert nur einen kleinen Teil des Frequenzspektrums heraus, lässt den übrigen Teil des NF Spektrums durch. Es ist also die Kerbenform von (A).
\end{sol} 
\begin{ohmchapter}
Sehr \textbf{schmalbandige Störungen} (QRM) können mit einem Kerbfilter auch \textbf{Notch-filter} eliminiert werden.
\end{ohmchapter}



\subsection{Rauschunterdrückung}
\begin{ohmchapter}
  Die \textbf{Rauschunterdrückung}, auch auf Englisch als Noise Reduction(NR) benannt dient der Unterdrückung von Rauschen.
  Der \textbf{Noise Blanker} hingegen eliminiert impulsartige Störungen, wie sie z.B. früher von Motor Zündungen erzeugt wurden.
\end{ohmchapter}  


\subsection{Frequenzmessung I}


\begin{sol}{EI502}
Der Zähler zeigt MHz an. Dies bezieht sich auf den Punkt hinter der Ziffer 5.
Wir Zählen die Stellen durch:
\begin{itemize}
  \item $5 \cdot \SI{1} {\mega\hertz}$
  \item $0 \cdot \SI{100} {\kilo\hertz}$
  \item $0 \cdot \SI{10} {\kilo\hertz}$  
  \item  $\underbrace{1 \cdot \SI{1} {\kilo\hertz}}_{\text{Stelle mit X}}$
\end{itemize}
\end{sol} 

\begin{sol}{EI503}
Der Zähler zeigt MHz an. Dies bezieht sich auf den Punkt hinter der Ziffer 5.
Wir Zählen die Stellen durch:
\begin{itemize}
  \item $5 \cdot \SI{1} {\mega\hertz}$
  \item $0 \cdot \SI{100} {\kilo\hertz}$
  \item $0 \cdot \SI{10} {\kilo\hertz}$
  \item $1 \cdot \SI{10} {\kilo\hertz}$
  \item $3 \cdot \SI{100} {\hertz}$
  \item $\underbrace{7 \cdot \SI{10} {\hertz}}_{\text{Stelle mit X}}$
\end{itemize}
\end{sol} 

\begin{sol}{EI504}
 Ein 10:1 Frequenzteiler hat die Frequenz um einen einen Faktor 10 reduziert, aus \SI{10}{\mega\hertz} wurde \SI{1}{\mega\hertz} in der Anzeige.
 Für die Aufgabe müssen wir den angezeigten Wert mit 10 multiplizieren.
 $$ \SI{14,5625}{\mega\hertz} \cdot 10 = \SI{145,625}{\mega\hertz}$$
 Check: die Frequent liegt im 2-Meter Amateurfunkband.
\end{sol}  


\begin{ohmchapter}
Frequenzzähler sind nützliche Messgeräte die, wie der Name bereits andeutet, um die Frequent eines Signals zu messen. Genauer gesagt: die Frequenz eines unmodulierten Hochfrequenzsignals.  Dies kann z.B. genutzt werden um die Frequenz z.B. eines lokalen Oszillator (LO) zu bestimmen.  
\end{ohmchapter}  


%
\setcounter{section}{10}
\section{Sender}
\subsection{ALC}
\begin{ohmchapter}
Wir haben bereits im Kapitel \ref{agc} über die AGC gesprochen. Dies ist eine automatische Verstärker Steuerung für den Empfänger.
Aber auch der Sender hat hat solch eine Steuerung: Automatic Level Control (ALC). Wie Du im Kapitel über SSB gelernt hast hängt hier die Signalstärke von der Amplitude des NF Signals ab, welches ganz natürlich schwankt wenn wir in das Mikrofon sprechen. 
Um diese Schwankungen entgegenzuwirken und damit die Endstufe zu schützen, reduziert die ALC Signalstärke (Amplitude) wenn sie über ein definiertes Limit geht.
\end{ohmchapter}  
\subsection{Senderausgangsleistung}

\begin{sol}{EF401}
Die \qq{Ausgangsleistung} ist natürlich die Leistung direkt am Senderausgang (vor Zusatzgeräten). Vor- bzw. rücklaufende Leistung spielen keine Rolle.
\end{sol}  
\begin{sol}{EF402}
Die Peak Envelop Power (PEP) oder auf Deutsch maximale Hüllkurvenleistung wird direkt am Senderausgang gemessen.
Mit der Klasse E sind oft 100 W zulässig. Eine Antenne mit Gewinn kann die Abstrahlung noch verstärken.
\end{sol}  

\begin{sol}{EJ209}
Hier geht es mit die Leistung die zu unerwünschten Aussendungen führt. Deshalb wird hier auch Stehwellenmessgerät und ggf. ein verwendeter Tiefpassfilter berücksichtigt werden.
\end{sol}  


\begin{ohmchapter}
  Die Definitionen der Senderausgangsleistung musst Du Dir einfach merken. Wie Du bereits bei der Klasse N gelernt hast, bist Du als  Funkamateur verpflichtet dich an entsprechende Grenzwerte zu halten.
\end{ohmchapter}

\subsection{Unerwünschte Aussendungen II}

\begin{sol}{EJ201}
Nur sinusförmige Schwingungen haben keine Oberwellen. In dieser 
\href{https://fritzsche.github.io/klasse-e/oberwelle.html}{App} kannst Du ausprobieren welche Oberwellen unterschiedliche Signale haben.

\end{sol}  

\begin{sol}{EF404}
Wenn die Senderendstufe neu eingestellt wurde wollte zur Sicherheit überprüft werden, dass keine Oberwellen entstehen.
Wenn nach der Einstellung z.B. kein reiner Sinus mehr erzeugt wird sind Oberwellen dabei.
\end{sol}  


\begin{sol}{EJ202}
Hochfrequente Störungen durch Harmonische werden durch Tiefpassfilter gefiltert. Hier wird explizit nach einen Oberwellenfilter gefragt. Nicht irritieren lassen!
Die Frage EJ203 wird der Begriff Tiefpassfilter verwendet.
\end{sol} 

\begin{sol}{EJ203}
In dieser Fragen finden wir schnell, dass uns ein Tiefpassfilter hilft.
Du kannst einen Tiefpassfilter in dieser \href{https://fritzsche.github.io/klasse-e/lpf.html}{App}
praktisch ausprobieren, in dem ein Rechtecksignal durch bei geeigneten Parametern durch einen Tiefpassfilter zu einem Sinus Signal wird.
\end{sol}  

\begin{sol}{EJ204}
Der Tiefpassfilter ist mal wieder die richtige Antwort.
\end{sol}  


\begin{sol}{EJ205}
Auch für UHF Sender wird man einen Tiefpassfilter verwenden, wenn man Oberwellen unterdrücken will.
\end{sol}  


\begin{sol}{EJ206}
Es gibt mehrere Fragen nach denen dem Schaltbild eines Filters gefragt wird. Bei all Fragen kann man sich die Position des Kondensators im Vergleich zur Spule ansehen. Ist der Kondensator \qq{unten} so handelt es sich um einen Tiefpass, sonst om einen Hochpass.
Da Oberwellen mit einem Tiefpassfilter gedämpft werden bleibt nur diese Antwort.
\end{sol}  

\begin{sol}{EJ206}
Es gibt mehrere Fragen nach denen dem Schaltbild eines Filters gefragt wird. Bei all Fragen kann man sich die Position des Kondensators im Vergleich zur Spule ansehen. Ist der Kondensator \qq{unten} so handelt es sich um einen Tiefpass, sonst om einen Hochpass.
Da Oberwellen mit einem Tiefpassfilter gedämpft werden bleibt nur diese Antwort. Beachte bitte, dass dies natürlich im allgemeinen nicht gild, da es davon abhängt wie der Schaltplan gezeichnet wurde, aber die Schaltpläne des aktuelle Fragekatalogs wurden alle so gezeichnet, dass die Regel gilt. 
\end{sol}  


\begin{sol}{EJ207}
Ein Filter zur Verringerung von Oberwellen ist ein Tiefpassfilter. D.h. die Tiefen Frequenzen werden ungehindert durchgelassen, die hohen Frequenzen werden abgeschwächt.
In der Frage geht es um einen Kurzwellen-Sender, d.h. wir wollen alle Frequenzen unterhalb von \SI{30}{\mega\hertz} durchlassen und nur oberhalb filtern. Dies finden wird in Bild A.
\end{sol}  


\begin{sol}{EJ208}
Wie bei Frage EJ207: Ein Filter zur Verringerung von Oberwellen ist ein Tiefpassfilter. D.h. die Tiefen Frequenzen werden ungehindert durchgelassen, die hohen Frequenzen werden abgeschwächt.
In der Frage geht es um einen Kurzwellen-Sender, d.h. wir wollen alle Frequenzen unterhalb von \SI{30}{\mega\hertz} durchlassen und nur oberhalb filtern. Dies finden wird in Bild A.
\end{sol}  

\begin{ohmchapter}
  Unerwünscht Aussendungen entstehen oft durch \textbf{Oberwellen}. Diese können in der Regel durch einen \textbf{Tiefpassfilter} vermieden werden.
\end{ohmchapter}


\subsection{Störende Beeinflussung elektronischer Geräte I}


\begin{sol}{EJ101}
Siehe Definition am Anfang des Kapitel.
\end{sol}  

\begin{sol}{EJ102}
Siehe Definition am Anfang des Kapitel.
\end{sol}  


\begin{sol}{EJ103}
Das Schlüsselwort ist \textbf{Übersteuerung}. D.h. das Signal ist einfach zu stark und überlasten den Empfänger in der Nähe.
\end{sol}


\begin{sol}{EJ104}
Dies ist leider ein Grundsatz der oft zu wenig berücksichtigt wird. 
\end{sol}

\begin{sol}{EJ105}
Nicht mit unnötig hoher Sendeleistung zu senden lohnt sich immer.
\end{sol}

\begin{sol}{EJ106}
Mit hohem Gewinn senden wir ein Signal in der Nachbarschaft von TV Kanälen aus. Dies kann den Empfänger im TV Gerät übersteuern.
\end{sol}

\begin{sol}{EJ107}
Bei sehr starken Signalen wird ein Empfänger (z.B. AGC) die Verstärker zurückfahren um eine Übersteuerung zu vermeiden.
Deshalb geht die Empfindlichkeit zurück.
\end{sol}

\begin{sol}{EJ108}
Das Abschirmgehäuse ist in der Regel aus Metall um unerwünschte Aussendungen abzufangen.
\end{sol}

\begin{sol}{EJ109}
Ohne Abschirmung können HF Signale in die 230 V Wechselstromleitung gelangen und dann über die Leitung in andere Geräte einströmen.
\end{sol}


\begin{sol}{EJ111}
Diese Frage lässt sich gut durch das Ausschlussprinzip beantworten. Aber auch direkt mach es Sinn eine getrennte HF Erdung zu verwenden.
\end{sol}

\begin{sol}{EJ112}
Einströmung via Netzanschluss.
\end{sol}

\begin{sol}{EJ113}
Die abgeschaltete Stereoanlage verhält sich hier wir ein Detektorenempfänger. 
\end{sol}

\begin{sol}{EJ114}
Geschirmte Laubsprecherbkabel können Einstrahlungen reduzieren.
\end{sol}

\begin{sol}{EJ115}
Abschirmen.
\end{sol}


\begin{sol}{EJ116}
Die Frequenz von \SI{28} {\mega\hertz } liegt im 10 m Band am oberen Ende der Kurzwelle. Dies sollte aus der Klasse N noch bekannt sein. Die TV Signale liegen vie höher. Wir wollen also hohe Frequenzen durchlassen (TV) aber die niedrigen HF Signale unterdrücken. Wir brauchen also einen Hochpassfilter.
\end{sol}


\begin{sol}{EJ117}
Wie in Frage EJ116 brauchen wir einen Hochpassfilter. Nach der Merkregel sind bei einem Hochpass Filter die Kondensatoren auch oben. Also Schaltbild A.
\end{sol}

\begin{sol}{EJ118}
Eine so genannte Mantelwellensperre oder auch Mantelwellendrossel reduziert Gleichtaktströme. Dies ist HF die sich z.B. auf dem Außenmantel von Koaxialleitungen bilden könnte.
\end{sol}

\begin{sol}{EJ119}
Wie in EJ118 kann eine Mantelwellendrossel diese unerwünschten Gleichtaktströme verringern und muss in das Koax vor dem Empfänger eingebaut werden.
\end{sol}


\begin{sol}{EJ120}
Es werden Mischfrequenzen erzeugt, Phantomsignale. Wir haben bereits über den Mischer gesprochen und eine der Frequenzen abgeschaltet wird verschwindet natürlich auch das Signal.
\end{sol}

\begin{sol}{EJ121}
Wir merken uns die Antwort.
\end{sol}

\begin{sol}{EJ122}
Das einfachste zuerst! Passen die Störungen überhaupt zeitlich zum Funkbetrieb?
Es ist bereits oft vorgekommen, dass die Störungen auftreten obwohl wohl die Amateurfunkanlage nicht in Betrieb war.
\end{sol}

\begin{sol}{EJ124}
Eine Zimmerantenne ist für den Empfang nicht optimal und führen zu einem schlechten Signal Rauschabstand. Durch sind im Verhältnis auch die Störungen viel Stärker (siehe auch AGC). Eine AUßenantenne kann die Situation verbessern.
\end{sol}

\begin{sol}{EJ212}
AFSK (Audio Frequency Shift Keying) wirkt hier wie FM. Durch absenken des audio Pegel reduzieren wir also die Bandbreite.
\end{sol}

\begin{sol}{EJ213}
Ist der Leistungsverstärker übersteuert, so sieht das Signal nicht mehr wie ein Sinus sondern eher wie ein Rechtecksignal. Dies führt zu viel Oberwellen. Aber auch Benachbarte Frequenzen werden durch sogenannte Splatter beeinflusst.
\end{sol}

\begin{sol}{EJ214}
Insbesondere im SSB Bereich sind Splatter (Störungen auf Nachbarfrequenzen) zu beobachten, wenn die Leistungsendstufe übersteuert wird. In einem modernen SDR Transceiver kann man sofort erkennen, dass dabei die 
übliche Bandbreite von ca. \SI{2.4} {\kilo\hertz }  deutlich überschritten wird.
\end{sol}

\begin{sol}{EJ216}
  Die Antwort sollte unmittelbar klar sein.
\end{sol}


\begin{ohmchapter}
Auch in diesem Kapitel geht es um unterschiedliche Störungen die mir einem Sender verursacht werden können.
Wir unterscheiden zwei unterschiedliche Arten von Störungen.

\begin{description}
  \item[Einströhmungen] Die Störung, bzw. die HF wird durch eine Zuleitung, z.B. Netzzuleitung, Antennenzuleitung, Lautsprecherkabel etc. verursacht. 
  \item[Einstrahlung] Die HF gelangt direkt in der gestörte Gerät, z.B. da die Abschirmung nicht ausreicht.
\end{description}

\end{ohmchapter}


%\setcounter{section}{11}
\section{Digitale Übertragungsverfahren}
\subsection{Binäres Zahlensystem} \label{sec:binär}



\begin{sol}{EA202}
  Wir haben also 3 Bits. Wir können die Zahl $111_2$ berechnen:
    \begin{center}
  \begin{tabular}{|c|c|c|c|c|c|c|c|} 
    128& 64&32&16 &8 & 4 & 2 & 1 \\ \hline
    0&0&0&0&0&1&1&1
  \end{tabular} 
  \end{center}
Also $111_2 = 4+2+1 = 7$. Dies ist aber keine der möglichen Antworten! Wir haben die 0 vergessen, es wurde ja nach \qq{unterschiedliche(n) Zuständen} gefragt! Also ist 8 die richtige Antwort.

Es geht auch einfacher $1000_2$ die nächst höhere Zahl ist können wird direkt mit $8 = 2^3$ antworten.
\end{sol}


\begin{sol}{EA203}
Analog zu Frage \qref{EA202} rechnen wir $2^4 = 16$.
\end{sol}

\begin{sol}{EA204}
Analog zu Frage \qref{EA202} rechnen wir $2^5 = 32$.
\end{sol}


\begin{sol}{EA205}
  Wir rechnen:
  \begin{center}
\begin{tabular}{|c|c|c|c|c|c|c|c|} 
128& 64&32&16 &8 & 4 & 2 & 1 \\ \hline
0&1&0&0&1&1&1&0
\end{tabular}

\end{center}
Also: $64+8+4+2 = 78$ 
\end{sol} 

\begin{sol}{EA206}
  Wir rechnen:
  \begin{center}
\begin{tabular}{|c|c|c|c|c|c|c|c|} 
128& 64&32&16 &8 & 4 & 2 & 1 \\ \hline
1&0&0&0&1&1&1&0
\end{tabular}

\end{center}
Also: $128+8+4+2 = 142$ 
\end{sol}  


\begin{sol}{EA207}
  Wir rechnen:
  \begin{center}
\begin{tabular}{|c|c|c|c|c|c|c|c|} 
128& 64&32&16 &8 & 4 & 2 & 1 \\ \hline
1&0&0&1&1&1&0&0
\end{tabular}
\end{center}
Also: $128+16+8+4 = 156$ 
\end{sol}  


\begin{sol}{EA208}
  Wir rechnen:
  \begin{center}
\begin{tabular}{|c|c|c|c|c|c|c|c|} 
128& 64&32&16 &8 & 4 & 2 & 1 \\ \hline
1&1&1&1&1&0&0&0
\end{tabular}
\end{center}
Also: $128+64+32+16 = 248$ 
\end{sol}  

\begin{ohmchapter}
  Wir verwenden im Alltag üblicherweise Zahlen im Dezimalsystem. D.h wir verwenden die Ziffern 0 bis 9 alle Zahlen darzustellen. 
  In der Digitaltechnik hat sich allerdings hauptsächlich das Binärsystem durchgesetzt in dem nur die Ziffern 0 und 1 verwendet werden, die durch zwei Zustände (Strom ist ein oder aus) abgebildet werden können.  Das Binärsystem wird manchmal auch Dualsystem genannt.
   Eine Ziffer die im Binärsystem (0 oder 1) wird auch Bit genannt.
  Üblicherweise fassen wir 8 Bit zu einer Zahl zusammen und nennen es Byte.
  Für die Stellen einer 8 Bit Zahl gilt: 
  \begin{align*}
    2^7 &= 128 \\         
    2^6 &= 64 \\
    2^5 &= 32 \\  
    2^4 &= 16 \\         
    2^3 &= 8  \\
    2^2 &= 4 \\         
    2^1 &= 2  \\   
    2^0 &= 1  
  \end{align*}  
Tipp für die Prüfung: viele Schul-Taschenrechner (nicht programmierbar und deshalb vielleicht von der Bundesnetzagentur für die Prüfung akzeptiert) können zwischen Zahlsystemen umrechnen.
Praktisch ausprobieren kannst Du die Umrechnung auch mit dieser kleinen App auf \href{https://fritzsche.github.io/klasse-e/binary.html}{Github}.

\end{ohmchapter}


\subsection{Digimode per SSB}

\begin{sol}{EE402}
  Wie im Eingang zu diesem Kapitel beschrieben wird in der Regel SSB verwendet.
\end{sol}


\begin{sol}{EE403}
Wie bereits im Kapitel zur Modulation erklärt sind bei SSB NF unf HF Bandbreite identisch.
\end{sol}


\begin{sol}{EE404}
Über die Audio Schnittstelle wird am PC die komplette Bandbreite von \SI{2.4}{\kilo\hertz} empfangen. 
Digitale Signale haben oft eine Bandbreite von nur wenigen Hertz und können somit gleichzeitig empfangen werden.
Hier ein Bild eines Wasserfalls mit Spektrum. Im Audio Empfangsbereich liegen viele Signale:
  \begin{center}
    \includegraphics[scale=.5]{bilder/ft8.png}    
  \end{center}
\end{sol}

\begin{sol}{EE415}
  Du musst die einfach merken, dass SSTV (Slow Scan TV) Bilder sind. Diese werden z.B. auch von der ISS im \SI{2}{\meter}  Band gesendet. Hier als Anschauung ein SSTV Bild:
  \begin{center}
    \includegraphics[scale=.15]{bilder/sstv.jpg}    
  \end{center}
  
\end{sol}


\begin{ohmchapter}
  Digitale Signale werden oft nicht direkt (nativ) vom Funkgerät verarbeitet.
  Vielmehr wird das Funkgerät im Modus SSB betrieben, die Audio Signale kommen aber natürlich nicht via Mikroton sondern werden per USB Audio interface von einem Computer erzeugt.
\end{ohmchapter}


\subsection{9600-Port }

\begin{sol}{EF219}
Wie wir im Eingang zu diesem Kapitel gelernt haben umgeht der 9600-Port den Audio Bereich des Funkgeräts und steuert direkt den Demodulator an. Der Demodulator ist zwischen 3 und 4. Wir wählen 4, da dies nach der NF Verarbeitung aber vor dem Demodulator ist.
\end{sol}

\begin{sol}{EF309}
Wir gehen analog zu Frage \qref{EF219} vor. Es kommt nur 2 in Frage, da dies nach dem NF Bandpassfilter aber vor dem Modular ist. Achtung: die Position 1 ist im NF Teil des Senders. Bitte nicht verwirren lassen.
\end{sol}

\begin{ohmchapter}
Der 9600-Port dient der schnellen digitalen Datenkommunikation (z.,B. Packet Radio, APRS) mit 9600 Baud. Er umgeht die sprachoptimierte Audioverarbeitung des Transceivers durch direkte Einspeisung analoger Audiosignale vom externen Modem/TNC in den Frequenzmodulator. Dies ermöglicht höhere Übertragungsraten im Vergleich zu herkömmlichen 1200-Baud-Verbindungen über den Mikrofonanschluss. Die übertragenen Signale sind analog (AFSK/FSK Töne), nicht digital im Sinne von TTL-Pegeln.
Damit hat der 9600-Port eine größere Bandbreite als die Anbindung via SSB die wir im vorherigen Kapitel kennengelernt haben und steuert den Modulator/Demodulator direkt an.

In der Praxis ist der 9600-Port oft mit Data bezeichnet wie bei diesem Yaesu FT-710.

  \begin{center}
    \includegraphics[scale=.3]{bilder/data_port.jpg}    
  \end{center}


\end{ohmchapter}






\subsection{Übersteuerung }

\begin{sol}{EJ217}
  Wenn die ALC eingreif ist das Audio Signal zu hoch eingestellt. Dadurch kann es zu Störungen auf den Nachbarfrequenzen kommen.
\end{sol}

\begin{sol}{EJ218}
Wie schon seit Frage \qref{EJ217} bekannt wollen wir nicht, dass die ALC aktiv wird. Allerdings sollte der Pegel natürlich möglichst hoch sein. Deshalb stellen wir den Pegel genau so hoch, dass die ALC gerade so keinen Ausschlag hat. Dies können wir z.B. für den FT-8 Betrieb machen in dem wir der Audio Regler (am Computer) des Audio Interface hochdrehen und dabei die ALC beobachten. Wenn die ALC ausschlägt gehen wir mit der Lautstärke noch etwas herunter. 
\end{sol}


\begin{sol}{EJ219}
Wie in Frage \qref{EJ218} erklärt reduzieren wir den NF-Pegel (Lautstärke) noch etwas.
\end{sol}

\begin{ohmchapter}
Wir haben gelernt, dass viele Digitale Signale über ein Audio Interface und dem Transceiver im SSB Modus erzeugt werden. Wir wissen auch, dass es in SSB auf den Audio Pegel ankommt um einen störungsfreien Betrieb durchzuführen.
\end{ohmchapter}


\subsection{Automatische Empfangsberichte}


\begin{ohmchapter}
Viele Stationen Empfangen Radio Signale und verbreiten diese via Internet (WebSDR),
Besonders im digitalen Bereich könne diese Signale automatisch dekodiert werden und der Empfang kann an zentrale Server berichtet werden. Dort können sie zentral eingesehen werden.
Z.B. werden automatisiert dekodierte CW Signale vom \href{https://www.reversebeacon.net/}{Reverse Beaken Network} gesammelt.
FT8 und PSK vom \href{https://pskreporter.info}{PSK reporter}.

Hier ist die Web-Seite von pskreporter zu sehen in der DN9KAI überprüft hat ob sein Digitalen Signale auch empfangen werden. In diesem Fall hat auch DP0GYN (Van Neumayer Station Antarktis) einen Empfangsbericht geschickt:
  \begin{center}
    \includegraphics[scale=.2]{bilder/DP0GVN.png}    
  \end{center}

\end{ohmchapter}



\subsection{Paketvermittelte Netzwerke}


\begin{sol}{EE412}
Wie allgemein bekannt werden Informationen im Internet via Paketen verteilt.
Das Internet besteht dabei aus vielen kleinen Netzwerken die diese Pakete austauschen. Du hast vermutlich bei Dir zuhause einen Internet Router (in vielen Fällen ist dies eine FritzBox). Er stellt für Dich die Verbindung von Deinem lokalen Netzwerk (WLAN) zu allen anderen Netzwerken her. D.h. wenn Du an Deinem Computer eine Internetseite öffnest, werden eine oder mehrere Pakete erzeugt die zunächst alle an Deinen Router gehen. Der Leitet sie dann an das Netzwerk Deines Internet Providers weiter und solange \textbf{weitergeleitet} bis das Paket den Zielserver erreicht.
\end{sol}

\begin{sol}{EE413}
Die IP-Adresse und die Subnetzmaske definieren zusammen das lokale Netzwerk, indem sie bestimmen, welcher Teil der IP-Adresse die Netzwerk-ID (das lokale Netz) und welcher Teil die Host-ID (ein spezifisches Gerät innerhalb dieses Netzes) identifiziert.
\end{sol}

\begin{sol}{EE414}
  Für diese Frage ist die Musterantwort schlecht formuliert. Das Internet ist zunächst ein Netzwerk, dass das Internet Protokoll (IP) befolgt. Diese IP Pakete können auch mit Amateurfunk weitergeleitet werden, wobei z.B. das Rufzeichen in höheren Netzwerkebene (z.B. TCP) ausgetauscht werden. Wir merken uns die Formulierung der Musterantwort.
\end{sol}


\begin{ohmchapter}
In diesem Kapitel haben wir einige Fragen zu den Grundlagen eines \textbf{Paketvermittelten Netzwerk}. Es geht konkret um das IP-Protokoll, dass dem Internet wie Du es kennst zu Grunde liegt. Die Details verrate ich jeweils bei den Fragen.

\end{ohmchapter}


\subsection{Amplituden- und Frequenzumtastung (ASK, FSK)}

\begin{sol}{EE406}
Nur in der Musterlösung A ändert sich die Amplitude.
\end{sol}


\begin{sol}{EE407}
Nur in der Musterlösung A ändert sich die Frequenz.
\end{sol}

\begin{ohmchapter}
  Im Kapitel \ref{sec:modulation} zur Modulation hast Du bereits FM und AM kennengelernt. 
  Diese Arten der Modulation lassen sich auch auf digitale Übertragungsverfahren anwenden.
  Da die Grundlage der Modulation hier digital ist sprechen wir von einer \textbf{Umtastung}.

  \begin{description}
    \item[ASK] (Amplitude Shift Keying) oder auf Deutsch Amplitudenumtastung.
    Hier werden für 0 bzw. 1 jeweils unterschiedliche Amplituden gesendet.
    \item[FSK] (Frequency Shift Keying) oder auf Deutsch Frequenzumtastung. Es werden unterschiedliche Frequenzen gesendet. 
  \end{description}

\end{ohmchapter}

\subsection{AFSK}
\begin{ohmchapter}
Wir haben bereits ASK und FSK im letzten Kapitel kennengelernt.
Bei AFSK (Audio Shift Keying) wird das NF Signal digital umgetastet, in dem verschiedene Tonhöhen für 0 und 1 erzeugt werden. Dies wird dann z.B. via FM moduliert und gesendet (also FSK).
Ein bekanntest AFSK Signal ist z.B. APRS (Automatic Packet Reporting System), dass in Europa auf \SI{144,800}{\mega\hertz} gesendet wird.

\end{ohmchapter}

\subsection{Datenübertragungsrate}

\begin{sol}{EE401}
Wie wissen die Bandbreite wird in Hertz angegeben es geht um den genutzten Frequenzbereich. Die Datenübertragungsrate aber in Bits pro Sekunde also der Datenmenge die pro Sekunde übertragen wird.

\end{sol}

\begin{ohmchapter}
  In diesem Kapitel geht es um die Datenübertragungsrate. Wir haben im Kapitel \ref{sec:binär} zum binären Zahlensystem bereits gelernt, was ein \textbf{Bit} ist. 
  Die Datenübertragungsrate gibt einfach an wie viele \textbf{Bits pro Sekunde} übertragen werden.
\end{ohmchapter}  


\subsection{Vielfachzugriff}

\begin{sol}{EE409}
Das T in TDMA steht für \qq{time} also \textbf{Zeit}.
\end{sol}

\begin{sol}{EE410}
Das F in FDMA steht für \qq{frequency} also \textbf{Frequenz}.
\end{sol}

\begin{sol}{EE411}
Das C in CDMA steht für \qq{code}, hier geht es um  \textbf{Spreizcodes}.
\end{sol}

\begin{ohmchapter}
In der drahtlosen Kommunikation sind \textbf{Frequenzmultiplex (FDMA)}, \textbf{Zeitmultiplex (TDMA)} und \textbf{Codemultiplex (CDMA)} die zentralen Verfahren, um das gemeinsame Frequenzspektrum effizient unter mehreren Nutzern aufzuteilen und Interferenzen zu minimieren. Die Wahl des Verfahrens hängt von den spezifischen Anforderungen an Bandbreite, Nutzerzahl und Robustheit ab.

\begin{description} 
    \item[\textbf{FDMA (Frequency Division Multiple Access)}]
     \ \\
    \begin{itemize}
        \item  \textit{Funktionsweise}: Das Frequenzband wird in mehrere getrennte Frequenzkanäle unterteilt, wobei jeder Kanal einem einzelnen Nutzer fest zugewiesen wird (\textbf{Trennung über Frequenz}).
        \item \textit{Kurzcharakteristik}: Einfaches, etabliertes Verfahren; jedoch bandbreitenineffizient bei vielen Nutzern.
        \item \textit{Anwendungsbeispiele}: Analoge Mobilfunknetze (z.B. AMPS), Satellitenkommunikation.
    \end{itemize}

    \item[\textbf{TDMA (Time Division Multiple Access)}]
    \ \\
    \begin{itemize}
        \item \textit{Funktionsweise}: Alle Nutzer teilen sich denselben Frequenzkanal, erhalten aber nacheinander in festgelegten Zeitintervallen Zugriff auf den Kanal (\textbf{Trennung über Zeit}).
        \item \textit{Kurzcharakteristik}: Hohe Frequenzeffizienz; erfordert jedoch präzise Synchronisation der Zeitschlitze.
        \item \textit{Anwendungsbeispiele}: GSM (2G Mobilfunknetze), DECT.
    \end{itemize}

    \item[\textbf{CDMA (Code Division Multiple Access)}]
    \ \\
    \begin{itemize}
        \item \textit{Funktionsweise}: Alle Nutzer nutzen denselben Frequenzkanal zur gleichen Zeit. Die Trennung erfolgt über individuelle, orthogonale \textbf{Spreizcodes}.
        \item \textit{Kurzcharakteristik}: Höchste Flexibilität und Kapazität; sehr robust gegen Störungen; erfordert aber komplexe Signalverarbeitung.
        \item \textit{Anwendungsbeispiele}: UMTS (3G Mobilfunknetze), GPS.
    \end{itemize}
\end{description}


\noindent \textbf{Zusammenfassung:} FDMA ist die einfachste Methode, während TDMA und insbesondere CDMA zunehmend effizienter und komplexer werden. CDMA bietet die größte Flexibilität bei begrenzter Bandbreite und vielen Nutzern, erfordert jedoch auch die technologisch aufwendigste Umsetzung.



\end{ohmchapter}  

%\setcounter{section}{12}
\section{Digitale Signalverarbeitung}
%\subsection{Digitale Signalverarbeitung}
% Es gibt kein Unterkapitel
\setcounter{subsection}{1}

\begin{sol}{EF601}
Wie eingangs des Kapitels beschrieben können brauchen wie zunächst einen A/D-Umsetzer und nach der Digitalen Verarbeitung einen D/A-Umsetzer.
\end{sol}

\begin{sol}{EF602}
digitalisiert werden.
\end{sol}

\begin{sol}{EF603}
Die Abkürzung SDR steht für Software Defined Radio. Hier wird mindestens ein Teil der Signalverarbeitung in Software realisiert. Manchmal wird dazu bereits sehr früh im HF-Teil des Empfängers digitalisiert, manchmal wird auch ein Teil z.B. die IF in klassischen Schaltungen realisiert und nur der NF Bereich ist Digital.

Einige SDR's sind über das Internet kostenlos erreichbar, z.B. über sogenannte \href{http://websdr.org}{WebSDR}.

\end{sol}

\begin{ohmchapter}
Um ein Signal digital verarbeiten zu können müssen wir es zunächst digitalisieren.  
Um ein analoges Signal zu Digitalisieren brauchen wir einen sogenannten \textbf{A/D-Umsetzer}. Das Blockschaltdiagramm sieht folgendermaßen aus:
\begin{center}
\begin{circuitikz}
\draw (0, 0) node[adcshape,box]{} (0, 0);
\end{circuitikz}
\end{center}
Natürlich können wir auch umgekehrt ein Digitales Signal in ein analoges umsetzen. Dies ist ein \textbf{D/A-Umsetzer} und das Blockschaltdiagramm sieht folgendermaßen aus.
\begin{center}
\begin{circuitikz}
\draw (0, 0) node[dacshape,box]{} (0, 0);
\end{circuitikz}
\end{center}



\end{ohmchapter}  



%\setcounter{section}{13}
\section{Antennen und Übertragungsleitungen}
\subsection{Polarisation II}


\begin{sol}{EB306}
Wir müssen und daran orientieren in welche Richtung das elektrische Feld in Bezug auf die Erde orientiert ist.
Die elektrische Feldkomponente (mit E bezeichnet) ist horizontal.
\end{sol}

\begin{sol}{EB307}
Analog zu Frage \qref{EB305}, nur ist das elektrische Feld hier vertikal ausgerichtet.
\end{sol}

\begin{sol}{EB308}
Die elektrische Feldkomponente dreht sich quasi im Kreis in Bezug auf Erde und Hauptstahlrichtung. Wir nennen dies als zirkulare Polarisation.
\end{sol}

\begin{sol}{EB309}
Bei allen dipolartigen Antennen können wir die Orientierung des Strahlers orientieren.
Dies hilt auch für diesen Beam bei dem der Strahler offenbar horizontal orientiert ist.
\end{sol}


\begin{sol}{EB310}
Analog zu Frage \qref{EB309}, die Ausrichtung des Strahlers ist vertical orientiert.
\end{sol}


\begin{ohmchapter}
    Die Polarization einer Antenne wird nach der Richtung der Hauptstahlrichtung in Bezug zur Erdoberfläche angegeben.
\end{ohmchapter}



\subsection{Antennenformen II}

\begin{sol}{EG101}
  Es gibt verschiedene Schleifenantennen, die wie der Name schon verrät eine Schleife bilden. Sie können z.B. als Quadrat oder Dreieck aufgespannt werden. In der Konfiguration als Dreieck sprechen wir von einer \textbf{Delta-Loop-Antenne}.
\end{sol}


\begin{sol}{EG103}
 Wir sehen die Antenne besteht aus einen Draht der offenbar am Ende gespeist wird, also eine sogenannte \textbf{Endgespeiste Antenne}.
 Die hohe Impedanz (einige tausend Ohm) einer Endgespeisten Antenne müssen wir auf die 50 Ohm anpassen die das Funkgerät erwartet. Wie haben nur noch die Option A und C über.
 In diesem Fall können wir vielleicht den Parallelschwingkreis erkennen. Dies ist eine sogenannte Fuchs Antenne (benannt nach Josef Fuchs) mit einfachem Anpassglied. 
\end{sol}

\begin{sol}{EG104}
    Siehe Frage \qref{EG103}.
\end{sol}


\begin{sol}{EG105}
    Die meisten Antennen strahlen über die elektrische Komponente des elektromagnetischen Felds.
    Das Schlüsselwort \qq{magnetisches Feld} finden wir bereits in der Frage. Wie wählen also die Antwort \textbf{magnetische Ringantenne}. Die Antenne ist beliebt, da sie eine relativ kleine Bauform von nur etwas $\frac{\lambda}{10}$ hat.
\end{sol}

\begin{sol}{EG106}
 Die einzige Antenne die wir hier noch nicht explizit besprochen haben ist die \textbf{Windom-Antenne}. Dies ist eine Mehrband Drahtantenne für die Kurzwelle. 
Die anderen Antenne aus Antwort (A) sollten Dir bekannt vorkommen.
Gegenprobe: (B) eine Parabolantenne kennst Du vermutlich als Satellitenschüssel. Für Kurzwelle mit Sicherheit zu groß.
(C) eine Patchantenne kann direkt auf einer Leiterbahn verwendet werden. GPS Empfänger verwenden manchmal solche Antenne. Nichts für die Kurzwelle.
(D) Ein Hornstrahler ist eine Mikrowellen-Anntenne (z.B. für Radio Astronomie), also keine Kurzwelle.
\end{sol}

\begin{sol}{EG107}
    Eine W3DZZ Antenne ist eine für 80m geeignete Antenne die mit Sperrkreisen arbeitet. Oft wird die W3DZZ auf 40m und 80m betrieben.
Wir können wieder einige Antennen aus den Antworten für das 80m Band ausschließen.
(B) Sowohl die Kreuz-Yagi-Uda Antenne wie auch die Groundplane währen für 80 m einfach zu groß.
(C) Eine Sperrkopfantenne ist eher für 70cm und (D) der Parabolspiegel müsste für 80 m gigantisch sein.
\end{sol}


\begin{sol}{EG108}
Eine $5/8-\lambda$ Antenne ist gegenüber einer $\lambda / 4$ Antenne zunächst länger. Im 70 cm Band ist dies aber oft noch unkritisch.
Bauartbedingt hat sie mehr Gewinn. Für die Prüfung kannst Du es Dir einfach so merken Längere Antenne ergibt mehr Gewinn. (gilt natürlich nicht im Allgemeinen, aber hoffentlich für diese Frage.)  
\end{sol}


\begin{sol}{EG213}
Eine symmetrische Antenne ist, wie der Name vermuten lässt symmetrische aufgebaut. Was wichtigste Beispiel ist der Dipol, der wegen der gleich aufgebauter Dipolhälften symmetrische ist.
Dies ist für (B),(C) und (D) der Fall. Hier wird aber gefragt, welche Antenne \underline{nicht} symmetrisch ist. Die Groundplane hat keine zweite Dipolhälfte und dafür Radials. Sie ist \underline{nicht} symmetrisch.
\end{sol}


\begin{sol}{EG214} 
    Wir erkennen Strahlungsdiagramm (B) als Beam (z.B. Yagi-Uda) (ähnlich Dipol aber mit Gewinn in eine Richtung) und (C) als Groundplane (wir sehen 3 Radials). 
    Für einen Dipol gilt, dass die Hauptstahlrichtung wie in (A) senkrecht zur Aufspannrichtung der Dipolhälften ist.    
\end{sol}


\begin{sol}{EG215} 
    Analog zur Frage \qref{EG214}.
\end{sol}

\begin{sol}{EG216} 
    Analog zur Frage \qref{EG214}.
\end{sol}


\begin{sol}{EG217} 
    Analog zur Frage \qref{EG214}.
\end{sol}

\begin{sol}{EG219} 
    Bereits eine $\lambda /4$ Vertikalantenne hat eine flache Abstrahlung. 
\end{sol}


\begin{ohmchapter}
Antennen sind für den Funkamateur eines der wichtigsten Themen. Die perfekte Antenne für alles gibt es nicht, jede Antenne bringt unterschiedliche Vor- und Nachteile mit sich.
Wie fangen direkt mit den Fragen an umd klären Vor- und Nachteile.
\end{ohmchapter}


\subsection{Antennenlänge und -resonanz}

\begin{sol}{EG102} 
    Diese Frage ist etwas verwirrend für jeden der zunächst an $\lambda / 2$, $\lambda / 4$ die in den falschen Antworten genannt werden.
    Tatsächlich kann man sehr viele verschiedene Längen von Drahtantennen anpassen damit sie auf einem Amateurfunk Kurzwellenband resonant ist.
    Dies ist die \qq{richtige} Antwort.

    Realitätscheck: Das ist dies natürlich nicht general korrekt: die Länge ist \underline{nicht} beliebig. Ist die Antenne viel zu kurz, so wird selbst im perfekt angepassten Aufbau diese Antenne einen so schlechten Wirkungsgrad haben, dass sie quasi unbrauchbar ist.
\end{sol}


\begin{sol}{EG109}
    Rechnung: 
    $$\lambda = \frac{300}{\SI{28,5}{\mega\hertz}} \approx 10,53$$
    $$\frac{5}{8}\cdot \lambda = \frac{5}{8} \cdot 10,53 \approx  6,58$$
\end{sol}

\begin{sol}{EG110}
  Ein \textbf{Faltdipol} ist quasi eine plattgedrückte Ganzwellenschleifen. Die Drahtlänge ist eine Wellenlänge.
\end{sol}


\begin{ohmchapter}
In diesem Kapitel geht es um die Resonanz von Antenne. Hier geht es aber nur um 3 relativ einfache Fragen.
\end{ohmchapter}


\subsection{Verkürzungsfaktor I}

Die Lichtgeschwindigkeit beträgt im Vakuum $c= \SI{299792458} {\meter/\second}$. Wir haben die bereits für die Klasse N verwendet um die Wellenlänge zu berechnen:
$$ \lambda = \frac{c}{f}$$.

Die Lichtgeschwindigkeit ist allerdings in Leitungen (z.B. Antennendrähten) etwas langsamer. Nach einer Faustregel ist die Geschwindigkeit etwa 95\%. Der Verkürzungsfaktor $k_v$ gibt dies an. Also nach Faustregel: $k_v  \approx 0.95$.

Für die Wellenlänge gilt:
$$ \lambda_{\text{Leitung}} = k_v \cdot \frac{c}{f}$$.

In der Realität gibt es unterschiedliche Verkürzungsfaktoren abhängig von der Art der Leitung. Oft gibt es ein Datenblatt in dem man genauere Angaben finden kann.

Wenn Du die Länge eines Antennendrahtes berechnest solltest Du trotz Berücksichtigung des Verkürzungsfaktor in der Regel immer ein 10-15\% längeres Stück abschneiden, dass kannst Du dann immer noch trimmen:

\qq{Abschneiden ist einfacher als dranschneiden.}

\begin{sol}{EG201}
Die falschen Antworten enthalten immer Begriffe / Dinge die mit der Verkürzungsfaktor zu tun haben. Selbst wenn Du die Formel nicht im Kopf hast. Du solltest Dir merken, dass der \textbf{Verkürzungsfaktor} etwas mit der Ausbreitungsgeschwindigkeit im Vakuum zu tun hat und landest sofort bei (A).
\end{sol}

\begin{sol}{EG202}
 Die Faustregel nach der der Verkürzungsfaktor etwa 95\% ist musst Du dir merken.
\end{sol}

\subsection{Fußpunktimpedanz I}

\begin{sol}{EG207}
Der Halbwellendipol ist die bekannteste Antenne. Du musst Dir merken, dass die Impedanz (wenn noch montiert) nicht \SI{50}{\ohm} beträgt sondern etwas höher ist.
Die Beträgt etwa \SI{75}{\ohm}. Die falschen Antworten kannst Du ggf. auch ausschließen.
\end{sol}

\begin{sol}{EG208}
    In Frage \qref{EG207} haben wir den Dipol mit \SI{75}{\ohm} angegeben. In der Praxis lieft es oft niedriger und stellt für unseren Transceiver kein Problem dar.    
\end{sol}

\begin{sol}{EG209}
    Analog zu Frage \qref{EG208}.
\end{sol}

\begin{sol}{EG210}
    Das kannst Du Dir die richtigen Werte merken: nimm den größten Wert. Generell gilt für den Faltdipol, dass die Spannung verdoppelt wird und der benötigte Strom sich halbiert. Dies entspricht einer Vervierfachung. 
\end{sol}

\begin{sol}{EG211}
    Die Groundplane Antenne ist ja eine Art von Dipol (also nur eine Dipolhälfte + Radials). Da oft auch sehr bodennah, kannst Du Dir merken, dass wir einen niedrigen Fußpunktwiederstand haben. Die wählen also Antwort (A) die auch unsere markanten \SI{50}{\ohm} enthält.
\end{sol}


\begin{ohmchapter}
Wir haben den Begriff \textbf{Impedanz} bereits als Wechselstromwiderstand kennengelernt.
Bei der  \textbf{Fußpunktimpedanz} geht es um die Impedanz am Einspeisepunkt der Antenne.

Unsere Transceiver erwarten in der Regel eine Impedanz von \SI{50}{\ohm}.
Unterschiedliche Antennen und Aufbauvarianten (z.B. Höhe) haben unterschiedliche Fußpunktimpedanz. Diese musst Du einfach lernen.

\end{ohmchapter}


\subsection{Yagi-Uda Antenne II}

\begin{sol}{EG211}
    Wie im Eingang des Kapitels beschrieben.
\end{sol}


\begin{sol}{EG212}
    Wie im Eingang des Kapitels beschrieben.
\end{sol}


\begin{sol}{EG218}
    Das Strahlungsdiagramm zeigt eine klare Richtcharakteristik. D.h. viel mehr Leistung geht nach rechts als nach links. Dies ist typisch für einen Beam.
    Nur Antwort (A) enthält mit der Yagi-Uda Antenne einen Beam der in Frage kommt.
\end{sol}

\begin{ohmchapter}
Die Yagi-Uda Antenne wurde ab 1924 von den Japanern Hidetsugu Yagi und Shintaro Uda entwickelt. 

Der generelle Aufbau ist vielen zugmindestems durch TV- und Rundfunk Antennen bekannt, grob gesprochen besteht sie aus unterschiedlichen Elementen die in Hauptstahlrichtung immer kleiner werden. 
Eines der Elemente ist der \textbf{Strahler}. Oft ist er als Dipol oder als Faltdipol ausgeführt. Er hat den Einspeisepunkt der ganzen Antenne.

Die Elemente länger als der Strahler werden \textbf{Reflektor} genannt. Die Elemente kürzer als der Strahler werden \textbf{Direktor} genannt. 

\end{ohmchapter}



\subsection{Parabolspiegel I}

\begin{sol}{EG113}
   Wie suchen eine Antenne für den Mikrowellenbereich die \qq{scharf bündelt}. Hier solltest Du die Parabolantenne kennen. Die Frage Antwort (B) ist auch etwas gemein formuliert: den isotropen Strahler gibt es in der Praxis ja nicht. Also geht es um die Erregerantenne (Feed).
\end{sol}

\begin{sol}{EG114}
Merke dir einfach, dass die Parabolantenne sehr groß ist.
\end{sol}

\begin{ohmchapter}
    Wenn Du bei einem Parabolspiegel an eine Satellitenschüssel denkst, dann ist das genau richtig. Diese Art der Antenne kann einen sehr großen Antennengewinn haben, da alle Radiowellen am sogenannten \textbf{Spiegelkörper} zu einem zentralen Punkt gebündelt werden. 
    
    \pic[1.5]{parabol}
    
    Da der Spiegelkörper mindestens fünf Wellenlängen entsprechen sollte, ist dies nur etwas für kleine Wellenlängen. In diesem Frequenzbereich von \SI{1}{\giga\hertz} bis \SI{300}{\giga\hertz} sind wir im so genannten Mikrowellenbereich.
\end{ohmchapter}


\subsection{Strom- und Spannungsspeisung I}

\begin{sol}{EG204}
    Wir haben in der Einleitung zu diesem Kapitel gelernt, dass der Dipol an seinem Einspeisepunkt (in der Mitte) einen Strombauch und Spannungsknoten hat.
\end{sol}


\begin{sol}{EG205}
    Die Situation ist genau umgekehrt zu Frage \qref{EG204}. Also etwa wir bei einer endgespeisten Antenne.
\end{sol}

\begin{sol}{EG205}
    Siehe Frage \qref{EG204}.
\end{sol}

\begin{ohmchapter}
    Wir wollen uns in diesem Kapitel damit beschäftigen, wie sich Strom und Spannung auf einer Antenne verteilen. Wenn wir dies für den Dipol aufmalen sieht es in etwa so aus:
    \pic[1]{strom_spannung}

    Du kannst Dir diese Verteilung ganz einfach merken, in dem Du betrachtest, was für eine Situation wir an den Enden des Dipols haben. Da der Leiter hier physikalisch zu Ende ist, kann hier kein Strom mehr fließen, der Widerstand ist also unendlich groß. Nach dem Ohmschen Gesetz haben wir dann auch unendlichen Spannung. Dies gilt zunächst mathematisch, in der Realität liegt die Impedanz natürlich nicht bei Unendlich, ist aber schon ca. \SI{6000}{\ohm}. Wir sprechen von einem \textbf{Stromknoten} und einem \textbf{Spannungsbauch}.

    Die Situation am Einspeisepunkt des Dipols ist genau umgekehrt. Hier fließt bei niedriger Impedanz der meiste Strom und deshalb ist die Spannung niedrig. Wie sprechen von einem \textbf{Strombauch} und einem \textbf{Spannungsknoten}.

    Eine Antenne die an einem Strombauch gespeist wird heisst auch \textbf{stromgespeist}. Umgekehrt eine Antenne mit Speisung an einem Spannungsbauch wird auch \textbf{spannungsgespeist} genannt.

\end{ohmchapter}


\subsection{Bauch und Knoten von Strom und Spannung}

\begin{sol}{EG203}
    Siehe Frage \qref{EG204}.
\end{sol}


\begin{ohmchapter}
\end{ohmchapter}


\subsection{Antennengewinn in dBi und dBd}


\begin{sol}{EG220}
    Der Buchstabe i in dBi verrät es: Isotropenstrahler.
\end{sol}


\begin{sol}{EG221}
    Wir müssen den Gewinn von 2,15 dB des Dipols addieren. 
    Tipp: den Gewinn des Dipols finest Du in der Formelsammlung.
    Rechnung: $$ \SI{5}{dBd} + \SI{2,15}{\dB} = \SI{7,15}{dBi}$$ 
\end{sol}

\begin{ohmchapter}
    Wir erinnern uns, dass die Angabe dBi ein Antennengewinn in Bezug auf den Isotropenstrahler ist und dBd den Gewinn in Bezug auf den Dipol. Der Unterschied sind die 2,15 dBi des Dipols. 
\end{ohmchapter}


\subsection{Standortwahl}

\begin{sol}{EG112}
Eine Richtantenne kann mit ihrem hohen Gewinn schnell zu Störungen verursachen. Um dies zu vermeiden sollte die Antenne so hoch und weit weg wie möglich montiert werden.
\end{sol}

\begin{sol}{EG223}
Im Haus verlaufen viel teils ungeschirmte Leistungen. Um zu vermeiden, dass die Sendeantenne HF in diese Leitungen koppelt ist es ratsam Sendeantennen außerhalb des Haus zu montieren.
\end{sol}

\begin{sol}{EJ110}
Erinnere Dich an die Richtcharakteristik des Dipols. Wenn der Draht rechtwinklig gespannt ist geht die meiste Strahlung nicht in Richtung der Häuserzeile.
\end{sol}

\begin{ohmchapter}
\end{ohmchapter}

\subsection{Übertragungsleitungen}

\begin{sol}{EG301}
  Wie in im Kapitel beschrieben ist die Impedanz unseres Übertragungsleitung konstant.
\end{sol}

\begin{sol}{EG302}
 Mit gutem Koaxkabel können wir die Kabeldämpfung gering halten und vermeiden, dass Störungen einstrahlen. 
\end{sol}


\begin{sol}{EG303}
Merken! Der N-Stecker gerne auch im UKW Bereich verwendet.
\end{sol}

\begin{sol}{EG304}
In diesem Kapitel haben wir bereits Koaxkabel und die parallele Zweidrahtleitung erwähnt. Bei einem Koaxkabel denen die beiden Leiter völlig anders aus, dies ist eine unsymmetrische Speiseleitung. Dahingehend sind bei einer symmetrischen Zweidrahtleitung (wie der Name schon sagt) beide Leitungen gleich geformt.
\end{sol}

\begin{sol}{EG305}
Eine Paralleldrahtleitung hat eine sehr geringe Dämpfung und ist deshalb relativ populär.
\end{sol}


\begin{sol}{EG306}
    Wenn HF- und Netzkabel parallel liegen, kann HF in die Netzkabel einströmen.
\end{sol}


\begin{ohmchapter}
    In diesem Kapitel soll es um die Übertragungsleitungen gehen, also um die Leistungen die wir verwenden um den Transceiver mit der Antenne zu verwenden.  In der Regel verwenden wir Koaxkabel, aber es gibt auch alternativen wie eine parallele Zweidrahtleitung. 

    \begin{center}
    \includegraphics[scale=.30]{bilder/Zweidrahtleitung.jpeg}    
    \end{center}

    Unsere Übertragungsleitungen, z.B. Koaxkabel haben auf der Kurzwelle einen annähern Konstanten Wert, der von der Leitungscharakteristik und z.B. dem Aufbau der Abschirmung abhängt.
\end{ohmchapter}

\subsection{Kabeldämpfung I}

\begin{sol}{EG307}
Einfache Rechnung für die beiden Kabelstücke gilt:
$$ \SI{2}{\dB} + \SI{3}{\dB} = \SI{5}{\dB}$$
\end{sol}

\begin{sol}{EG308}
Einfache dB Überlegung: die Leistung halbiert sich auf der Übertragungsleitung (bei optimalen SWR von 1).
Ein Faktor 2 entspricht aber genau \SI{3}{\dB} (Leistungsverhältnis).
Dies hast Du Dir gemerkt, oder du findest sie in der Formelsammlung im Abschnitt Pegel. Da es um Dämpfung geht ist der positive Wert \SI{3}{\dB} richtig und \SI{-3}{\dB} falsch.
\end{sol}

\begin{sol}{EG309}
Aus Frage \qref{EG308} wissen wir, dass \SI{3}{\dB} der Hälfte Entspricht.
Ein viertel ist die Hälfte der Hälfte. In dB müssen wir die Werte addieren:
$$ \SI{3}{\dB} + \SI{3}{\dB} = \SI{6}{\dB}$$

Alternativ kannst Du auch den Faktor 4 in der DB Tabelle der Formelsammlung finden und direkt auf \SI{6}{\dB} kommen.
\end{sol}

\begin{sol}{EG310}
    Analog zu Frage \qref{EG309}. Ein Faktor 10 entspricht \SI{10}{\dB}.
\end{sol}

\begin{sol}{EG311}
    Dreisatz: $$ \frac{\SI{20}{\meter}}{\SI{100}{\meter}}  \cdot \SI{20}{\dB} =  \SI{4}{\dB}$$
\end{sol}


\begin{sol}{EG312}
 Du musst im Diagramm ablesen. Pass auf, dass Du die Kurve für das richtige Kabel verwendest. Einige Angaben klingen zu ähnlich. Prüfe, dass Du die  Kurve  mit Koax mit \SI{4,95}{\milli\meter} Durchmesser verwendest.
 Du kannst \SI{20}{\dB} direkt ablesen.
\end{sol}

\begin{sol}{EG313}
 Zunächst wie in Frage \qref{EG312} ablesen. Die \SI{20}{\dB} gelten aber für \SI{100}{\meter}. Du brauchst also den Dreisatz wie in Frage  \qref{EG311}.
 $$ \frac{\SI{15}{\meter}}{\SI{100}{\meter}}  \cdot \SI{20}{\dB} =  \SI{3}{\dB} $$
\end{sol}

\begin{sol}{EG314}
Wie in den vorherigen Fragen lesen wir zunächst im Kabeldämpfungsdiagramm ab und finden \SI{40}{\dB} für \SI{100}{\meter}. Ein Kabel, dass nur halb so lang ist hat also  \SI{20}{\dB} Dämpfung.
\end{sol}


\begin{sol}{EG315}
Wie in den vorherigen Fragen lesen wir zunächst im Kabeldämpfungsdiagramm ab und finden \SI{7}{\dB} für \SI{100}{\meter}. 

 $$ \frac{\SI{40}{\meter}}{\SI{100}{\meter}}  \cdot \SI{7}{\dB} =  \SI{2,8}{\dB} $$

\end{sol}

\begin{sol}{EG316}
Wie in den vorherigen Fragen lesen wir zunächst im Kabeldämpfungsdiagramm ab und finden \SI{20,5}{\dB} für \SI{100}{\meter}. 

 $$ \frac{\SI{40}{\meter}}{\SI{100}{\meter}}  \cdot \SI{20,5}{\dB} =  \SI{8,2}{\dB} $$

 Selbst wenn Du z.B. nicht exakt \SI{20,5}{\dB} abgelesen hast, bist Du noch nahe genug am Ergebnis um Antwort (A) zu wählen.

\end{sol}


\begin{ohmchapter}
Die Kabeldämpfung unterschiedlicher Arten von Koaxkabel unterscheidet sich. In der Regel wird die Dämpfung pro \SI{100}{\meter} Kabel in Datenblättern angegeben.
Für die Prüfung bekommst das folgende Diagram als Anhang zur Formelsammlung, in der Du alle Werte ablesen kannst. Du musst also für die Kabeldämpfung nichts auswendig lernen.

    \pic{daempfung}
\end{ohmchapter}


\subsection{Stehwellenverhältnis (SWR) II}


\begin{sol}{EG401}
 \SI{100}{\watt} bei SWR 3 bedeutet nach Regel (25\%) also  \SI{25}{\watt} rücklaufende Leistung.
\end{sol}


\begin{sol}{EG402}
 Die Regel die wir uns gemerkt haben.
\end{sol}


\begin{sol}{EG403}
 Analog zu \qref{EG402}. Nur wird hier nach der vorlaufenden Leistung gefragt. Die müssen dann 75\% sein.
\end{sol}


\begin{ohmchapter}
Wir merken uns die Angabe, dass ein SWR von 3 bedeutet, dass in etwa 25\% der Leistung reflektiert wird.bedeutet.
\end{ohmchapter}

\end{document}