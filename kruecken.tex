\documentclass[12pt,a4paper,ngerman]{article}
\usepackage[utf8]{inputenc}
\usepackage{tkz-euclide}
\usepackage[T1]{fontenc}

%\usepackage[]{ntheorem}
%\usepackage[amsthm,thmmarks]{ntheorem}
\usepackage{amsmath}
%\usepackage{amsthm}
\usepackage{hyperref}
\usepackage{cleveref}

\usepackage{amsthm}
\usepackage{thmtools}
\usepackage{amsfonts}
\usepackage{amssymb}
\usepackage{babel}
\usepackage{textcomp}

\newtheoremstyle{mytheorem}
  {} % Space above (default: 3pt)
  {} % Space below (default: 3pt)
  {\itshape} % Body font: set to italic (standard for plain theorem style)
  {} % Indent amount
  {\bfseries} % Head font: set to bold
  {.} % Punctuation after theorem head (e.g., Theorem 1.1.)
  {.5em} % Space after theorem head (e.g., .5em or " " for normal interword space)
  {\thmname{#1}\thmnumber{ #2}\thmnote{ (\textbf{#3})}}


\theoremstyle{definition}
\newtheorem{defn}{Definition}
\newtheorem*{example}{Beispiel}

\theoremstyle{plain}
\theoremstyle{mytheorem}
\newtheorem{theorem}{Satz}
\newtheorem{lemma}{Lemma}
\newtheorem*{axiom}{Axiom}

\newcommand{\R}{\mathbb{R}}
\newcommand{\N}{\mathbb{N}}
\newcommand{\Q}{\mathbb{Q}}
\newcommand{\Z}{\mathbb{Z}}
\newcommand{\Prim}{\mathbb{P}}
\newcommand{\Nnull}{\mathbb{N}_0}

%*****************************************
\usepackage{adjustbox}
%\usepackage{stackengine}
%\usepackage{paralist}
\usepackage{xfp}
\usepackage{siunitx} % Wichtig für Zahlenformatierung
\usepackage{xparse}
\usepackage{xstring}
\usepackage{enumitem}

\usepackage{svg}
%\usepackage{array}

\theoremstyle{definition}
\newtheorem*{auf}{Aufgabe}

\usepackage{circuitikz}[european]


%\newcommand{\frage}[6]{%
%  \item[#1] #2
%  \begin{enumerate}
%    \item #3
%    \item #4
%    \item #5
%    \item #6
%  \end{enumerate}
%}

\newcommand{\qgrafic}[1]{%
\begin{center}
    \begin{tikzpicture}[scale=1.3]
        \input{pic/#1_q.tex}        
    \end{tikzpicture}
\end{center}    
}


\newcommand{\agrafic}[2]{%
%  \begin{center}
      \begin{tikzpicture}[scale=1.5]
          \input{pic/#1_#2.tex}        
      \end{tikzpicture}
%  \end{center}     
}


\ExplSyntaxOn
\NewDocumentCommand{\frage}{ m m m m m m m m }{
  \item[#1] #2
    \tl_if_eq:nnTF {#7} {true} {
        \qgrafic{#1}
    }{}

\tl_if_eq:nnTF {#8} {true} {
\begin{center}
\begin{tabular}{cc}
  (A) \adjustbox{valign=c}{\agrafic{#1}{a}} & (B) \adjustbox{valign=c}{\agrafic{#1}{b}} \\
  (C) \adjustbox{valign=c}{\agrafic{#1}{c}} & (D) \adjustbox{valign=c}{\agrafic{#1}{d}} \\
\end{tabular}
\end{center}


%\begin{enumerate}[label=(\Alph*)]
%    \item \adjustbox{valign=c}{\agrafic{#1}{a}} 
%    \item \adjustbox{valign=c}{\agrafic{#1}{b}} 
%    \item \adjustbox{valign=c}{\agrafic{#1}{c}} 
%    \item \adjustbox{valign=c}{\agrafic{#1}{d}}   
%  \end{enumerate}        
} {
  \begin{enumerate}
    \item #3
    \item #4
    \item #5
    \item #6
  \end{enumerate}      
}
  %\begin{enumerate}
%    \item \tl_if_eq:nnTF {#8} {true} {        
%        \agrafic{#1}{a}
%    }{#3}
%    \item \tl_if_eq:nnTF {#8} {true} {        
%        \agrafic{#1}{b}
%    }{#4}
%    \item \tl_if_eq:nnTF {#8} {true} {        
%        \agrafic{#1}{c}
%    }{#5}
%    \item \tl_if_eq:nnTF {#8} {true} {        
%        \agrafic{#1}{d}
%    }{#6}
%  \end{enumerate}    
}
\ExplSyntaxOff

\newenvironment{ohmchapter}
{

}
{

  \subsubsection*{Lösungen}
  \input{tex_sections/\arabic{section}S\arabic{subsection}.tex}
}



\newcommand{\DEnumber}[1]{%
    \num[
        parse-numbers=true,          % Wichtig: Erzwingt die Analyse der Zahl
        output-decimal-marker={,},   % Für das Komma als Dezimaltrennzeichen
        group-separator={\,},        % Optional: Dünner Zwischenraum als Tausender-Trennzeichen
        scientific-notation=false    % Sicherstellen, dass keine wissenschaftliche Notation verwendet wird
    ]{#1}%
}

\newcommand{\calc}[1]{%
  \DEnumber{\fpeval{#1}}
}

\newcommand{\mischer}[3]{%
    \item[#1]\
    \begin{itemize}
        \item \DEnumber{#2} MHz + \DEnumber{#3} MHz = \calc{#2+#3} MHz
        \item \DEnumber{#2} MHz + \DEnumber{#3} MHz = \calc{#2-#3} MHz
    \end{itemize}  
}

\author{Thomas Fritzsche}
\title{Lernmaterial für den Amateurfunkkurs der Klasse E von A02}
\begin{document}

\setcounter{section}{7}
\section{Grundlegende Schaltungen}
\setcounter{subsection}{3}

\subsection{Mischer (Klasse E)}

\begin{ohmchapter}





In einem Mischer werden zwei Eingangssignale zu einem Ausgangssignal gemischt. Das Blockschaltdiagramm eines Mischers sieht aus wie eine Waschmaschine.
Beim Mischen entsteht aus den beiden Eingangsfrequenzen die Summe und Differenz Frequenz:

%$$f_{\text{out 1}} = f_{\text{in 1}} + f_{\text{in 2}} \textrm{\ und\ }f_{\text{out 2}} = | f_{\text{in 1}} - f_{\text{in 2}} |$$
%Als Blockschaltdiagramm sieht es folgendermaßen aus:
\begin{center}
\begin{adjustbox}{margin=10pt}
\begin{circuitikz}
\draw (0,0) node[mixer,boxed] (M) {}; 
\draw[->] (-2, 0)node[left] {$f_{\text{in 1}}$} -- (M.w); 
\draw[->] (0, -2)node[below] {$f_{\text{in 2}}$} -- (M.s); 
\draw[->] (M.out) -- (2, 0) node[right,text width=5cm] {
$\begin{aligned}
      &f_{\text{out 1}} = f_{\text{in 1}} + f_{\text{in 2}}\\
      &f_{\text{out 2}} = | f_{\text{in 1}} - f_{\text{in 2}} | 
    \end{aligned}$
    }; 
\end{circuitikz}
\end{adjustbox}
\end{center}
\end{ohmchapter}


\subsubsection{Lösungen:}
\begin{description}
    \mischer{EF201}{31.7}{21}
    \mischer{EF202}{38.7}{28}
    \mischer{EF203}{39}{30}
    \mischer{EF204}{145}{136}
    \mischer{EF205}{145}{136}
    \item[EF206] Sie sollte gut abgeschirmt sein.    
\end{description}  

\subsection{Konverter und Transverter}
\begin{ohmchapter}
  \begin{description}
  \item[Konverter] setzen das Signal nur in eine Richtung um (entweder im Sendepfad oder im Empfangspfad).
  \item[Transverter] verfügen über eine interne Sende-/Empfangsumschaltung und setzen das Signal in Sende- und Empfangsrichtung um (ähnlich wie ein Transceiver).
  \end{description}  
\end{ohmchapter}
%\begin{enumerate}
%    \renewcommand{\labelenumi}{(\alph{enumi})}    
%    \item Ja, mit der Zustimmung des Schiffsführers
%    \item Ja, mit der Zustimmung eines beliebigen Crewmitglieds
%    \item Ja, mit einer Genehmigung der BNetzA
%    \item Ja, mit einer Genehmigung des Bundesamtes für Seeschifffahrt und Hydrographie
%\end{enumerate}

%\begin{description}
%\frage{VE706}
    {Darf eine Amateurfunkstelle auch an Bord eines Schiffes, welches sich in internationalen Gewässern befindet, betrieben werden?}
    {Ja, mit der Zustimmung des Schiffsführers}
    {Ja, mit der Zustimmung eines beliebigen Crewmitglieds}
    {Ja, mit einer Genehmigung der BNetzA}
    {Ja, mit einer Genehmigung des Bundesamtes für Seeschifffahrt und Hydrographie}
    {false}{false}
%\end{description}




%% Kapitel 8, Sektion 4: Mischer

\begin{description}
    \frage{EF201}
    {Welche wesentlichen Ausgangsfrequenzen erzeugt die in der Abbildung dargestellte Stufe?}
    {10,7 MHz und 52,7 MHz}
    {42 MHz und 63,4 MHz}
    {21 MHz und 63,4 MHz}
    {21,4 MHz und 105,4 MHz}
    {true}{false}
    \frage{EF202}
    {Einem Mischer werden die Frequenzen 28 MHz und 38,7 MHz zugeführt. Welche Mischfrequenzen werden hauptsächlich erzeugt?}
    {10,7 MHz und 66,7 MHz}
    {17,3 MHz und 49,4 MHz}
    {56 MHz und 77,4 MHz}
    {45,3 MHz und 88,1 MHz}
    {false}{false}
    \frage{EF203}
    {Welches sind die erwünschten Produkte, die bei der Mischung der Frequenzen 30 MHz und 39 MHz am Ausgang des Mischers entstehen?}
    {9 MHz und 69 MHz}
    {9 MHz und 39 MHz}
    {30 MHz und 39 MHz}
    {39 MHz und 69 MHz}
    {false}{false}
    \frage{EF204}
    {Einem Mischer werden die Frequenzen 136 MHz und 145 MHz zugeführt. Welche Mischfrequenzen werden hauptsächlich erzeugt?}
    {9 MHz und 281 MHz}
    {127 MHz und 154 MHz}
    {272 MHz und 290 MHz}
    {118 MHz und 163 MHz}
    {false}{false}
    \frage{EF205}
    {Welches sind die erwünschten Produkte, die bei der Mischung der Frequenzen 136 MHz und 145 MHz am Ausgang des Mischers entstehen?}
    {9 MHz und 281 MHz}
    {127 MHz und 154 MHz}
    {272 MHz und 290 MHz}
    {154 MHz und 281 MHz}
    {false}{false}
    \frage{EF206}
    {Wie sollte eine Mischstufe beschaffen sein, um unerwünschte Abstrahlungen zu vermeiden?}
    {Sie sollte gut abgeschirmt sein.}
    {Sie sollte niederfrequent entkoppelt werden.}
    {Sie sollte nicht geerdet werden.}
    {Sie sollte möglichst lose mit dem VFO gekoppelt sein. }
    {false}{false}
\end{description}

%% Kapitel 1, Sektion 1: Der erste Schritt

\begin{description}
    \frage{NA102}
    {Aus 250 m Draht sollen Antennen hergestellt werden. Pro Antenne werden 18,5 m benötigt. Wie viele Antennen können maximal aus dem vorhandenen Draht hergestellt werden?}
    {13}
    {14}
    {12}
    {15}
    {false}{false}
    \frage{NA103}
    {Laut Datenblatt wiegen 100 m eines bestimmten Drahtes 210 g. Ein vorliegendes Drahtstück desselben Materials wiegt 55 g. Wie lang ist das Drahtstück in etwa?}
    {26,2 m}
    {382 m}
    {115 m}
    {38,2 m}
    {false}{false}
    \frage{NA101}
    {Ein 20 m langer Draht wird bei 2/3 seiner Länge zertrennt. Wie lang sind die resultierenden Stücke in etwa?}
    {13,33 m und 6,67 m}
    {12,22 m und 7,78 m}
    {11,11 m und 8,89 m}
    {14,44 m und 5,56 m}
    {false}{false}
    \frage{VD102}
    {Was gilt in Bezug auf den Empfang von Amateurfunkaussendungen?}
    {Es ist keine Zulassung zur Teilnahme am Amateurfunkdienst erforderlich.}
    {Es dürfen nur TKG-zugelassene Empfangsgeräte verwendet werden.}
    {Es bedarf der Zuteilung eines Hörerrufzeichens aus der ""DE-Reihe"".}
    {Die Anerkennung als ""SWL"" ist erforderlich in Verbindung mit der Mitgliedschaft in einer Amateurfunkvereinigung.}
    {false}{false}
    \frage{NF108}
    {Wie wird die Taste am Mikrofon bezeichnet, mit der man einen Transceiver auf Sendung schalten kann?}
    {PTT}
    {VOX}
    {RIT}
    {SSB}
    {false}{false}
\end{description}


%\begin{description}
%\frage{AC511}
    {Welcher der folgenden Transistoren ist ein selbstleitender P-Kanal-MOSFET?}
    {}
    {}
    {}
    {}
    {false}{true}
%\frage{AC511}
%    {Welche wesentlichen Ausgangsfrequenzen erzeugt die in der Abbildung dargestellte Stufe?}
%    {10,7 MHz und 52,7 MHz}
%    {42 MHz und 63,4 MHz}
%    {21 MHz und 63,4 MHz}
%    {21,4 MHz und 105,4 MHz}
%    {false}{true}
%\end{description}

%\begin{description}
%\frage{EF201}
%    {Welche wesentlichen Ausgangsfrequenzen erzeugt die in der Abbildung dargestellte Stufe?}
%    {10,7 MHz und 52,7 MHz}
%    {42 MHz und 63,4 MHz}
%    {21 MHz und 63,4 MHz}
%    {21,4 MHz und 105,4 MHz}
%    {true}{true}
%\end{description}



%  \path[draw=black,line cap=butt,line join=miter,line width=0.0211cm,miter limit=10.0,cm={ 0.9966,-0.0,-0.0,-0.9966,(-1.0061, 0.1794)}] (4.0527, -2.0411) -- (4.7877, -2.0411) -- (4.7877, -1.3061) -- (4.0527, -1.3061) -- cycle(4.0527, -2.0411);



  \path[draw=black,line cap=butt,line join=miter,line width=0.0105cm,miter limit=10.0,cm={ 0.9966,-0.0,-0.0,-0.9966,(-1.0061, 0.1794)}] (4.6775, -1.6736).. controls (4.6775, -1.8157) and (4.5624, -1.9309) .. (4.4203, -1.9309).. controls (4.2782, -1.9309) and (4.163, -1.8157) .. (4.163, -1.6736).. controls (4.163, -1.5315) and (4.2782, -1.4164) .. (4.4203, -1.4164).. controls (4.5624, -1.4164) and (4.6775, -1.5315) .. (4.6775, -1.6736) -- cycle(4.6775, -1.6736);



  \path[draw=black,line cap=butt,line join=miter,line width=0.0105cm,miter limit=10.0,cm={ 0.9966,-0.0,-0.0,-0.9966,(-1.0061, 0.1794)}] (4.2384, -1.8555) -- (4.6022, -1.4917)(4.6022, -1.8555) -- (4.2384, -1.4917);



  \path[draw=black,line cap=butt,line join=miter,line width=0.0105cm,miter limit=10.0,cm={ 0.9966,-0.0,-0.0,-0.9966,(-1.0061, 0.1794)}] (4.0527, -1.6736) -- (2.9477, -1.6736)(2.9477, -1.6736) -- (1.8427, -1.6736);



  \path[fill=black,nonzero rule] (2.9216, 1.8471) -- (2.9216, 1.7948) -- (3.0327, 1.8471) -- (2.9216, 1.8995) -- cycle(2.9216, 1.8471);



  \path[draw=black,line cap=butt,line join=miter,line width=0.0105cm,miter limit=10.0,cm={ 0.9966,-0.0,-0.0,-0.9966,(-1.0061, 0.1794)}] (4.7877, -1.6736) -- (5.8929, -1.6736)(5.8929, -1.6736) -- (6.9979, -1.6736);



  \path[fill=black,nonzero rule] (5.8565, 1.8471) -- (5.8565, 1.7948) -- (5.9677, 1.8471) -- (5.8565, 1.8995) -- cycle(5.8565, 1.8471);



  \path[draw=black,line cap=butt,line join=miter,line width=0.0105cm,miter limit=10.0,cm={ 0.9966,-0.0,-0.0,-0.9966,(-1.0061, 0.1794)}] (4.4203, -1.3061) -- (4.4203, -0.5694)(4.4203, -0.5694) -- (4.4203, 0.1673);



  \path[fill=black,nonzero rule] (3.399, 1.3697) -- (3.4513, 1.3697) -- (3.399, 1.4809) -- (3.3467, 1.3697) -- cycle(3.399, 1.3697);



  \begin{scope}[fill=black]
    \begin{scope}[fill=black,shift={(0.0788, -0.247)}]
      \path[fill=black] (0.0112, 2.7586) -- (0.0513, 2.8045).. controls (0.0667, 2.8224) and (0.0744, 2.8401) .. (0.0744, 2.8574).. controls (0.0744, 2.8794) and (0.0664, 2.8905) .. (0.0504, 2.8905).. controls (0.0443, 2.8905) and (0.0381, 2.8889) .. (0.0318, 2.8855).. controls (0.0299, 2.8791) and (0.0287, 2.8715) .. (0.0285, 2.8624).. controls (0.0266, 2.8621) and (0.0248, 2.862) .. (0.0232, 2.862).. controls (0.0157, 2.862) and (0.012, 2.865) .. (0.012, 2.8711).. controls (0.012, 2.8807) and (0.0157, 2.8884) .. (0.0232, 2.8942).. controls (0.0306, 2.9002) and (0.0407, 2.9033) .. (0.0537, 2.9033).. controls (0.0664, 2.9033) and (0.0767, 2.8996) .. (0.0847, 2.8922).. controls (0.0927, 2.8849) and (0.0967, 2.8748) .. (0.0967, 2.8616).. controls (0.0967, 2.8497) and (0.093, 2.8377) .. (0.0856, 2.8256).. controls (0.0808, 2.8178) and (0.0729, 2.8079) .. (0.0616, 2.7958) -- (0.0356, 2.7677) -- (0.0356, 2.7673) -- (0.0893, 2.7673) -- (0.0943, 2.788).. controls (0.0956, 2.7885) and (0.0973, 2.7888) .. (0.0992, 2.7888).. controls (0.1034, 2.7888) and (0.1054, 2.7857) .. (0.1054, 2.7797).. controls (0.1054, 2.7733) and (0.1046, 2.7623) .. (0.1029, 2.7466) -- (0.0132, 2.7466) -- cycle(0.0112, 2.7586);



    \end{scope}
    \begin{scope}[fill=black,shift={(0.197, -0.247)}]
      \path[fill=black] (0.0537, 2.7619) -- (0.0537, 2.8789) -- (0.0182, 2.8723).. controls (0.0179, 2.8742) and (0.0178, 2.8758) .. (0.0178, 2.8773).. controls (0.0178, 2.8808) and (0.0196, 2.8833) .. (0.0236, 2.8847) -- (0.0624, 2.9004) -- (0.0744, 2.9004) -- (0.0744, 2.7619) -- (0.1029, 2.7574).. controls (0.1054, 2.7571) and (0.1067, 2.7561) .. (0.1067, 2.7545).. controls (0.1067, 2.753) and (0.1056, 2.7505) .. (0.1038, 2.7466) -- (0.0227, 2.7466).. controls (0.0227, 2.7472) and (0.0227, 2.7479) .. (0.0227, 2.7487).. controls (0.0227, 2.7542) and (0.0252, 2.7574) .. (0.0302, 2.7582) -- cycle(0.0537, 2.7619);



    \end{scope}
  \end{scope}
  \begin{scope}[fill=black]
    \begin{scope}[fill=black,shift={(0.3544, -0.247)}]
      \path[fill=black] (0.1592, 2.7595) -- (0.1571, 2.8715) -- (0.1567, 2.8715) -- (0.1067, 2.7487) -- (0.0938, 2.7487) -- (0.0451, 2.8723) -- (0.0446, 2.8723) -- (0.0422, 2.7595) -- (0.0587, 2.7561).. controls (0.0609, 2.7558) and (0.062, 2.755) .. (0.062, 2.7537).. controls (0.062, 2.7528) and (0.0612, 2.7505) .. (0.0595, 2.7466) -- (0.0103, 2.7466).. controls (0.0103, 2.7472) and (0.0103, 2.7479) .. (0.0103, 2.7487).. controls (0.0103, 2.753) and (0.0129, 2.7559) .. (0.0182, 2.7574) -- (0.0285, 2.7595) -- (0.0302, 2.8971) -- (0.0157, 2.9004).. controls (0.0134, 2.9009) and (0.0124, 2.9019) .. (0.0124, 2.9033).. controls (0.0124, 2.9047) and (0.0132, 2.9068) .. (0.0149, 2.9099) -- (0.0521, 2.9099) -- (0.1038, 2.7797) -- (0.1042, 2.7797) -- (0.1567, 2.9099) -- (0.1968, 2.9099).. controls (0.1968, 2.9091) and (0.1968, 2.9084) .. (0.1968, 2.9079).. controls (0.1968, 2.9037) and (0.1941, 2.9009) .. (0.1889, 2.8996) -- (0.1778, 2.8971) -- (0.1798, 2.7595) -- (0.196, 2.7561).. controls (0.1981, 2.7558) and (0.1993, 2.755) .. (0.1993, 2.7537).. controls (0.1993, 2.7528) and (0.1984, 2.7505) .. (0.1968, 2.7466) -- (0.141, 2.7466).. controls (0.141, 2.7472) and (0.141, 2.7479) .. (0.141, 2.7487).. controls (0.141, 2.753) and (0.1436, 2.7559) .. (0.1488, 2.7574) -- cycle(0.1592, 2.7595);



    \end{scope}
    \begin{scope}[fill=black,shift={(0.5627, -0.247)}]
      \path[fill=black] (0.0686, 2.9099).. controls (0.0686, 2.9091) and (0.0686, 2.9084) .. (0.0686, 2.9079).. controls (0.0686, 2.9037) and (0.0659, 2.9009) .. (0.0608, 2.8996) -- (0.0496, 2.8971) -- (0.0496, 2.8372) -- (0.1286, 2.8372) -- (0.1286, 2.8971) -- (0.1141, 2.9004).. controls (0.1118, 2.9009) and (0.1108, 2.9019) .. (0.1108, 2.9033).. controls (0.1108, 2.9047) and (0.1116, 2.9068) .. (0.1133, 2.9099) -- (0.1683, 2.9099).. controls (0.1683, 2.9091) and (0.1683, 2.9084) .. (0.1683, 2.9079).. controls (0.1683, 2.9037) and (0.1656, 2.9009) .. (0.1604, 2.8996) -- (0.1492, 2.8971) -- (0.1492, 2.7595) -- (0.1654, 2.7561).. controls (0.1675, 2.7558) and (0.1687, 2.755) .. (0.1687, 2.7537).. controls (0.1687, 2.7528) and (0.1678, 2.7505) .. (0.1662, 2.7466) -- (0.1108, 2.7466).. controls (0.1105, 2.7472) and (0.1104, 2.7479) .. (0.1104, 2.7487).. controls (0.1104, 2.753) and (0.1131, 2.7559) .. (0.1186, 2.7574) -- (0.1286, 2.7595) -- (0.1286, 2.8231) -- (0.0496, 2.8231) -- (0.0496, 2.7595) -- (0.0661, 2.7561).. controls (0.0683, 2.7558) and (0.0695, 2.755) .. (0.0695, 2.7537).. controls (0.0695, 2.7528) and (0.0684, 2.7505) .. (0.0666, 2.7466) -- (0.0112, 2.7466).. controls (0.0109, 2.7472) and (0.0107, 2.7479) .. (0.0107, 2.7487).. controls (0.0107, 2.753) and (0.0134, 2.7559) .. (0.019, 2.7574) -- (0.0289, 2.7595) -- (0.0289, 2.8971) -- (0.0145, 2.9004).. controls (0.0122, 2.9009) and (0.0112, 2.9019) .. (0.0112, 2.9033).. controls (0.0112, 2.9047) and (0.012, 2.9068) .. (0.0136, 2.9099) -- cycle(0.0686, 2.9099);



    \end{scope}
    \begin{scope}[fill=black,shift={(0.7415, -0.247)}]
      \path[fill=black] (0.0124, 2.8368).. controls (0.0124, 2.8426) and (0.0129, 2.8504) .. (0.0141, 2.8603) -- (0.091, 2.8603).. controls (0.0915, 2.8587) and (0.0918, 2.857) .. (0.0918, 2.8554).. controls (0.0918, 2.8529) and (0.091, 2.8502) .. (0.0893, 2.8475) -- (0.0302, 2.759) -- (0.0302, 2.7586) -- (0.0765, 2.7586) -- (0.0847, 2.7822).. controls (0.0861, 2.7822) and (0.0873, 2.7822) .. (0.0885, 2.7822).. controls (0.0928, 2.7822) and (0.0951, 2.7798) .. (0.0951, 2.7752).. controls (0.0951, 2.7735) and (0.0939, 2.764) .. (0.0918, 2.7466) -- (0.0107, 2.7466).. controls (0.0091, 2.7491) and (0.0083, 2.752) .. (0.0083, 2.7553).. controls (0.0083, 2.7575) and (0.0091, 2.7599) .. (0.0107, 2.7623) -- (0.0678, 2.8479) -- (0.0678, 2.8488) -- (0.0289, 2.8488) -- (0.0223, 2.8289).. controls (0.0209, 2.8286) and (0.0198, 2.8285) .. (0.019, 2.8285).. controls (0.0146, 2.8285) and (0.0124, 2.8312) .. (0.0124, 2.8368) -- cycle(0.0124, 2.8368);



    \end{scope}
  \end{scope}
  \begin{scope}[fill=black]
    \begin{scope}[fill=black,shift={(6.0232, -0.247)}]
      \path[fill=black] (0.0364, 2.7921).. controls (0.0344, 2.7956) and (0.0335, 2.8014) .. (0.0335, 2.8095).. controls (0.0335, 2.8166) and (0.0361, 2.8241) .. (0.0413, 2.8318).. controls (0.0463, 2.8386) and (0.0513, 2.8454) .. (0.0562, 2.8521).. controls (0.0617, 2.8605) and (0.0645, 2.8694) .. (0.0645, 2.8785).. controls (0.0645, 2.8928) and (0.0568, 2.9) .. (0.0418, 2.9).. controls (0.0376, 2.9) and (0.0333, 2.8993) .. (0.0289, 2.8979).. controls (0.027, 2.8938) and (0.0252, 2.8868) .. (0.0236, 2.8769).. controls (0.0211, 2.8769) and (0.0194, 2.8769) .. (0.0186, 2.8769).. controls (0.0109, 2.8769) and (0.007, 2.8802) .. (0.007, 2.8868).. controls (0.007, 2.8939) and (0.0109, 2.9) .. (0.0186, 2.905).. controls (0.0263, 2.9101) and (0.0351, 2.9128) .. (0.0451, 2.9128).. controls (0.0563, 2.9128) and (0.0655, 2.9101) .. (0.0728, 2.905).. controls (0.0807, 2.8986) and (0.0847, 2.8894) .. (0.0847, 2.8773).. controls (0.0847, 2.8682) and (0.0815, 2.8591) .. (0.0752, 2.85).. controls (0.0695, 2.8428) and (0.0637, 2.8356) .. (0.0579, 2.8285).. controls (0.0515, 2.8202) and (0.0484, 2.8122) .. (0.0484, 2.8045).. controls (0.0484, 2.7987) and (0.0488, 2.7946) .. (0.0496, 2.7921) -- cycle(0.0554, 2.7566).. controls (0.0554, 2.7526) and (0.0542, 2.7495) .. (0.0517, 2.7471).. controls (0.0492, 2.7449) and (0.046, 2.7437) .. (0.0422, 2.7437).. controls (0.0386, 2.7437) and (0.0356, 2.7449) .. (0.0331, 2.7471).. controls (0.0308, 2.7495) and (0.0298, 2.7526) .. (0.0298, 2.7566).. controls (0.0298, 2.7604) and (0.0308, 2.7636) .. (0.0331, 2.7661).. controls (0.0356, 2.7686) and (0.0386, 2.7698) .. (0.0422, 2.7698).. controls (0.046, 2.7698) and (0.0492, 2.7686) .. (0.0517, 2.7661).. controls (0.0542, 2.7636) and (0.0554, 2.7604) .. (0.0554, 2.7566) -- cycle(0.0554, 2.7566);



    \end{scope}
  \end{scope}
  \begin{scope}[fill=black]
    \begin{scope}[fill=black,shift={(6.1567, -0.247)}]
      \path[fill=black] (0.1592, 2.7595) -- (0.1571, 2.8715) -- (0.1567, 2.8715) -- (0.1067, 2.7487) -- (0.0938, 2.7487) -- (0.0451, 2.8723) -- (0.0446, 2.8723) -- (0.0422, 2.7595) -- (0.0587, 2.7561).. controls (0.0609, 2.7558) and (0.062, 2.755) .. (0.062, 2.7537).. controls (0.062, 2.7528) and (0.0612, 2.7505) .. (0.0595, 2.7466) -- (0.0103, 2.7466).. controls (0.0103, 2.7472) and (0.0103, 2.7479) .. (0.0103, 2.7487).. controls (0.0103, 2.753) and (0.0129, 2.7559) .. (0.0182, 2.7574) -- (0.0285, 2.7595) -- (0.0302, 2.8971) -- (0.0157, 2.9004).. controls (0.0134, 2.9009) and (0.0124, 2.9019) .. (0.0124, 2.9033).. controls (0.0124, 2.9047) and (0.0132, 2.9068) .. (0.0149, 2.9099) -- (0.0521, 2.9099) -- (0.1038, 2.7797) -- (0.1042, 2.7797) -- (0.1567, 2.9099) -- (0.1968, 2.9099).. controls (0.1968, 2.9091) and (0.1968, 2.9084) .. (0.1968, 2.9079).. controls (0.1968, 2.9037) and (0.1941, 2.9009) .. (0.1889, 2.8996) -- (0.1778, 2.8971) -- (0.1798, 2.7595) -- (0.196, 2.7561).. controls (0.1981, 2.7558) and (0.1993, 2.755) .. (0.1993, 2.7537).. controls (0.1993, 2.7528) and (0.1984, 2.7505) .. (0.1968, 2.7466) -- (0.141, 2.7466).. controls (0.141, 2.7472) and (0.141, 2.7479) .. (0.141, 2.7487).. controls (0.141, 2.753) and (0.1436, 2.7559) .. (0.1488, 2.7574) -- cycle(0.1592, 2.7595);



    \end{scope}
    \begin{scope}[fill=black,shift={(6.365, -0.247)}]
      \path[fill=black] (0.0686, 2.9099).. controls (0.0686, 2.9091) and (0.0686, 2.9084) .. (0.0686, 2.9079).. controls (0.0686, 2.9037) and (0.0659, 2.9009) .. (0.0608, 2.8996) -- (0.0496, 2.8971) -- (0.0496, 2.8372) -- (0.1286, 2.8372) -- (0.1286, 2.8971) -- (0.1141, 2.9004).. controls (0.1118, 2.9009) and (0.1108, 2.9019) .. (0.1108, 2.9033).. controls (0.1108, 2.9047) and (0.1116, 2.9068) .. (0.1133, 2.9099) -- (0.1683, 2.9099).. controls (0.1683, 2.9091) and (0.1683, 2.9084) .. (0.1683, 2.9079).. controls (0.1683, 2.9037) and (0.1656, 2.9009) .. (0.1604, 2.8996) -- (0.1492, 2.8971) -- (0.1492, 2.7595) -- (0.1654, 2.7561).. controls (0.1675, 2.7558) and (0.1687, 2.755) .. (0.1687, 2.7537).. controls (0.1687, 2.7528) and (0.1678, 2.7505) .. (0.1662, 2.7466) -- (0.1108, 2.7466).. controls (0.1105, 2.7472) and (0.1104, 2.7479) .. (0.1104, 2.7487).. controls (0.1104, 2.753) and (0.1131, 2.7559) .. (0.1186, 2.7574) -- (0.1286, 2.7595) -- (0.1286, 2.8231) -- (0.0496, 2.8231) -- (0.0496, 2.7595) -- (0.0661, 2.7561).. controls (0.0683, 2.7558) and (0.0695, 2.755) .. (0.0695, 2.7537).. controls (0.0695, 2.7528) and (0.0684, 2.7505) .. (0.0666, 2.7466) -- (0.0112, 2.7466).. controls (0.0109, 2.7472) and (0.0107, 2.7479) .. (0.0107, 2.7487).. controls (0.0107, 2.753) and (0.0134, 2.7559) .. (0.019, 2.7574) -- (0.0289, 2.7595) -- (0.0289, 2.8971) -- (0.0145, 2.9004).. controls (0.0122, 2.9009) and (0.0112, 2.9019) .. (0.0112, 2.9033).. controls (0.0112, 2.9047) and (0.012, 2.9068) .. (0.0136, 2.9099) -- cycle(0.0686, 2.9099);



    \end{scope}
    \begin{scope}[fill=black,shift={(6.5438, -0.247)}]
      \path[fill=black] (0.0124, 2.8368).. controls (0.0124, 2.8426) and (0.0129, 2.8504) .. (0.0141, 2.8603) -- (0.091, 2.8603).. controls (0.0915, 2.8587) and (0.0918, 2.857) .. (0.0918, 2.8554).. controls (0.0918, 2.8529) and (0.091, 2.8502) .. (0.0893, 2.8475) -- (0.0302, 2.759) -- (0.0302, 2.7586) -- (0.0765, 2.7586) -- (0.0847, 2.7822).. controls (0.0861, 2.7822) and (0.0873, 2.7822) .. (0.0885, 2.7822).. controls (0.0928, 2.7822) and (0.0951, 2.7798) .. (0.0951, 2.7752).. controls (0.0951, 2.7735) and (0.0939, 2.764) .. (0.0918, 2.7466) -- (0.0107, 2.7466).. controls (0.0091, 2.7491) and (0.0083, 2.752) .. (0.0083, 2.7553).. controls (0.0083, 2.7575) and (0.0091, 2.7599) .. (0.0107, 2.7623) -- (0.0678, 2.8479) -- (0.0678, 2.8488) -- (0.0289, 2.8488) -- (0.0223, 2.8289).. controls (0.0209, 2.8286) and (0.0198, 2.8285) .. (0.019, 2.8285).. controls (0.0146, 2.8285) and (0.0124, 2.8312) .. (0.0124, 2.8368) -- cycle(0.0124, 2.8368);



    \end{scope}
  \end{scope}
  \begin{scope}[fill=black]
    \begin{scope}[fill=black,shift={(4.4002, -2.4339)}]
      \path[fill=black] (0.0368, 2.8301).. controls (0.0635, 2.8331) and (0.0769, 2.8438) .. (0.0769, 2.862).. controls (0.0769, 2.881) and (0.0688, 2.8905) .. (0.0529, 2.8905).. controls (0.0457, 2.8905) and (0.0395, 2.8886) .. (0.0343, 2.8851).. controls (0.0323, 2.8787) and (0.0314, 2.8716) .. (0.0314, 2.8636).. controls (0.0295, 2.863) and (0.0275, 2.8628) .. (0.0256, 2.8628).. controls (0.0182, 2.8628) and (0.0145, 2.8659) .. (0.0145, 2.8723).. controls (0.0145, 2.8814) and (0.0183, 2.8889) .. (0.026, 2.8946).. controls (0.034, 2.9004) and (0.0444, 2.9033) .. (0.0575, 2.9033).. controls (0.0701, 2.9033) and (0.08, 2.9) .. (0.0872, 2.8934).. controls (0.0944, 2.8868) and (0.098, 2.8777) .. (0.098, 2.8661).. controls (0.098, 2.8551) and (0.0943, 2.8461) .. (0.0868, 2.8392).. controls (0.0819, 2.8343) and (0.074, 2.8299) .. (0.0633, 2.8264) -- (0.0633, 2.826).. controls (0.0757, 2.8241) and (0.0853, 2.8198) .. (0.0922, 2.8132).. controls (0.099, 2.8068) and (0.1025, 2.7987) .. (0.1025, 2.7888).. controls (0.1025, 2.7758) and (0.0977, 2.765) .. (0.0881, 2.7566).. controls (0.0783, 2.748) and (0.0657, 2.7437) .. (0.05, 2.7437).. controls (0.0373, 2.7437) and (0.0278, 2.7463) .. (0.0215, 2.7512).. controls (0.0165, 2.7547) and (0.0141, 2.7591) .. (0.0141, 2.7644).. controls (0.0141, 2.7715) and (0.0175, 2.7764) .. (0.0244, 2.7789).. controls (0.0271, 2.7728) and (0.0298, 2.7683) .. (0.0322, 2.7657).. controls (0.0374, 2.7599) and (0.0448, 2.757) .. (0.0542, 2.757).. controls (0.0717, 2.757) and (0.0806, 2.7663) .. (0.0806, 2.7851).. controls (0.0806, 2.8046) and (0.0659, 2.8155) .. (0.0368, 2.8177) -- cycle(0.0368, 2.8301);



    \end{scope}
    \begin{scope}[fill=black,shift={(4.5184, -2.4339)}]
      \path[fill=black] (0.0537, 2.7619) -- (0.0537, 2.8789) -- (0.0182, 2.8723).. controls (0.0179, 2.8742) and (0.0178, 2.8758) .. (0.0178, 2.8773).. controls (0.0178, 2.8808) and (0.0196, 2.8833) .. (0.0236, 2.8847) -- (0.0624, 2.9004) -- (0.0744, 2.9004) -- (0.0744, 2.7619) -- (0.1029, 2.7574).. controls (0.1054, 2.7571) and (0.1067, 2.7561) .. (0.1067, 2.7545).. controls (0.1067, 2.753) and (0.1056, 2.7505) .. (0.1038, 2.7466) -- (0.0227, 2.7466).. controls (0.0227, 2.7472) and (0.0227, 2.7479) .. (0.0227, 2.7487).. controls (0.0227, 2.7542) and (0.0252, 2.7574) .. (0.0302, 2.7582) -- cycle(0.0537, 2.7619);



    \end{scope}
    \begin{scope}[fill=black,shift={(4.6366, -2.4339)}]
      \path[fill=black] (0.0236, 2.771).. controls (0.0279, 2.771) and (0.0315, 2.7692) .. (0.0343, 2.7657).. controls (0.0373, 2.762) and (0.0389, 2.757) .. (0.0389, 2.7504).. controls (0.0389, 2.7399) and (0.0356, 2.7312) .. (0.0289, 2.7243).. controls (0.0234, 2.7183) and (0.018, 2.7152) .. (0.0128, 2.7152).. controls (0.0103, 2.7152) and (0.0085, 2.7175) .. (0.0074, 2.7218).. controls (0.0176, 2.7271) and (0.0229, 2.735) .. (0.0236, 2.7454).. controls (0.0199, 2.7454) and (0.0169, 2.7465) .. (0.0145, 2.7487).. controls (0.012, 2.7512) and (0.0107, 2.7542) .. (0.0107, 2.7578).. controls (0.0107, 2.7619) and (0.0118, 2.765) .. (0.0141, 2.7673).. controls (0.0165, 2.7698) and (0.0196, 2.771) .. (0.0236, 2.771) -- cycle(0.0236, 2.771);



    \end{scope}
    \begin{scope}[fill=black,shift={(4.6856, -2.4339)}]
      \path[fill=black] (0.0128, 2.8698).. controls (0.0128, 2.8798) and (0.0133, 2.8899) .. (0.0145, 2.9004) -- (0.1091, 2.9004).. controls (0.11, 2.8993) and (0.1104, 2.8976) .. (0.1104, 2.8955).. controls (0.1104, 2.8922) and (0.1093, 2.8885) .. (0.1075, 2.8847) -- (0.0451, 2.7466) -- (0.0277, 2.7466) -- (0.0914, 2.8806) -- (0.0285, 2.8806) -- (0.0248, 2.8607).. controls (0.0232, 2.8599) and (0.0213, 2.8595) .. (0.0194, 2.8595).. controls (0.015, 2.8595) and (0.0128, 2.8629) .. (0.0128, 2.8698) -- cycle(0.0128, 2.8698);



    \end{scope}
  \end{scope}
  \begin{scope}[fill=black]
    \begin{scope}[fill=black,shift={(4.843, -2.4339)}]
      \path[fill=black] (0.1592, 2.7595) -- (0.1571, 2.8715) -- (0.1567, 2.8715) -- (0.1067, 2.7487) -- (0.0938, 2.7487) -- (0.0451, 2.8723) -- (0.0446, 2.8723) -- (0.0422, 2.7595) -- (0.0587, 2.7561).. controls (0.0609, 2.7558) and (0.062, 2.755) .. (0.062, 2.7537).. controls (0.062, 2.7528) and (0.0612, 2.7505) .. (0.0595, 2.7466) -- (0.0103, 2.7466).. controls (0.0103, 2.7472) and (0.0103, 2.7479) .. (0.0103, 2.7487).. controls (0.0103, 2.753) and (0.0129, 2.7559) .. (0.0182, 2.7574) -- (0.0285, 2.7595) -- (0.0302, 2.8971) -- (0.0157, 2.9004).. controls (0.0134, 2.9009) and (0.0124, 2.9019) .. (0.0124, 2.9033).. controls (0.0124, 2.9047) and (0.0132, 2.9068) .. (0.0149, 2.9099) -- (0.0521, 2.9099) -- (0.1038, 2.7797) -- (0.1042, 2.7797) -- (0.1567, 2.9099) -- (0.1968, 2.9099).. controls (0.1968, 2.9091) and (0.1968, 2.9084) .. (0.1968, 2.9079).. controls (0.1968, 2.9037) and (0.1941, 2.9009) .. (0.1889, 2.8996) -- (0.1778, 2.8971) -- (0.1798, 2.7595) -- (0.196, 2.7561).. controls (0.1981, 2.7558) and (0.1993, 2.755) .. (0.1993, 2.7537).. controls (0.1993, 2.7528) and (0.1984, 2.7505) .. (0.1968, 2.7466) -- (0.141, 2.7466).. controls (0.141, 2.7472) and (0.141, 2.7479) .. (0.141, 2.7487).. controls (0.141, 2.753) and (0.1436, 2.7559) .. (0.1488, 2.7574) -- cycle(0.1592, 2.7595);



    \end{scope}
    \begin{scope}[fill=black,shift={(5.0513, -2.4339)}]
      \path[fill=black] (0.0686, 2.9099).. controls (0.0686, 2.9091) and (0.0686, 2.9084) .. (0.0686, 2.9079).. controls (0.0686, 2.9037) and (0.0659, 2.9009) .. (0.0608, 2.8996) -- (0.0496, 2.8971) -- (0.0496, 2.8372) -- (0.1286, 2.8372) -- (0.1286, 2.8971) -- (0.1141, 2.9004).. controls (0.1118, 2.9009) and (0.1108, 2.9019) .. (0.1108, 2.9033).. controls (0.1108, 2.9047) and (0.1116, 2.9068) .. (0.1133, 2.9099) -- (0.1683, 2.9099).. controls (0.1683, 2.9091) and (0.1683, 2.9084) .. (0.1683, 2.9079).. controls (0.1683, 2.9037) and (0.1656, 2.9009) .. (0.1604, 2.8996) -- (0.1492, 2.8971) -- (0.1492, 2.7595) -- (0.1654, 2.7561).. controls (0.1675, 2.7558) and (0.1687, 2.755) .. (0.1687, 2.7537).. controls (0.1687, 2.7528) and (0.1678, 2.7505) .. (0.1662, 2.7466) -- (0.1108, 2.7466).. controls (0.1105, 2.7472) and (0.1104, 2.7479) .. (0.1104, 2.7487).. controls (0.1104, 2.753) and (0.1131, 2.7559) .. (0.1186, 2.7574) -- (0.1286, 2.7595) -- (0.1286, 2.8231) -- (0.0496, 2.8231) -- (0.0496, 2.7595) -- (0.0661, 2.7561).. controls (0.0683, 2.7558) and (0.0695, 2.755) .. (0.0695, 2.7537).. controls (0.0695, 2.7528) and (0.0684, 2.7505) .. (0.0666, 2.7466) -- (0.0112, 2.7466).. controls (0.0109, 2.7472) and (0.0107, 2.7479) .. (0.0107, 2.7487).. controls (0.0107, 2.753) and (0.0134, 2.7559) .. (0.019, 2.7574) -- (0.0289, 2.7595) -- (0.0289, 2.8971) -- (0.0145, 2.9004).. controls (0.0122, 2.9009) and (0.0112, 2.9019) .. (0.0112, 2.9033).. controls (0.0112, 2.9047) and (0.012, 2.9068) .. (0.0136, 2.9099) -- cycle(0.0686, 2.9099);



    \end{scope}
    \begin{scope}[fill=black,shift={(5.23, -2.4339)}]
      \path[fill=black] (0.0124, 2.8368).. controls (0.0124, 2.8426) and (0.0129, 2.8504) .. (0.0141, 2.8603) -- (0.091, 2.8603).. controls (0.0915, 2.8587) and (0.0918, 2.857) .. (0.0918, 2.8554).. controls (0.0918, 2.8529) and (0.091, 2.8502) .. (0.0893, 2.8475) -- (0.0302, 2.759) -- (0.0302, 2.7586) -- (0.0765, 2.7586) -- (0.0847, 2.7822).. controls (0.0861, 2.7822) and (0.0873, 2.7822) .. (0.0885, 2.7822).. controls (0.0928, 2.7822) and (0.0951, 2.7798) .. (0.0951, 2.7752).. controls (0.0951, 2.7735) and (0.0939, 2.764) .. (0.0918, 2.7466) -- (0.0107, 2.7466).. controls (0.0091, 2.7491) and (0.0083, 2.752) .. (0.0083, 2.7553).. controls (0.0083, 2.7575) and (0.0091, 2.7599) .. (0.0107, 2.7623) -- (0.0678, 2.8479) -- (0.0678, 2.8488) -- (0.0289, 2.8488) -- (0.0223, 2.8289).. controls (0.0209, 2.8286) and (0.0198, 2.8285) .. (0.019, 2.8285).. controls (0.0146, 2.8285) and (0.0124, 2.8312) .. (0.0124, 2.8368) -- cycle(0.0124, 2.8368);



    \end{scope}
  \end{scope}

% svg2tikz --codeonly --output=bild.tex bild.svg
% svg2tikz --codeonly --output=bild.tex bild.svg
\end{document}
