\setcounter{section}{14}
\section{Personenschutzabstand}
\subsection{Äquivalente isotrope Strahlungsleistung (EIRP) II}

\begin{sol}{EG501}
Ich finde diese Art von Fragen schwierig, da die Antworten alle ähnlich klingen. 
Es hilft zunächst alle Antworten zu lesen und zu festzustellen wo die Unterschiede sind.
Es geht um EIRP und an dem Buchstaben I sollte sofort klar sein, dass es um den isotropen Strahler geht, d.h. Antwort (B) und (D) können sofort ausgeschlossen werden, da es hier um einen Dipol geht. Jetzt schauen wir uns Antwort (C) an und bemerken, dass es um die \qq{höchste Spitze der Modulationshüllkurve} geht. Das hat was PEP (Peak Envelop Power) zu tun ist in der Tat eine falsche Antwort. Also ist (A) richtig.
\end{sol}

\begin{sol}{EG502}
Wie in Frage \qref{EG501} geht es um EIRP also können nur die Antworten mit \qq{isotropen Streahler} richtig sein, dies sind (A) imd (C). Wir stellen fest, dass der Gewinn $G_\text{Antenne}$ einmal multipliziert wird und einem addiert. Multiplizieren ist hier Richtig, da $G$ als Gewinnfaktor verwendet wird (siehe z.B. Formelsymbole in der Formelsammlung)
\end{sol}


\begin{sol}{EG503}
Diese Frage ist viel einfacher zu lösen wie es zunächst aussieht.
Es geht um einen Parabolspiegel mit $\SI{26}{dBi} = \SI{20}{dBi} + \SI{6}{dBi}$.
Warum den Gewinn gerade so aufteilen? In der Formelsammlung finden wir im Abschnitt Pegel Angaben: Der Gewinnfaktor (Leistung) von \SI{20}{dB} entspricht Faktor 100. Der Gewinnfaktor von \SI{6}{dBi} ist 4. Zusammen ergibt sich also der Gewinnfaktor von $100 \cdot 4 = 400$. Erinnere Dich daran: Die Addition in der dB-Rechnung entspricht der Multiplikation wenn wir mit entsprechenden Gewinnfaktoren arbeiten. 
Also werden die \SI{250}{\milli\watt} um einen Antennengewinn von Faktor 400 verstärkt.
$$ \SI{250}{\milli\watt} \cdot 400 = \SI{100}{\watt} $$
\end{sol}


\begin{sol}{EG504}
    Analog zu \qref{EG503} zerlegen wir die $\SI{36}{dBi} =  \SI{20}{dB} + \SI{10}{dB} + \SI{6}{dB}$.
    Mit den Gewinnfaktoren aus der Umrechnungstabelle in der Formelsammlung ergibt sich: $100 \cdot 10 \cdot 4 = 4000$.
    Also:
    $$ \SI{5}{\watt} \cdot 4000 = \SI{20000}{\watt} $$
\end{sol}

\begin{sol}{EG505}
    Wir haben \SI{11}{dBi} Antennengewinn. Davon können wir aber \SI{1}{\dB} Dämpfung abziehen. In der Formelsammlung (Kapitel Pegel) entspricht dies einem Faktor 10.
    Also $$ \SI{100}{\watt} \cdot 10 = \SI{1000}{\watt} $$

\end{sol}

\begin{sol}{EG506}
Wir haben \SI{75}{\watt} Ausgangsleistung. 
%und rechnen mit Faktoren. Unter Dipol hat \SI{2,15}{dBi} Gewinn. Dies entspricht einem Faktor $1,64$. 
Die Kabelverluste entsprechen ganz genau dem Antennengewinn eines Dipol.
Diese Angaben über den Dipol sind in der Formelsammlung (Strahlungsleistung und Gewinn von Antennen).

Da sich Gewinn und Verlust ausgleichen kommen auch \SI{70}{\watt} EIRP heraus.
\end{sol}

\begin{sol}{EG507}
 Die \SI{10}{\dB} Dämpfung entsprechen einem Faktor von 10. D.h. \SI{100}{\watt} Ausgangsleistung werden zu \SI{10}{\watt} an der Antenne. Es handelt sich um einen Dipol also müssen wir noch den Antennengewinn mit dem Faktor 1,64 berücksichtigen und bekommen also $\SI{10}{\watt} \cdot 1,64 = \SI{16,4}{\watt}$ heraus.
\end{sol}

\begin{sol}{EG508}
    Ähnlich zu Frage \qref{EG507}!
    Aus Kebelverlust und Antennengewinn haben wir \SI{3}{\dB} also ein Faktor 2 laut der Umrechnungstabelle. Aus \SI{5}{\watt} werden mit dem Faktor 2 also \SI{10}{\watt}.
    Jetzt aber aufpassen: der Antennengewinn was in dB (also in Bezug auf Dipol) angegeben. Um auf EIRP (Bezug Isotropenstrahler) zu kommen müssen wir den Gewinn des Dipols mit dem Faktor 1,64 berücksichtigen. Das Ergebnis ist also auch $\SI{10}{\watt} \cdot 1,64 = \SI{16,4}{\watt}$.
\end{sol}

\begin{sol}{EG509}
    Aus Antennengewinn und Kebelverlust ergeben sind \SI{10}{\dB}, also Faktor 10 Gewinn. Aus \SI{0,6}{\watt} werden also \SI{6}{\watt}. Wir berücksichtigen aber noch den Dipol Gewinnfaktor von 1,64 und erhalten: $\SI{6}{\watt} \cdot 1,64 = \SI{9,84}{\watt}$
\end{sol}

\begin{sol}{EG510}
    Ich gebe zwei mögliche Lösungen an. Der Lösungsweg auf der 50 Ohm Webseite halte ich für unnötig kompliziert!

    Zunächst haben wir \SI{1,5}{\dB} Kabelverlust aber einen Dipol mit Gewinn \SI{2,15}{\dB}. In Summe bleiben nur \SI{0,64}{\dB} Gewinn über.

    Zunächst die Lösung mit \textbf{Überschlagsrechnung}:
    Schauen wir in der Tabelle die Gewinnfaktoren von 0 dB und 1 dB nach so entspricht dies Faktor 1 bis 1,26. Es gibt aber nur die Antwort (A) die sich aus \SI{8,5}{\watt} Ausgangsleistung und solch einem Faktor nahe 1 ergeben könnte.

    Eine \textbf{exakte Lösung}:    
    In der Formelsammlung im Kapitel \qq{Strahlungsleistung und Gewinn von Antennen} finden wir die Formel $$ G = 10^\frac{g}{\SI{10}{\dB}}$$. Wir setzten ein um den exakten Gewinnfaktor zu errechnen:
    $$G = 10^\frac{0,64}{\SI{10}{\dB}} \approx 1,16 $$.
    $$\SI{8,5}{\watt} \cdot 1,16 = \SI{9,86}{\watt}$$

\end{sol}

\begin{sol}{EG511}
    Dies ist vermutlich eine der Kompliziertesten Fragen im ganzen Katalog der Klasse E. Wenn Du Dir die Antwort (A) mit \SI{3}{\watt} merkst, ist dies sicherlich geschickt.

    Dennoch geben wir wie bei Frage \qref{EG510} wieder zwei Lösungen an:

    Zunächst die Lösung mit \textbf{Überschlagsrechnung}:
    Die \SI{5,15}{dBi} sind annähernd \SI{5}{dBi} und die können wir zerlegen in $\SI{5}{dBi} = \SI{3}{dBi} + \SI{1}{dBi} + \SI{1}{dBi}$. Als Gewinnfaktor haben wir also etwa: $2*1,26*1,26=3,1752$. 
    Rechnen wir von den maximal \SI{10}{\watt} EIRP zurück, ergibt sich: $$10/3,1752 \approx \SI{3,1494}{\watt}$$ die wir maximal aussengen können.
    D.h., auch wenn das Ergebnis denkbar knapp ist, \SI{3}{\watt} können wir senden.
    
    Eine \textbf{exakte Lösung}:   
    Wir verwenden die Formel aus Frage \qref{EG510} um den Gewinnfaktor exakt zu berechnen.
    $$G = 10^\frac{5,15}{\SI{10}{\dB}} \approx 3,273 $$
    Mit dem Gewinnfaktor können wir die maximale Ausgangsleistung mit $10 / 3,273 \approx 3,0553$ ermitteln.
    Knapp aber 3 Watt sollten gerade noch unter dem Grenzwert liegen.
\end{sol}


\begin{ohmchapter}
 
\end{ohmchapter}


\subsection{Personenschutzabstand II }

\begin{sol}{EK104}
    Nach all den Fragen in vorherigen Kapitel, sollte unmittelbar klar sein, dass der enorme Antennengewinn von 13 dBd der Yagi Uda Antenne. Problemlos ausreichen um bei 6 W Ausgangsleistung über den Grenzwert zu kommen. Selbst wenn wir den Gewinn mit 10 dBd etwas abrunden und den Gewinn des Dipols unberücksichtigt lassen, sind mit (Gewinnfaktor 10) schon über 10 W EIRP.
    D.h. ohne Rechnung ist bereits klar, dass wir in dieser Situation die Einhaltung der Personenschutzgrenzwerte nachzuweisen haben.

\end{sol}

\begin{sol}{EK107}
Eine Frage der Vorschriften die Du lernen musst, die aber auch Sinn macht.

\end{sol}


\begin{ohmchapter}
 Es geht in diesem Kapitel mal wieder um die 10 W EIRP Strahlungsleistung.
\end{ohmchapter}


\subsection{Grenzwerte }

\begin{sol}{EK101}
Die Antwort sollte klar sein. 
\end{sol}
\begin{ohmchapter}
 In diesem Kapitel befassen wir und mit Fragen wie wird den Personenschutz-Sicherheitsabstand bestimmen können, bzw. wann eine Näherungsformel für die Fernfeldberechnung zum Einsatz kommen kann.
\end{ohmchapter}

\subsection{Näherungsformel I}

\begin{ohmchapter}
 In diesem Kapitel befassen wir und mit Fragen wie wird den Personenschutz-Sicherheitsabstand bestimmen können, bzw. wann eine Näherungsformel für die Fernfeldberechnung zum Einsatz kommen kann.
\end{ohmchapter}
