\setcounter{section}{14}
\section{Personenschutzabstand}
\subsection{Äquivalente isotrope Strahlungsleistung (EIRP) II}

\begin{sol}{EG501}
Ich finde diese Art von Fragen schwierig, da die Antworten alle ähnlich klingen. 
Es hilft zunächst alle Antworten zu lesen und zu festzustellen wo die Unterschiede sind.
Es geht um EIRP und an dem Buchstaben I sollte sofort klar sein, dass es um den isotropen Strahler geht, d.h. Antwort (B) und (D) können sofort ausgeschlossen werden, da es hier um einen Dipol geht. Jetzt schauen wir uns Antwort (C) an und bemerken, dass es um die \qq{höchste Spitze der Modulationshüllkurve} geht. Das hat was PEP (Peak Envelop Power) zu tun ist in der Tat eine falsche Antwort. Also ist (A) richtig.
\end{sol}

\begin{sol}{EG502}
Wie in Frage \qref{EG501} geht es um EIRP also können nur die Antworten mit \qq{isotropen Streahler} richtig sein, dies sind (A) imd (C). Wir stellen fest, dass der Gewinn $G_\text{Antenne}$ einmal multipliziert wird und einem addiert. Multiplizieren ist hier Richtig, da $G$ als Gewinnfaktor verwendet wird (siehe z.B. Formelsymbole in der Formelsammlung)
\end{sol}


\begin{ohmchapter}
 
\end{ohmchapter}