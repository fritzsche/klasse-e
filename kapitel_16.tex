\setcounter{section}{15}
\section{Sicherheit}

\subsection{Öffnen elektrischer Geräte I}


\begin{sol}{EK203}
  Diese Frage wäre ohne Gefahren sicher nicht im Katalog!
  Ohne Strom vom Netz können in Kondensatoren aber immer noch sehr hohe Ladungen gespeichert sein.   
\end{sol}

\begin{ohmchapter}
\end{ohmchapter}

\subsection{Blitzerdung}


\begin{sol}{EK209}
Beim Blitzschutz kein Unterschied zu anderen Anlagen.
\end{sol}


\begin{sol}{EK210}
Auswendig lernen.
\end{sol}

\begin{sol}{EK211}
Schlüsselwort: Blitzschutzkonzept.
\end{sol}

\begin{ohmchapter}
Antennenanlagen erhöhen in der Regel nicht die Wahrscheinlichkeit, dass ein Blitz einschlägt, aber wenn er einschlägt wird vermutlich eine vorhandene exponierte Antenne das Ziel sein.
Deshalb müssen Antennenanlagen auf oder an Gebäuden geerdet bzw. in ein vorhandenes Blitzschutzsystem integriert werden.    
\end{ohmchapter}

\subsection{Schutzerdung und Potentialausgleich I}


\begin{sol}{EK208}
Bei Koaxialkabel ist der Schutz recht einfach zu realisieren. Man verbindet die Schirme aller Koaxialkabel miteinander und schließt sie zusätzlich an die Haupterdungsschiene an.
\end{sol}
\begin{ohmchapter}
\end{ohmchapter}


\subsection{Statische Aufladung von Antennen}

\begin{sol}{EK206}
Schon Regen und Hagel.
\end{sol}

\begin{sol}{EK207}
Wir können Ableitwiederstände verwenden. Müssen hochohmig sein, um die Antenne nicht zu beeinflussen.
\end{sol}


\begin{ohmchapter}
\end{ohmchapter}

\subsection{Berühren von Antennen I}

\begin{sol}{EK202}
 Ohne Gefahr gäbe es diese Frage nicht im Katalog, wir schließen also Antwort (B) und (C) sofort aus.
 Antwort (D) kann es auch nicht sein, da wir der Antenne keinen Gleichspannung zuführen. 
 Antwort (A) klingt auch direkt logisch.
\end{sol}


\begin{ohmchapter}
\end{ohmchapter}

\subsection{Aufenthalt im Strahlengang}

\begin{sol}{EK201}
Du kannst Dich vielleicht noch an den enorm hohen Antennengewinn von Parabolspiegeln erinnern.
Jeder der zuhause eine Handelsübliche Mikrowelle besitzt sollte also die Antwort direkt auswählen können.
\end{sol}

\begin{ohmchapter}
\end{ohmchapter}