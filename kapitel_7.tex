\setcounter{section}{6}
\section{Strom- und Spannungsversorgung}

\subsection{Spannungsquellen}

\begin{sol}{ED301}
    Funkgeräte benötigen oft sehr schnell einen hohen Strom, z.B. wenn wir in CW sehr schnell einen Träger ein und ausschalten.
    Dabei ist es sehr wichtig, dass die Gleichspannungsquelle sehr schnell reagiert und dabei die Spannung konstant hält.     
\end{sol}


\begin{sol}{EK205}
  Du hast bestimmt schon so eine Kabel gesehen (siehe Bild am Anfang des Kapitel), wenn Du z.B. eine Lampe montiert hast.
  Das braune Kabel ist, dass eigentliche Stromführende Kabel gegenüber dem Neutralleiter. Dies wir auch als Phase genannt.
  Wenn Du Farben kennst bleiben nur noch Antwort (C) und (A). Nur in Antwort (A) ist richtig. 
\end{sol}


\begin{ohmchapter}
   Normgerechte Adernfarben von 3-adrigen, isolierten Energieleitungen:
   \begin{description}
    \item[Neutralleiter (N)] blau
    \item[Schutzleiter (PE)] grüngelb
    \item[Außenleiter (L)] braun  
   \end{description}
  Eine solte 3-adrige Energieleitung sieht etwas folgendermaßen aus:
  \begin{center}
    \includegraphics[scale=.07]{bilder/energieleitung.png}    
  \end{center}
\end{ohmchapter}


\subsection{Gleichrichter I}
\begin{sol}{ED304} 
    In dem Schaltplan sehen wir einen Transformator, eine Diode und einen Lastwiderstand.
    Auf der linken Seite sehen wir eine Sinuskurve die offenbar den Spannungsverlauf wiedergibt.
    Nach dem Transformator haben wir immer noch einen Sinus. Dann kommt die Diode, hier wir jeweils bei einer Hälfte des Sinus in Sperrrichtung betrieben wird und deshalb keine Spannung anliegt. Dies entspricht Bild (A).
\end{sol}

\begin{ohmchapter}
\end{ohmchapter}


\subsection{Schaltnetzteil I}
\begin{sol}{ED302} 
  Ein Schaltnetzteil ist klein, hat relativ kleines Gewicht und eine hohe Effizienz.
\end{sol}

\begin{sol}{ED303} 
  Der Hauptnachteil eines Schaltnetzteils sind hochfrequente Störungen die durch den Schaltvorgang erzeugt werden, z.B. Oberwellen.
  Ein Schaltnetzteil sollte deshalb für den Amateurfunk ausgelegt sein und diese Störungen auf eine vom Amateurfunk ungenutzte Frequenz schieben und die so gut es geht filtern.
\end{sol}


\begin{ohmchapter}
\end{ohmchapter}


\subsection{Sicherungen}

\begin{sol}{EK204} 
  Es ist sehr wichtig, dass Stromwert und Auslösecharakteristik verwendet wird, damit die Scherung nicht zu früh oder zu spät auslöst.
\end{sol}

\begin{ohmchapter}
\end{ohmchapter}
